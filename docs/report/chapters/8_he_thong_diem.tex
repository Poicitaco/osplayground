% ============================================================================
% CHƯƠNG 8: HỆ THỐNG ĐIỂM VÀ ĐÁNH GIÁ
% ============================================================================

\chapter{Hệ thống điểm và đánh giá}
\label{ch:scoring}

\section{Khái niệm tổng quát}

OSPlayground sử dụng hệ thống điểm tổng hợp (System Stability Score) để đánh giá 
toàn bộ hiệu năng của mô phỏng hệ thống. Điểm được tính dựa trên ba thành phần:

\begin{enumerate}[leftmargin=1.5cm]
  \item \textbf{CPU Score}: Đánh giá hiệu năng lập lịch CPU
  \item \textbf{Memory Score}: Đánh giá hiệu năng quản lý bộ nhớ
  \item \textbf{IPC Score}: Đánh giá ảnh hưởng của giao tiếp liên tiến trình
\end{enumerate}

Các điểm thành phần được kết hợp với trọng số để tạo ra System Stability Score:

\begin{equation}
\text{TOTAL} = 0.4 \times \text{CPU} + 0.4 \times \text{Memory} + 0.2 \times \text{IPC}
\end{equation}

\section{CPU Score}

\subsection{Công thức}

\begin{equation}
\text{CPU Score} = 100 \cdot \frac{B}{B + (\text{AWT} + \text{ATT})}, \quad B = 20
\end{equation}

Trong đó:
\begin{itemize}[leftmargin=1.5cm]
  \item $B = 20$ là hằng số chuẩn hoá
  \item AWT = Average Waiting Time
  \item ATT = Average Turnaround Time
\end{itemize}

\subsection{Giải thích}

Công thức này được thiết kế để:
\begin{enumerate}[leftmargin=1.5cm]
  \item Khuyến khích AWT và ATT thấp
  \item Giữ điểm trong khoảng [0, 100]
  \item Đạt 50 điểm khi $(AWT + ATT) = B = 20$
  \item Đạt 100 điểm khi AWT + ATT = 0 (lý tưởng)
\end{enumerate}

\subsection{Ví dụ}

\begin{table}[H]
\centering
\caption{Ví dụ tính CPU Score}
\begin{tabular}{cccc}
\toprule
\textbf{AWT} & \textbf{ATT} & \textbf{AWT+ATT} & \textbf{CPU Score} \\
\midrule
0 & 0 & 0 & 100.0 \\
5 & 5 & 10 & 66.7 \\
10 & 10 & 20 & 50.0 \\
15 & 15 & 30 & 40.0 \\
20 & 20 & 40 & 33.3 \\
\bottomrule
\end{tabular}
\end{table}

\subsection{Cài đặt}

\begin{lstlisting}[language=C++,caption={CPU Score Computation}]
double computeCPUScore(double awt, double att) {
  const double B = 20.0;
  double denominator = awt + att;
  
  if (denominator == 0) {
    return 100.0;  // Perfect case
  }
  
  double score = 100.0 * B / (B + denominator);
  return min(score, 100.0);  // Cap at 100
}
\end{lstlisting}

\section{Memory Score}

\subsection{Công thức}

\begin{equation}
\text{Memory Score} = 100 \times \left(1 - \frac{\text{faults}}{\text{steps}}\right)
\end{equation}

Trong đó:
\begin{itemize}[leftmargin=1.5cm]
  \item \texttt{faults} = số lần page fault xảy ra
  \item \texttt{steps} = tổng số memory references
\end{itemize}

\subsection{Giải thích}

\begin{itemize}[leftmargin=1.5cm]
  \item Fault rate = faults / steps (tỷ lệ lỗi trang)
  \item Hit rate = 1 - fault rate (tỷ lệ thành công)
  \item Memory Score = Hit rate × 100
  \item Điểm 100 khi không có fault (tất cả hit)
  \item Điểm 0 khi fault rate = 100\%
\end{itemize}

\subsection{Ví dụ}

\begin{table}[H]
\centering
\caption{Ví dụ tính Memory Score}
\begin{tabular}{ccccc}
\toprule
\textbf{Faults} & \textbf{Steps} & \textbf{Hit Rate} & \textbf{Score} \\
\midrule
0 & 100 & 100\% & 100.0 \\
10 & 100 & 90\% & 90.0 \\
20 & 100 & 80\% & 80.0 \\
50 & 100 & 50\% & 50.0 \\
100 & 100 & 0\% & 0.0 \\
\bottomrule
\end{tabular}
\end{table}

\subsection{Cài đặt}

\begin{lstlisting}[language=C++,caption={Memory Score Computation}]
double computeMemoryScore(int faults, int steps) {
  if (steps == 0) {
    return 100.0;  // No simulation
  }
  
  double fault_rate = static_cast<double>(faults) / steps;
  double hit_rate = 1.0 - fault_rate;
  double score = hit_rate * 100.0;
  
  return max(0.0, min(score, 100.0));
}
\end{lstlisting}

\section{IPC Score}

\subsection{Công thức}

\begin{equation}
\text{IPC Score} = \max\left(0, 100 - 15 \times \text{kills}\right)
\end{equation}

Trong đó:
\begin{itemize}[leftmargin=1.5cm]
  \item \texttt{kills} = số lần SIGKILL được gửi
  \item Hệ số 15 = penalty per kill (có thể tuỳ chỉnh)
\end{itemize}

\subsection{Giải thích}

\begin{itemize}[leftmargin=1.5cm]
  \item Mục tiêu: khuyến khích sử dụng signal có trách nhiệm
  \item Mỗi SIGKILL làm giảm 15 điểm
  \item Khi kills ≥ 7, điểm trở thành 0
  \item SIGINT, SIGUSR1 không phạt (chỉ là thông báo)
\end{itemize}

\subsection{Ví dụ}

\begin{table}[H]
\centering
\caption{Ví dụ tính IPC Score}
\begin{tabular}{cc}
\toprule
\textbf{Kills} & \textbf{IPC Score} \\
\midrule
0 & 100.0 \\
1 & 85.0 \\
2 & 70.0 \\
3 & 55.0 \\
4 & 40.0 \\
5 & 25.0 \\
6 & 10.0 \\
7+ & 0.0 \\
\bottomrule
\end{tabular}
\end{table}

\subsection{Cài đặt}

\begin{lstlisting}[language=C++,caption={IPC Score Computation}]
double computeIPCScore(int kills) {
  const int PENALTY_PER_KILL = 15;
  double score = 100.0 - (kills * PENALTY_PER_KILL);
  return max(0.0, min(score, 100.0));
}
\end{lstlisting}

\section{System Stability Score}

\subsection{Công thức tổng hợp}

\begin{equation}
\text{System Stability Score} = 0.4 \times \text{CPU} + 0.4 \times \text{Memory} + 0.2 \times \text{IPC}
\end{equation}

\subsection{Giải thích trọng số}

\begin{itemize}[leftmargin=1.5cm]
  \item \textbf{CPU (40\%)}: Lập lịch CPU là khía cạnh chính của OS
  \item \textbf{Memory (40\%)}: Quản lý bộ nhớ cũng rất quan trọng
  \item \textbf{IPC (20\%)}: Giao tiếp liên tiến trình ít hơn nhưng cần xem xét
\end{itemize}

\subsection{Ví dụ}

\begin{table}[H]
\centering
\caption{Ví dụ tính System Stability Score}
\begin{tabular}{ccccc}
\toprule
\textbf{CPU} & \textbf{Memory} & \textbf{IPC} & \textbf{Calculation} & \textbf{Total} \\
\midrule
100 & 100 & 100 & $0.4 \times 100 + 0.4 \times 100 + 0.2 \times 100$ & 100.0 \\
80 & 90 & 70 & $0.4 \times 80 + 0.4 \times 90 + 0.2 \times 70$ & 82.0 \\
50 & 50 & 50 & $0.4 \times 50 + 0.4 \times 50 + 0.2 \times 50$ & 50.0 \\
100 & 0 & 100 & $0.4 \times 100 + 0.4 \times 0 + 0.2 \times 100$ & 60.0 \\
\bottomrule
\end{tabular}
