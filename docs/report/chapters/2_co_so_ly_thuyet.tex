% ============================================================================
% CHƯƠNG 2: CƠ SỞ LÝ THUYẾT
% ============================================================================

\chapter{Cơ sở lý thuyết}
\label{ch:theory}

\section{Khái niệm cơ bản về hệ điều hành}

\subsection{Tiến trình (Process), luồng (Thread) và bối cảnh (Context)}

\textbf{Tiến trình (Process)} là một chương trình đang thực thi. Mỗi tiến trình có:
\begin{itemize}[leftmargin=1.5cm]
  \item \textbf{Bộ nhớ}: Bao gồm code, data, heap, stack
  \item \textbf{Trạng thái CPU}: Giá trị các thanh ghi
  \item \textbf{Thông tin bối cảnh}: PID (Process ID), cha mẹ, ưu tiên, ...
  \item \textbf{Tài nguyên}: Tập tin mở, socket, bộ nhớ chia sẻ, ...
\end{itemize}

\textbf{Luồng (Thread)} là đơn vị lập lịch nhỏ hơn, nằm bên trong tiến trình. 
Các luồng cùng tiến trình chia sẻ bộ nhớ nhưng có stack và thanh ghi riêng.

\textbf{Bối cảnh (Context)} là tập hợp thông tin CPU cần để tiếp tục thực thi một 
tiến trình: thanh ghi, con trỏ chương trình (PC), con trỏ ngăn xếp (SP), ...

\textbf{Chuyển ngữ cảnh (Context Switch)} xảy ra khi CPU chuyển từ tiến trình này sang 
tiến trình khác. Việc chuyển đổi này có chi phí (overhead) vì phải lưu/khôi phục trạng thái.

\subsection{Trạng thái tiến trình}

Trong mô phỏng, một tiến trình có thể ở các trạng thái:

\begin{itemize}[leftmargin=1.5cm]
  \item \textbf{New}: Tiến trình vừa được tạo, chưa sẵn sàng
  \item \textbf{Ready}: Tiến trình sẵn sàng chạy, chờ CPU
  \item \textbf{Running}: Tiến trình đang chạy trên CPU
  \item \textbf{Blocked}: Tiến trình chờ I/O hoặc tài nguyên
  \item \textbf{Terminated}: Tiến trình kết thúc
\end{itemize}

Trong dự án này, vì không mô phỏng I/O nên chủ yếu quan tâm đến Ready, Running, Terminated.

\section{Lập lịch CPU (CPU Scheduling)}

\subsection{Mục tiêu của lập lịch CPU}

Lập lịch CPU (CPU Scheduling) quyết định tiến trình nào được chạy trên CPU tại thời điểm nào. 
Mục tiêu chính:

\begin{itemize}[leftmargin=1.5cm]
  \item \textbf{Độ sử dụng CPU cao}: Giữ CPU luôn bận rộn, giảm thời gian rỗi
  \item \textbf{Throughput cao}: Hoàn thành nhiều tiến trình trong một đơn vị thời gian
  \item \textbf{Turnaround time thấp}: Giảm thời gian tổng để hoàn thành một tiến trình
  \item \textbf{Waiting time thấp}: Giảm thời gian chờ trong hàng đợi
  \item \textbf{Response time thấp}: Phản ứng nhanh đối với yêu cầu tương tác
  \item \textbf{Công bằng}: Mỗi tiến trình được cơ hội công bằng trên CPU
\end{itemize}

Các mục tiêu này thường xung đột với nhau, vì vậy mỗi thuật toán lập lịch là một 
thỏa hiệp giữa chúng.

\subsection{Các chỉ số đo lường}

Đối với tiến trình $i$, đặt:
\begin{itemize}[leftmargin=1.5cm]
  \item $A_i$ = arrival time (thời điểm đến)
  \item $B_i$ = burst time (thời gian CPU cần)
  \item $S_i$ = start time (lúc bắt đầu chạy)
  \item $F_i$ = finish time (lúc hoàn thành)
\end{itemize}

Khi đó:
\begin{align}
WT_i &= S_i - A_i & \text{(Waiting Time)} \\
TT_i &= F_i - A_i = WT_i + B_i & \text{(Turnaround Time)} \\
RT_i &= TT_i - B_i & \text{(Response Time)}
\end{align}

Các chỉ số trung bình:
\begin{align}
AWT &= \frac{1}{n}\sum_{i=1}^{n} WT_i & \text{(Average Waiting Time)} \\
ATT &= \frac{1}{n}\sum_{i=1}^{n} TT_i & \text{(Average Turnaround Time)}
\end{align}

\subsection{Tiền nhiệm vs. Không tiền nhiệm}

\textbf{Non-preemptive (Không tiền nhiệm)}: Khi một tiến trình bắt đầu chạy, 
nó sẽ chạy liên tục cho đến khi hoàn thành hoặc bị chặn (I/O). CPU không thể 
lấy lại từ tiến trình đó cho đến khi nó tự động từ bỏ.

\textbf{Preemptive (Tiền nhiệm)}: CPU có thể bị lấy lại từ tiến trình đang chạy 
để cấp cho tiến trình khác, thường dựa trên prioritization.

\section{Các thuật toán lập lịch}

\subsection{1. FCFS - First Come First Serve}

\textbf{Nguyên lý}: Tiến trình nào đến trước phục vụ trước. Không tiền nhiệm.

\textbf{Giả mã}:
\begin{lstlisting}[language=Python,caption={FCFS pseudocode}]
sort processes by arrival time
time = 0
for each process p in sorted list:
    if time < p.arrival:
        time = p.arrival  # CPU idle, jump to next arrival
    p.start_time = time
    p.finish_time = time + p.burst
    p.wait_time = p.start_time - p.arrival
    time = p.finish_time
\end{lstlisting}

\textbf{Độ phức tạp}: $O(n \log n)$ do sắp xếp

\textbf{Ưu/nhược điểm}:
\begin{itemize}[leftmargin=1.5cm]
  \item \textbf{Ưu}: Đơn giản, dễ cài đặt, công bằng
  \item \textbf{Nhược}: Gây \textit{Convoy Effect} - nếu tiến trình dài đến trước, 
  tất cả tiến trình ngắn phía sau phải chờ lâu, làm AWT tệ
\end{itemize}

\subsection{2. SJF - Shortest Job First}

\textbf{Nguyên lý}: Tại mỗi lần CPU rỗi, chọn tiến trình có burst time nhỏ nhất 
trong số những tiến trình đã đến. Không tiền nhiệm.

\textbf{Tính chất}:
\begin{itemize}[leftmargin=1.5cm]
  \item SJF tối ưu hoá AWT theo lý thuyết (giảm nhất trong các non-preemptive algorithms)
  \item Nhưng có khuyết điểm: có thể gây \textit{starvation} cho tiến trình dài nếu luôn có tiến trình ngắn đến
  \item Yêu cầu biết trước burst time, điều không thực tế trong hệ thống thực
\end{itemize}

\subsection{3. SRTF - Shortest Remaining Time First}

\textbf{Nguyên lý}: Phiên bản preemptive của SJF. Tại mỗi tick thời gian, 
CPU chọn tiến trình có \textit{remaining time} (thời gian còn lại) nhỏ nhất. 
Nếu tiến trình mới đến có remaining time nhỏ hơn tiến trình đang chạy, 
thì ngắt tiến trình hiện tại và chạy tiến trình mới.

\textbf{Tính chất}:
\begin{itemize}[leftmargin=1.5cm]
  \item Tối ưu hoá ATT trong tất cả các preemptive algorithms
  \item Chi phí context switch cao nếu quá chi tiết
  \item Cũng gây starvation cho tiến trình dài
\end{itemize}

\subsection{4. RR - Round Robin}

\textbf{Nguyên lý}: CPU được chia thành các time slice (quantum) bằng nhau. 
Mỗi tiến trình trong hàng đợi được cấp CPU tối đa một quantum. 
Nếu không hoàn thành, tiến trình được đặt ở cuối hàng đợi.

\textbf{Giả mã}:
\begin{lstlisting}[language=Python,caption={RR pseudocode}]
queue Q = empty
time = 0
enqueue all processes that arrive at time 0

while Q is not empty:
    if Q is empty:
        time = next arrival time
        enqueue all processes arriving at time
        continue
    
    p = Q.dequeue()
    slice = min(quantum, p.remaining_time)
    time += slice
    p.remaining_time -= slice
    
    enqueue all processes arriving at time
    
    if p.remaining_time > 0:
        Q.enqueue(p)
    else:
        p.finish_time = time
\end{lstlisting}

\textbf{Ảnh hưởng của quantum}:
\begin{itemize}[leftmargin=1.5cm]
  \item Quantum nhỏ → Response time tốt, nhưng overhead context switch cao
  \item Quantum lớn → Overhead thấp, nhưng giống FCFS hơn
  \item Quantum = $\infty$ → Trở thành FCFS
\end{itemize}

\subsection{5. Priority Scheduling}

\textbf{Nguyên lý}: Mỗi tiến trình có một mức ưu tiên (priority number). 
CPU luôn chọn tiến trình có ưu tiên cao nhất (priority number nhỏ nhất theo quy ước).

\textbf{Hai biến thể}:
\begin{itemize}[leftmargin=1.5cm]
  \item \textbf{Non-preemptive}: Tiến trình chạy cho đến khi xong
  \item \textbf{Preemptive}: Nếu tiến trình mới đến có ưu tiên cao hơn, ngắt tiến trình hiện tại
\end{itemize}

\textbf{Vấn đề Starvation}: Tiến trình ưu tiên thấp có thể chờ mãi mãi.

\textbf{Giải pháp - Aging}: Tăng dần ưu tiên của tiến trình theo thời gian chờ. 
Công thức ví dụ:
\begin{equation}
\text{priority}(t) = \text{original\_priority} - \lfloor t / \text{aging\_factor} \rfloor
\end{equation}

\section{Quản lý bộ nhớ ảo}

\subsection{Paging}

\textbf{Khái niệm}: Chia không gian nhớ logic (process memory space) thành các đơn vị 
cố định gọi là \textbf{trang (pages)}, và chia bộ nhớ vật lý thành các đơn vị 
tương ứng gọi là \textbf{khung (frames)}.

\textbf{Dịch địa chỉ}:
Một địa chỉ logic $L$ (logical address) bao gồm:
\begin{equation}
L = \text{page\_number} \times \text{pageSize} + \text{offset}
\end{equation}

Bảng trang (page table) ánh xạ page number sang frame number:
\begin{equation}
\text{physical\_address} = \text{pageTable}[\text{page\_number}] \times \text{pageSize} + \text{offset}
\end{equation}

\textbf{Page Fault}: Khi trang cần không có trong bộ nhớ chính (RAM), xảy ra page fault. 
Hệ điều hành phải:
\begin{enumerate}[leftmargin=1.5cm]
  \item Chọn một khung nạn nhân (theo chính sách thay trang)
  \item Ghi nội dung khung nạn nhân xuống disk
  \item Nạp trang cần từ disk vào khung
  \item Cập nhật bảng trang
  \item Tiếp tục tiến trình bị ngắt
\end{enumerate}

\subsection{Segmentation}

\textbf{Khái niệm}: Chia không gian nhớ logic thành các phần tử có ý nghĩa 
(segments): code, data, heap, stack, ...

\textbf{Dịch địa chỉ}:
Một địa chỉ logic bao gồm $(segment\_number, offset)$:
\begin{equation}
\text{physical\_address} = \text{base}[\text{segment}] + \text{offset}
\end{equation}

Điều kiện hợp lệ: $0 \le \text{offset} < \text{limit}[\text{segment}]$

\subsection{Các chính sách thay trang}

Khi xảy ra page fault, phải chọn khung nào để loại bỏ. Các chính sách:

\subsubsection*{FIFO - First In First Out}

Loại bỏ trang được nạp vào sớm nhất.

\textbf{Giả mã}:
\begin{lstlisting}[language=Python,caption={FIFO replacement}]
queue = []  # khung được nạp theo thứ tự

for each memory reference m:
    page = get_page(m)
    if page is in memory:
        continue
    else:  # page fault
        if queue.length < num_frames:
            add_to_memory(page)
            queue.append(page)
        else:
            victim = queue.pop_front()
            remove_from_memory(victim)
            add_to_memory(page)
            queue.append(page)
        faults += 1
\end{lstlisting}

\textbf{Vấn đề Belady's Anomaly}: Tăng số khung lại có thể tăng số page fault. 
Ví dụ: với sequence $1,2,3,4,1,2,5,1,2,3,4,5$ và 4 khung:
\begin{itemize}[leftmargin=1.5cm]
  \item 3 khung: 9 faults
  \item 4 khung: 10 faults
\end{itemize}

\subsubsection*{LRU - Least Recently Used}

Loại bỏ trang được dùng \textbf{lâu nhất} trong quá khứ.

\textbf{Giả mã}:
\begin{lstlisting}[language=Python,caption={LRU replacement}]
last_used = {}  # mapping page -> last use time

for each memory reference m at time t:
    page = get_page(m)
    last_used[page] = t
    
    if page is in memory:
        continue
    else:  # page fault
        if memory has free frame:
            add_to_memory(page)
        else:
            victim = page with minimum last_used time
            remove_from_memory(victim)
            add_to_memory(page)
        faults += 1
\end{lstlisting}

\textbf{Cơ sở lý thuyết}: LRU dựa trên \textit{Principle of Locality} - 
tiến trình có xu hướng sử dụng lại dữ liệu gần đây.

\subsubsection*{OPT - Optimal}

Loại bỏ trang sẽ được dùng \textbf{xa nhất} trong \textbf{tương lai}.

\textbf{Giả mã}:
\begin{lstlisting}[language=Python,caption={OPT replacement}]
next_use = {}  # mapping page -> next use time

for each memory reference m at position i:
    page = get_page(m)
    next_use[page] = next occurrence of page in reference string
    
    if page is in memory:
        continue
    else:  # page fault
        if memory has free frame:
            add_to_memory(page)
        else:
            victim = page with maximum next_use time
            remove_from_memory(victim)
            add_to_memory(page)
        faults += 1
\end{lstlisting}

\textbf{Tính chất}:
\begin{itemize}[leftmargin=1.5cm]
  \item Không thể triển khai thực tế vì không biết tương lai
  \item Sử dụng làm chuẩn so sánh (benchmark) cho các chính sách khác
  \item OPT cho số fault tối thiểu
\end{itemize}

\section{Giao tiếp liên tiến trình (IPC)}

\subsection{Pipe}

\textbf{Khái niệm}: Pipe là một kênh giao tiếp một chiều, tức là FIFO (First In First Out). 
Dữ liệu được ghi vào một đầu, đọc từ đầu kia.

\textbf{Quy tắc}:
\begin{itemize}[leftmargin=1.5cm]
  \item \textbf{Write}: Thêm dữ liệu vào cuối pipe
  \item \textbf{Read}: Lấy dữ liệu từ đầu pipe
  \item \textbf{Blocking}: Nếu pipe trống, read chờ; nếu pipe đầy, write chờ
\end{itemize}

\textbf{Ứng dụng}: Trong shell, \texttt{command1 | command2} ghi output của command1 vào pipe, 
command2 đọc từ pipe đó.

\subsection{Signal}

\textbf{Khái niệm}: Signal là một thông điệp bất đồng bộ (asynchronous). 
Khi một signal gửi tới tiến trình, hệ điều hành ngắt tiến trình đó, 
gọi signal handler, rồi tiếp tục tiến trình.

\textbf{Các signal phổ biến}:
\begin{itemize}[leftmargin=1.5cm]
  \item \textbf{SIGINT} (2): Interrupt - người dùng nhấn Ctrl+C
  \item \textbf{SIGKILL} (9): Kill - bắt buộc chết (không thể bắt)
  \item \textbf{SIGTERM} (15): Termination - yêu cầu kết thúc (có thể bắt)
  \item \textbf{SIGUSR1}, \textbf{SIGUSR2}: User-defined signals
  \item \textbf{SIGSTOP}, \textbf{SIGCONT}: Dừng/tiếp tục tiến trình
\end{itemize}

\section{Shell và Pipeline}

\textbf{Shell}: Trình thông dịch dòng lệnh (CLI interpreter). Đọc lệnh từ người dùng, 
phân tích cú pháp, gọi chương trình tương ứng.

\textbf{Pipeline}: Kết nối các lệnh bằng \texttt{|}, tức là output của lệnh trước 
trở thành input của lệnh sau.

\textbf{Ví dụ}:
\begin{verbatim}
cat file.txt | grep "search" | wc -l
\end{verbatim}

Luồng dữ liệu:
\begin{enumerate}[leftmargin=1.5cm]
  \item \texttt{cat file.txt} đọc file, ghi tất cả dòng vào pipe 1
  \item \texttt{grep "search"} đọc từ pipe 1, lọc dòng chứa "search", ghi vào pipe 2
  \item \texttt{wc -l} đếm số dòng từ pipe 2, in ra stdout
\end{enumerate}

\clearpage
