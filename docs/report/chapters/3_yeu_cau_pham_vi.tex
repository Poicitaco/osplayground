% ============================================================================
% CHƯƠNG 3: YÊU CẦU VÀ PHẠM VI
% ============================================================================

\chapter{Yêu cầu và phạm vi}
\label{ch:requirements}

\section{Yêu cầu chức năng}

Phần này liệt kê các yêu cầu chức năng (Functional Requirements) của dự án, 
được phân chia theo các mô-đun chính.

\subsection{Yêu cầu CPU Scheduling (CPU-001 đến CPU-010)}

\begin{itemize}[leftmargin=1.5cm]
  \item \textbf{CPU-001}: Hệ thống phải cài đặt thuật toán FCFS với kết quả AWT/ATT chính xác
  \item \textbf{CPU-002}: Hệ thống phải cài đặt thuật toán SJF không tiền nhiệm
  \item \textbf{CPU-003}: Hệ thống phải cài đặt thuật toán SRTF (preemptive SJF)
  \item \textbf{CPU-004}: Hệ thống phải cài đặt thuật toán Round Robin với quantum tuỳ chỉnh
  \item \textbf{CPU-005}: Hệ thống phải cài đặt Priority Scheduling (non-preemptive)
  \item \textbf{CPU-006}: Hệ thống phải cài đặt Priority Scheduling (preemptive)
  \item \textbf{CPU-007}: Tất cả thuật toán phải xuất biểu đồ Gantt (ASCII trong CLI, vector trong GUI)
  \item \textbf{CPU-008}: Tính toán AWT, ATT phải chính xác để 2 chữ số thập phân
  \item \textbf{CPU-009}: So sánh hiệu năng các thuật toán trên cùng dataset
  \item \textbf{CPU-010}: Log chi tiết từng bước thực thi mỗi thuật toán
\end{itemize}

\subsection{Yêu cầu Memory Management (MEM-001 đến MEM-010)}

\begin{itemize}[leftmargin=1.5cm]
  \item \textbf{MEM-001}: Hỗ trợ dịch địa chỉ Paging (logic → physical)
  \item \textbf{MEM-002}: Hỗ trợ dịch địa chỉ Segmentation
  \item \textbf{MEM-003}: Cài đặt chính sách thay trang FIFO
  \item \textbf{MEM-004}: Cài đặt chính sách thay trang LRU
  \item \textbf{MEM-005}: Cài đặt chính sách thay trang OPT (Optimal)
  \item \textbf{MEM-006}: Tính toán số page fault chính xác
  \item \textbf{MEM-007}: Lưu trữ trạng thái khung trang (\textit{frame states}) theo từng bước
  \item \textbf{MEM-008}: Hiển thị lưới khung trang (frame grid) trong GUI
  \item \textbf{MEM-009}: Hỗ trợ cấu hình số khung (frames) từ kịch bản
  \item \textbf{MEM-010}: Xuất báo cáo fault rate, so sánh các chính sách
\end{itemize}

\subsection{Yêu cầu IPC và Signal (IPC-001 đến IPC-008)}

\begin{itemize}[leftmargin=1.5cm]
  \item \textbf{IPC-001}: Mô phỏng Pipe với FIFO semantics
  \item \textbf{IPC-002}: Hỗ trợ Signal SIGINT (tín hiệu ngắt)
  \item \textbf{IPC-003}: Hỗ trợ Signal SIGUSR1 (tín hiệu người dùng 1)
  \item \textbf{IPC-004}: Hỗ trợ Signal SIGKILL (bắt buộc kết thúc)
  \item \textbf{IPC-005}: Xử lý signal handler và cập nhật trạng thái tiến trình
  \item \textbf{IPC-006}: Log sự kiện gửi/nhận signal
  \item \textbf{IPC-007}: Các signal ảnh hưởng tới điểm số IPC
  \item \textbf{IPC-008}: Giao diện điều khiển signal qua CLI/TUI/GUI
\end{itemize}

\subsection{Yêu cầu Shell (SHELL-001 đến SHELL-008)}

\begin{itemize}[leftmargin=1.5cm]
  \item \textbf{SHELL-001}: Triển khai 8+ lệnh cơ bản (ps, kill, cpu, mem, cat, grep, wc, echo)
  \item \textbf{SHELL-002}: Hỗ trợ pipeline với toán tử \verb+|+
  \item \textbf{SHELL-003}: Lệnh \verb|ps| liệt kê tiến trình mô phỏng
  \item \textbf{SHELL-004}: Lệnh \verb|kill| gửi SIGKILL đến tiến trình
  \item \textbf{SHELL-005}: Lệnh \verb|cpu <algo> [q]| chạy lập lịch CPU
  \item \textbf{SHELL-006}: Lệnh \verb|mem <policy> <F>| chạy thay trang
  \item \textbf{SHELL-007}: Xử lý lỗi và thông báo lỗi rõ ràng
  \item \textbf{SHELL-008}: Hỗ trợ lịch sử lệnh (command history)
\end{itemize}

\subsection{Yêu cầu Giao diện (UI-001 đến UI-015)}

\begin{itemize}[leftmargin=1.5cm]
  \item \textbf{UI-001}: Giao diện CLI cơ bản với menu chính
  \item \textbf{UI-002}: Giao diện TUI Dashboard với các panel
  \item \textbf{UI-003}: Giao diện GUI SDL2/ImGui (tùy chọn)
  \item \textbf{UI-004}: Panel CPU với Gantt chart tương tác
  \item \textbf{UI-005}: Panel Memory với lưới khung trang
  \item \textbf{UI-006}: Panel IPC hiển thị pipe, signal
  \item \textbf{UI-007}: Panel Scenarios: nạp/lưu kịch bản JSON
  \item \textbf{UI-008}: Panel Score hiển thị điểm tổng hợp
  \item \textbf{UI-009}: Dark theme UI, responsive layout
  \item \textbf{UI-010}: Zoom, scroll, pan trong GUI
  \item \textbf{UI-011}: Xuất ảnh chụp (screenshot) từ GUI
  \item \textbf{UI-012}: Hỗ trợ click lựa chọn thuật toán, tham số
  \item \textbf{UI-013}: Hiển thị trạng thái tức thời (real-time)
  \item \textbf{UI-014}: Thông báo lỗi rõ ràng (error messages)
  \item \textbf{UI-015}: Hỗ trợ keyboard shortcuts
\end{itemize}

\subsection{Yêu cầu hệ thống điểm (SCORE-001 đến SCORE-005)}

\begin{itemize}[leftmargin=1.5cm]
  \item \textbf{SCORE-001}: Tính CPU Score dựa trên công thức chuẩn hoá
  \item \textbf{SCORE-002}: Tính Memory Score dựa trên fault rate
  \item \textbf{SCORE-003}: Tính IPC Score dựa trên số signal
  \item \textbf{SCORE-004}: Tính System Stability Score (trung bình có trọng số)
  \item \textbf{SCORE-005}: Lưu và so sánh điểm qua các lần chạy
\end{itemize}

\section{Yêu cầu phi chức năng}

\subsection{Hiệu năng (Performance)}

\begin{itemize}[leftmargin=1.5cm]
  \item \textbf{PERF-001}: Thời gian xử lý CPU Scheduling < 100ms cho 100 tiến trình
  \item \textbf{PERF-002}: Thời gian xử lý Memory Simulation < 50ms cho 1000 references
  \item \textbf{PERF-003}: Khởi động GUI < 2 giây
  \item \textbf{PERF-004}: Render Gantt chart ≥ 30 FPS
\end{itemize}

\subsection{Khả năng sử dụng (Usability)}

\begin{itemize}[leftmargin=1.5cm]
  \item \textbf{USE-001}: Giao diện trực quan, dễ học cho sinh viên mới
  \item \textbf{USE-002}: Tài liệu hướng dẫn chi tiết với ví dụ
  \item \textbf{USE-003}: Hỗ trợ multiple language (Vietnamese, English)
  \item \textbf{USE-004}: Tùy chỉnh giao diện (theme, font size)
\end{itemize}

\subsection{Độ tin cậy (Reliability)}

\begin{itemize}[leftmargin=1.5cm]
  \item \textbf{REL-001}: Kết quả tính toán phải chính xác tới 2 chữ số thập phân
  \item \textbf{REL-002}: Không crash trên input hợp lệ
  \item \textbf{REL-003}: Xử lý gracefully các input không hợp lệ
  \item \textbf{REL-004}: Recovery từ file JSON lỗi với thông báo rõ ràng
\end{itemize}

\subsection{Duy trì (Maintainability)}

\begin{itemize}[leftmargin=1.5cm]
  \item \textbf{MAINT-001}: Code theo chuẩn C++17, SOLID principles
  \item \textbf{MAINT-002}: Mô-đun hóa: mỗi khối có giao diện rõ ràng
  \item \textbf{MAINT-003}: Unit test coverage ≥ 70\%
  \item \textbf{MAINT-004}: Inline documentation + separate docs
  \item \textbf{MAINT-005}: Build system (CMake) dễ cấu hình trên nhiều nền tảng
\end{itemize}

\subsection{Bảo mật (Security)}

\begin{itemize}[leftmargin=1.5cm]
  \item \textbf{SEC-001}: Input validation trên tất cả user input
  \item \textbf{SEC-002}: Không bị buffer overflow
  \item \textbf{SEC-003}: Xử lý file JSON an toàn
\end{itemize}

\subsection{Tính khả chuyển (Portability)}

\begin{itemize}[leftmargin=1.5cm]
  \item \textbf{PORT-001}: Chạy trên Windows (MSVC), Linux (GCC/Clang), macOS (Clang)
  \item \textbf{PORT-002}: Hỗ trợ 64-bit, 32-bit (nếu có)
  \item \textbf{PORT-003}: Không phụ thuộc vào thư viện OS-specific
\end{itemize}

\section{Các ràng buộc và giới hạn}

\subsection{Giới hạn chức năng}

\begin{table}[H]
\centering
\caption{Giới hạn chức năng của hệ thống}
\begin{tabular}{ll}
\toprule
\textbf{Chỉ tiêu} & \textbf{Giới hạn} \\
\midrule
Số tiến trình tối đa & 1000 \\
Thời gian burst tối đa & 10000 units \\
Số khung bộ nhớ tối đa & 10000 \\
Độ dài reference string & 100000 \\
Quantum RR & 1 đến 1000 units \\
Mức ưu tiên & 1 đến 40 \\
\bottomrule
\end{tabular}
\end{table}

\subsection{Giới hạn kỹ thuật}

\begin{itemize}[leftmargin=1.5cm]
  \item \textbf{Ngôn ngữ}: C++17 trở lên (C++20 không yêu cầu)
  \item \textbf{Compiler}: GCC 7+, Clang 5+, MSVC 2017+
  \item \textbf{RAM}: Tối thiểu 512MB (khuyến cáo 2GB)
  \item \textbf{Disk}: 100MB để cài đặt (có GUI)
  \item \textbf{GPU}: Không yêu cầu (GUI dùng software rendering)
\end{itemize}

\subsection{Ràng buộc phát triển}

\begin{itemize}[leftmargin=1.5cm]
  \item Dự án là open-source (license MIT)
  \item Không sử dụng thư viện GPL để tránh ràng buộc GPL
  \item Tất cả dependencies phải được ghi rõ trong CMakeLists.txt
  \item Thường xuyên cập nhật documentation
\end{itemize}

\section{Tiêu chí chấp nhận}

Dự án được coi là hoàn thành khi:

\begin{enumerate}[leftmargin=1.5cm]
  \item ✓ Tất cả yêu cầu chức năng đạt (CPU-001 → SHELL-008)
  \item ✓ Kiểm thử chi tiết (unit test, integration test, UAT)
  \item ✓ Tất cả lỗi nghiêm trọng được sửa
  \item ✓ Documentation hoàn chỉnh (user guide, API doc, architecture)
  \item ✓ Hiệu năng đạt mục tiêu (< 100ms cho CPU, < 50ms cho Memory)
  \item ✓ GUI chạy mượt mà, không flickering
  \item ✓ Cross-platform testing thành công
  \item ✓ Code review pass
  \item ✓ Báo cáo đồ án hoàn chỉnh, trình bày rõ ràng
\end{enumerate}

\clearpage
