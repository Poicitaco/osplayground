% ============================================================================
% CHƯƠNG 7: CHI TIẾT IPC VÀ SHELL
% ============================================================================

\chapter{Thiết kế chi tiết - IPC và Shell}
\label{ch:ipc_shell}

\section{Inter-Process Communication (IPC)}

\subsection{Pipe}

\subsubsection{Mô hình}

Pipe là một hàng đợi FIFO một chiều:
\begin{itemize}[leftmargin=1.5cm]
  \item \textbf{Write end}: Ghi dữ liệu vào cuối
  \item \textbf{Read end}: Đọc dữ liệu từ đầu
  \item \textbf{Buffer}: Lưu trữ tối đa N bytes
\end{itemize}

\subsubsection{Cài đặt}

\begin{lstlisting}[language=C++,caption={Pipe implementation}]
class Pipe {
private:
  queue<char> buffer;
  const int MAX_SIZE = 1024;
  
public:
  bool write(const string& data) {
    for (char c : data) {
      if (buffer.size() >= MAX_SIZE) {
        return false;  // Buffer đầy, write block
      }
      buffer.push(c);
    }
    return true;
  }
  
  string read(int count) {
    string result;
    while (count > 0 && !buffer.empty()) {
      result += buffer.front();
      buffer.pop();
      count--;
    }
    return result;
  }
  
  bool isEmpty() const { return buffer.empty(); }
  int size() const { return buffer.size(); }
};
\end{lstlisting}

\subsection{Signal}

\subsubsection{Các signal hỗ trợ}

\begin{table}[H]
\centering
\caption{Các signal được hỗ trợ}
\begin{tabular}{lll}
\toprule
\textbf{Signal} & \textbf{Số} & \textbf{Ý nghĩa} \\
\midrule
SIGINT & 2 & Interrupt (Ctrl+C) \\
SIGUSR1 & 10 & User-defined 1 \\
SIGKILL & 9 & Kill (không bắt được) \\
\bottomrule
\end{tabular}
\end{table}

\subsubsection{Cài đặt}

\begin{lstlisting}[language=C++,caption={Signal handling}]
enum class Signal { SIGINT, SIGUSR1, SIGKILL };

struct SignalEvent {
  int pid;
  Signal sig;
  int timestamp;
  bool delivered;
};

class SignalHandler {
private:
  vector<SignalEvent> events;
  
public:
  void sendSignal(int pid, Signal sig, int time) {
    events.push_back({pid, sig, time, false});
    
    // Log event
    string sig_name = (sig == Signal::SIGINT) ? "SIGINT" :
                      (sig == Signal::SIGUSR1) ? "SIGUSR1" :
                      "SIGKILL";
    log("Send " + sig_name + " to PID " + to_string(pid));
  }
  
  void handleSignal(int pid, Signal sig) {
    for (auto& event : events) {
      if (event.pid == pid && event.sig == sig && !event.delivered) {
        event.delivered = true;
        
        if (sig == Signal::SIGKILL) {
          // Tiến trình bị kill
          // Cập nhật trạng thái tiến trình
        }
        break;
      }
    }
  }
  
  const vector<SignalEvent>& getEvents() const {
    return events;
  }
};
\end{lstlisting}

\subsection{IPCManager}

\begin{lstlisting}[language=C++,caption={IPC Manager}]
class IPCManager {
private:
  map<int, Pipe> pipes;
  SignalHandler signal_handler;
  
public:
  void createPipe(int pipe_id) {
    pipes[pipe_id] = Pipe();
  }
  
  void writeToPipe(int pipe_id, const string& data) {
    if (pipes.find(pipe_id) != pipes.end()) {
      pipes[pipe_id].write(data);
    }
  }
  
  string readFromPipe(int pipe_id, int count) {
    if (pipes.find(pipe_id) != pipes.end()) {
      return pipes[pipe_id].read(count);
    }
    return "";
  }
  
  void sendSignal(int pid, Signal sig, int time) {
    signal_handler.sendSignal(pid, sig, time);
  }
  
  void handleSignal(int pid, Signal sig) {
    signal_handler.handleSignal(pid, sig);
  }
  
  const vector<SignalEvent>& getSignalEvents() const {
    return signal_handler.getEvents();
  }
};
\end{lstlisting}

\section{Shell mini}

\subsection{Cú pháp và ngữ pháp}

\subsubsection{EBNF - Extended Backus-Naur Form}

\begin{lstlisting}[caption={EBNF của Shell mini}]
program       ::= command_line EOF
command_line  ::= command ( '|' command )*
command       ::= CMD_NAME [ ARG { ARG } ]
CMD_NAME      ::= 'ps' | 'kill' | 'cpu' | 'mem' | 'cat' 
                | 'grep' | 'wc' | 'echo' | 'sig' | 'ls'
ARG           ::= STRING
STRING        ::= sequence_of_non_whitespace_characters
\end{lstlisting}

\subsection{Các lệnh được hỗ trợ}

\subsubsection{ps - List processes}

\begin{lstlisting}
ps [options]
  -a: Show all processes
  -r: Show running processes only
  (default: show all)
\end{lstlisting}

Ví dụ output:
\begin{verbatim}
PID  ARRIVAL BURST PRIORITY STATE
1    0       10    2        Ready
2    5       15    1        Running
3    10      8     3        Terminated
\end{verbatim}

\subsubsection{kill - Kill a process}

\begin{lstlisting}
kill <PID>
  Gửi SIGKILL đến PID
\end{lstlisting}

\subsubsection{sig - Send signal}

\begin{lstlisting}
sig <PID> <SIGNAME>
  SIGNAME: INT, USR1, KILL
\end{lstlisting}

\subsubsection{cpu - Run CPU scheduler}

\begin{lstlisting}
cpu <ALGORITHM> [QUANTUM]
  ALGORITHM: fcfs, sjf, srtf, rr, prio_np, prio_p
  QUANTUM: cho RR (default: 4)
\end{lstlisting}

\subsubsection{mem - Run memory simulator}

\begin{lstlisting}
mem <POLICY> <FRAMES> [REFERENCES...]
  POLICY: fifo, lru, opt, all
  FRAMES: số khung bộ nhớ
  REFERENCES: chuỗi tham chiếu (cách nhau bằng dấu cách)
\end{lstlisting}

\subsubsection{cat, grep, wc - Text processing}

\begin{lstlisting}
cat [FILE]
  Đọc file hoặc stdin

grep PATTERN
  Lọc dòng chứa pattern

wc [OPTIONS]
  -l: đếm dòng (default)
  -w: đếm từ
  -c: đếm ký tự
\end{lstlisting}

\subsubsection{echo - Print}

\begin{lstlisting}
echo [TEXT...]
  In text ra stdout
\end{lstlisting}

\subsubsection{ls - List files}

\begin{lstlisting}
ls [DIRECTORY]
  Liệt kê file/thư mục
\end{lstlisting}

\subsection{Pipeline}

Pipeline kết nối các lệnh bằng toán tử \verb+|+:

\begin{lstlisting}
command1 | command2 | command3
\end{lstlisting}

Luồng dữ liệu:
\begin{enumerate}[leftmargin=1.5cm]
  \item \texttt{command1} tạo output → vector<string> (mỗi dòng là một string)
  \item Output truyền tới \texttt{command2} làm input
  \item \texttt{command2} xử lý → output mới
  \item \texttt{command3} nhận input từ command2
  \item Kết quả cuối cùng được in ra stdout
\end{enumerate}

\subsection{Cài đặt Shell Parser}

\begin{lstlisting}[language=C++,caption={Shell Parser}]
struct Command {
  string name;
  vector<string> args;
};

class ShellParser {
public:
  vector<Command> parseLine(const string& line) {
    vector<Command> commands;
    
    // Tách bằng '|'
    vector<string> pipeline_parts = split(line, '|');
    
    for (const string& part : pipeline_parts) {
      // Tách bằng whitespace
      vector<string> tokens = split(part, ' ');
      
      if (tokens.empty()) continue;
      
      Command cmd;
      cmd.name = tokens[0];
      for (int i = 1; i < tokens.size(); ++i) {
        cmd.args.push_back(tokens[i]);
      }
      
      commands.push_back(cmd);
    }
    
    return commands;
  }
  
private:
  vector<string> split(const string& s, char delim) {
    vector<string> result;
    string current;
    
    for (char c : s) {
      if (c == delim) {
        if (!current.empty()) {
          result.push_back(current);
          current.clear();
        }
      } else {
        current += c;
      }
    }
    
    if (!current.empty()) {
      result.push_back(current);
    }
    
    return result;
  }
};
\end{lstlisting}

\subsection{Cài đặt Shell Executor}

\begin{lstlisting}[language=C++,caption={Shell Executor}]
class ShellExecutor {
private:
  SystemState& system_state;
  
public:
  ShellExecutor(SystemState& sys) : system_state(sys) {}
  
  vector<string> executePipeline(const vector<Command>& commands) {
    vector<string> output;
    
    for (int i = 0; i < commands.size(); ++i) {
      vector<string> input = output;
      output = executeCommand(commands[i], input);
    }
    
    return output;
  }
  
private:
  vector<string> executeCommand(const Command& cmd, 
                                const vector<string>& input) {
    vector<string> output;
    
    if (cmd.name == "ps") {
      output = executPS(cmd.args);
    } else if (cmd.name == "kill") {
      output = executeKill(cmd.args);
    } else if (cmd.name == "cpu") {
      output = executeCPU(cmd.args);
    } else if (cmd.name == "mem") {
      output = executeMEM(cmd.args);
    } else if (cmd.name == "grep") {
      output = executeGrep(cmd.args, input);
    } else if (cmd.name == "wc") {
      output = executeWC(cmd.args, input);
    } else if (cmd.name == "cat") {
      output = executeCat(cmd.args);
    } else if (cmd.name == "echo") {
      output = executeEcho(cmd.args);
    }
    
    return output;
  }
  
  vector<string> executeGrep(const vector<string>& args, 
                            const vector<string>& input) {
    vector<string> output;
    if (args.empty()) return output;
    
    string pattern = args[0];
    for (const string& line : input) {
      if (line.find(pattern) != string::npos) {
        output.push_back(line);
      }
    }
    return output;
  }
  
  vector<string> executeWC(const vector<string>& args, 
                          const vector<string>& input) {
    vector<string> output;
    
    bool count_lines = true;
    if (!args.empty() && args[0] == "-w") {
      count_lines = false;
    }
    
    if (count_lines) {
      output.push_back(to_string(input.size()));
    } else {
      int total_words = 0;
      for (const string& line : input) {
        total_words += count_words(line);
      }
      output.push_back(to_string(total_words));
    }
    
    return output;
  }
  
  int count_words(const string& s) {
    int count = 0;
    bool in_word = false;
    
    for (char c : s) {
      if (isspace(c)) {
        in_word = false;
      } else if (!in_word) {
        in_word = true;
        count++;
      }
    }
    
    return count;
  }
};
\end{lstlisting}

\section{Ví dụ sử dụng Shell}

\subsection{Ví dụ 1: Liệt kê tiến trình}

\begin{verbatim}
$ ps
PID  ARRIVAL BURST PRIORITY STATE
1    0       10    2        Ready
2    5       15    1        Ready
3    10      8     3        Ready
\end{verbatim}

\subsection{Ví dụ 2: Chạy CPU Scheduler}

\begin{verbatim}
$ cpu rr 4
Running Round Robin with quantum=4 on 3 processes
Average Waiting Time: 8.33
Average Turnaround Time: 18.33
Gantt: P1 | P2 | P3 | P1 | P2
\end{verbatim}

\subsection{Ví dụ 3: Pipeline}

\begin{verbatim}
$ ps | grep "PID" | wc -l
3
\end{verbatim}

Giải thích:
\begin{enumerate}[leftmargin=1.5cm]
  \item \texttt{ps} xuất ra danh sách tiến trình (4 dòng: header + 3 process)
  \item \texttt{grep "PID"} lọc dòng chứa "PID" → 1 dòng (header)
  \item \texttt{wc -l} đếm số dòng → 1
\end{enumerate}

\clearpage
