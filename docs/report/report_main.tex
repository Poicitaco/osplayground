% !TEX program = xelatex
\documentclass[12pt,a4paper]{report}

% ============================================================================
% PACKAGE IMPORTS
% ============================================================================

% Geometry and typesetting
\usepackage[a4paper,left=3cm,right=2cm,top=2.5cm,bottom=2.5cm]{geometry}
\usepackage{setspace}\onehalfspacing
\usepackage{enumitem}
\usepackage{booktabs}
\usepackage{array}
\usepackage{multirow}
\usepackage{longtable}
\usepackage{graphicx}
\usepackage{subcaption}
\usepackage{float}
\usepackage{tikz}
\usetikzlibrary{arrows.meta,positioning,fit,shapes.multipart}
\usepackage{pgfplots}
\usepackage{pgfplotstable}
\pgfplotsset{compat=1.18}
\usepackage{xcolor}
\usepackage{hyperref}
\usepackage{cleveref}
\usepackage{amsmath}
\usepackage{amssymb}
\usepackage{listings}
\usepackage{caption}
\usepackage{fancybox}

% Vietnamese with XeLaTeX
\usepackage{fontspec}
\usepackage{polyglossia}
\setdefaultlanguage{vietnamese}

% Fonts
\newfontfamily\vnmain{Times New Roman}[Scale=1.0]
\setmainfont{Times New Roman}
\setsansfont{Arial}
\setmonofont{Courier New}

% Fallback fonts
\IfFontExistsTF{Times New Roman}{}{\setmainfont{TeX Gyre Termes}}
\IfFontExistsTF{Arial}{}{\setsansfont{TeX Gyre Heros}}
\IfFontExistsTF{Courier New}{}{\setmonofont{Inconsolata}}

% Listings setup
\lstset{
  basicstyle=\ttfamily\small,
  breaklines=true,
  frame=single,
  rulecolor=\color{black!40},
  numbers=left,
  numberstyle=\tiny,
  keywordstyle=\bfseries\color{blue!60!black},
  commentstyle=\itshape\color{green!40!black},
  stringstyle=\color{red!60!black},
  showstringspaces=false,
  tabsize=2,
  breakatwhitespace=true
}

% Define additional language for pseudocode
\lstdefinelanguage{Python}{
  keywords={and, as, assert, async, await, break, class, continue, def, del, elif, else, except, finally, for, from, global, if, import, in, is, lambda, nonlocal, not, or, pass, raise, return, try, while, with, yield},
  sensitive=true,
  comment=[l]{\#},
  morestring=[b]',
  morestring=[b]"
}

% ============================================================================
% METADATA MACROS
% ============================================================================
\newcommand{\university}{TRƯỜNG ĐẠI HỌC [TÊN TRƯỜNG]}
\newcommand{\faculty}{KHOA CÔNG NGHỆ THÔNG TIN}
\newcommand{\course}{MÔN: Hệ Điều Hành}
\newcommand{\reporttitle}{OSPlayground – Mô Phỏng Hệ Điều Hành Tương Tác}
\newcommand{\studentname}{Họ Và Tên Sinh Viên}
\newcommand{\studentid}{[MSSV]}
\newcommand{\instructor}{[Họ và tên GVHD]}
\newcommand{\dateofreport}{\today}

% ============================================================================
% DOCUMENT SETUP
% ============================================================================

\title{\reporttitle}
\author{\studentname}
\date{\dateofreport}

% ============================================================================
% BEGIN DOCUMENT
% ============================================================================

\begin{document}

% ============================================================================
% COVER PAGE
% ============================================================================
\begin{titlepage}
  \begin{center}
    {\Large \textbf{\university}}\\[4pt]
    {\large \textbf{\faculty}}\\[20pt]
    {\large \course}\\[24pt]
    \rule{0.8\linewidth}{0.6pt}\\[12pt]
    {\LARGE \textbf{\reporttitle}}\\[8pt]
    {\large Báo Cáo Đồ Án Tốt Nghiệp}
    \rule{0.8\linewidth}{0.6pt}\\[32pt]
    \vspace{8pt}
    \begin{tabular}{p{3.5cm}l}
      \textbf{Họ và tên sinh viên:} & \studentname\\[8pt]
      \textbf{Mã số sinh viên:} & \studentid\\[8pt]
      \textbf{Giảng viên hướng dẫn:} & \instructor\\[8pt]
      \textbf{Ngày nộp báo cáo:} & \dateofreport\\[8pt]
    \end{tabular}
    \\
    \vfill
    {\small \textit{Báo cáo được biên dịch bằng \texttt{XeLaTeX} (Overleaf)}}\\
    {\small \textit{Năm học: 2024-2025}}
  \end{center}
\end{titlepage}

\clearpage

% ============================================================================
% INCLUDE CHAPTERS
% ============================================================================

% ============================================================================
% CHƯƠNG 0: MỤC LỤC VÀ TÓM TẮT
% ============================================================================

% ============================================================================
% Mục lục
% ============================================================================
{\hypersetup{linkcolor=black}
\pagenumbering{roman}
\setcounter{page}{1}

\tableofcontents
\clearpage

\listoffigures
\clearpage

\listoftables
\clearpage
}

% ============================================================================
% TÓM TẮT (Abstract)
% ============================================================================
\chapter*{Tóm tắt}
\addcontentsline{toc}{chapter}{Tóm tắt}

\textbf{OSPlayground} là dự án mô phỏng toàn diện các khái niệm cốt lõi của hệ điều hành (Operating System), 
được xây dựng dưới dạng các "mini-worlds" tương tác. Dự án cung cấp công cụ học tập trực quan hoá 
các thuật toán phức tạp của OS thông qua các giao diện người dùng đa dạng.

\subsection*{Phạm vi chính}

\begin{itemize}[leftmargin=1.5cm]
  \item \textbf{Lập lịch CPU (CPU Scheduling):} 
  Mô phỏng 6 thuật toán chính: FCFS (First Come First Serve), SJF (Shortest Job First), 
  SRTF (Shortest Remaining Time First), RR (Round Robin), Priority Non-preemptive, Priority Preemptive. 
  Các thuật toán được trực quan hoá bằng biểu đồ Gantt và tính toán các chỉ số hiệu năng (AWT, ATT).

  \item \textbf{Quản lý bộ nhớ (Memory Management):} 
  Mô phỏng các kỹ thuật quản lý bộ nhớ ảo bao gồm Paging và Segmentation. 
  Triển khai ba chính sách thay trang (Page Replacement): FIFO (First In First Out), 
  LRU (Least Recently Used), và OPT (Optimal). Hiển thị trạng thái khung trang theo thời gian thực.

  \item \textbf{Giao tiếp liên tiến trình (IPC - Inter-Process Communication):} 
  Mô phỏng Pipe (ống dẫn), Signal (tín hiệu) bao gồm SIGINT, SIGUSR1, SIGKILL. 
  Quan sát tác động của các tín hiệu lên trạng thái và điểm số hệ thống.

  \item \textbf{Shell mini:} 
  Triển khai shell dòng lệnh với hỗ trợ pipeline, cho phép kết hợp các lệnh 
  (\texttt{ps}, \texttt{kill}, \texttt{cpu}, \texttt{mem}, \texttt{grep}, \texttt{wc}) 
  để tương tác với mô phỏng hệ thống.

  \item \textbf{Giao diện người dùng:} 
  Ba lớp giao diện:
  \begin{enumerate}[nosep]
    \item CLI (Command Line Interface) cơ bản
    \item TUI (Text User Interface) - Dashboard tương tác
    \item GUI (Graphical User Interface) sử dụng SDL2 và Dear ImGui với biểu đồ Gantt, lưới khung trang
  \end{enumerate}

  \item \textbf{Hệ thống điểm tổng hợp (System Stability Score):} 
  Đánh giá toàn bộ hiệu năng hệ thống thông qua công thức tích hợp, kết hợp 
  các chỉ số từ CPU, Memory, và IPC với trọng số thích hợp.
\end{itemize}

\subsection*{Công nghệ sử dụng}

\begin{itemize}[leftmargin=1.5cm]
  \item \textbf{Ngôn ngữ:} C++17 (tiêu chuẩn C++ năm 2017)
  \item \textbf{Build system:} CMake 3.20+
  \item \textbf{Giao diện:} SDL2, Dear ImGui (Dear ImGui v1.89+)
  \item \textbf{Logging:} Hệ thống log tùy chỉnh với timestamp
  \item \textbf{Cấu hình:} JSON (libcurl/nlohmann json)
  \item \textbf{Platform:} Windows (MSVC), Linux/WSL (GCC/Clang), macOS (Clang)
\end{itemize}

\subsection*{Mục tiêu học tập}

Dự án hướng đến việc giúp sinh viên, nhà giáo dục, và những người yêu thích hệ điều hành:
\begin{enumerate}[leftmargin=1.5cm]
  \item Hiểu sâu các thuật toán lập lịch CPU qua trực quan hoá Gantt chart
  \item Nắm bắt được ảnh hưởng của các chỉ số (AWT, ATT) đối với hiệu năng
  \item Quan sát các chính sách thay trang và tương tác giữa bộ nhớ ảo và vật lý
  \item Thực hành giao tiếp liên tiến trình thông qua pipeline và signal
  \item Tổng hợp kiến thức thông qua hệ điểm toàn diện

\end{enumerate}

\subsection*{Kết quả chính}

\begin{itemize}[leftmargin=1.5cm]
  \item Mô-đun CPU với 6 thuật toán + Gantt chart (ASCII và GUI)
  \item Mô-đun Memory với 3 chính sách thay trang + lưới khung trang tương tác
  \item Mô-đun IPC với Pipe và 3 loại Signal
  \item Shell mini hỗ trợ 8+ lệnh và pipeline
  \item GUI đầy đủ với 5 panel chính
  \item Hệ thống điểm tổng hợp và ghi log chi tiết
  \item Hỗ trợ kịch bản JSON để tái lập các thử nghiệm
\end{itemize}

\subsection*{Cấu trúc báo cáo}

\begin{description}[leftmargin=2cm,style=nextline]
  \item[\textbf{Chương 1: Giới thiệu}] Động cơ, mục tiêu, đóng góp chính của dự án
  \item[\textbf{Chương 2: Cơ sở lý thuyết}] Kiến thức OS nền tảng, các khái niệm cốt lõi
  \item[\textbf{Chương 3: Yêu cầu và phạm vi}] Thông số kỹ thuật, giới hạn, ràng buộc
  \item[\textbf{Chương 4: Kiến trúc hệ thống}] Thiết kế tổng thể, mô-đun, flow dữ liệu
  \item[\textbf{Chương 5: Chi tiết CPU Scheduling}] Thiết kế, thuật toán, cài đặt
  \item[\textbf{Chương 6: Chi tiết Memory Management}] Dịch địa chỉ, thay trang, trực quan
  \item[\textbf{Chương 7: Chi tiết IPC và Shell}] Pipe, Signal, lệnh shell, pipeline
  \item[\textbf{Chương 8: Hệ thống điểm và đánh giá}] Công thức, chỉ số, phương pháp
  \item[\textbf{Chương 9: Giao diện người dùng}] CLI, TUI, GUI - thiết kế và tính năng
  \item[\textbf{Chương 10: Hướng dẫn sử dụng}] Build, chạy, flow demo, thao tác
  \item[\textbf{Chương 11: Kế hoạch kiểm thử}] Unit test, functional test, chấp nhận
  \item[\textbf{Chương 12: Kết quả thực nghiệm}] Dữ liệu, biểu đồ, so sánh hiệu năng
  \item[\textbf{Chương 13: Kết luận}] Tổng kết, hướng phát triển, bài học
\end{description}

\clearpage

% ============================================================================
% LỜI CẢM ƠN
% ============================================================================
\chapter*{Lời cảm ơn}
\addcontentsline{toc}{chapter}{Lời cảm ơn}

Xin chân thành cảm ơn:

\begin{itemize}[leftmargin=1.5cm]
  \item Thầy/cô giảng viên hướng dẫn đã tạo điều kiện, hướng dẫn tận tình trong suốt quá trình thực hiện dự án
  \item Các bạn cùng lớp, cùng nhóm đã hỗ trợ, góp ý và cung cấp feedback quý báu
  \item Đội ngũ Open Source tại Dear ImGui, SDL2 và cộng đồng C++ vì những thư viện tuyệt vời
  \item Gia đình, bạn bè đã khuyến khích, động viên trong quá trình học tập và phát triển
\end{itemize}

Hy vọng báo cáo này sẽ có ích cho những ai quan tâm đến Hệ điều hành.

\clearpage


\pagenumbering{arabic}
\setcounter{page}{1}

% ============================================================================
% CHƯƠNG 1: GIỚI THIỆU
% ============================================================================

\chapter{Giới thiệu}
\label{ch:introduction}

\section{Động cơ và lý do thực hiện}

Hệ điều hành (Operating System) là một trong những môn học cơ bản và quan trọng nhất 
trong chương trình đào tạo Khoa học máy tính và Công nghệ thông tin. Các khái niệm như 
lập lịch CPU, quản lý bộ nhớ, giao tiếp liên tiến trình là những nền tảng không thể thiếu 
để hiểu rõ cách máy tính hoạt động.

Tuy nhiên, các khái niệm này thường được trình bày một cách trừu tượng trong giáo trình lý thuyết, 
gây khó khăn cho sinh viên trong việc hình dung và thực hành. Một số thách thức chính:

\begin{itemize}[leftmargin=1.5cm]
  \item \textbf{Tính trừu tượng cao:} Các thuật toán lập lịch CPU, thay trang bộ nhớ 
  thường chỉ được giải thích qua ví dụ tính tay trên giấy, không có công cụ tương tác để thử nghiệm.
  
  \item \textbf{Khó khăn trong hình dung:} Sinh viên khó tưởng tượng được cách các tiến trình 
  chuyển đổi, bộ nhớ được chia sẻ, hoặc các tín hiệu được gửi trong hệ thống thực.
  
  \item \textbf{Thiếu công cụ học tập tương tác:} Không có nền tảng cho phép học sinh 
  "chơi" với các khái niệm này một cách trực quan.
  
  \item \textbf{Khó so sánh:} Để so sánh hiệu năng của các thuật toán khác nhau 
  trên cùng bộ dữ liệu đòi hỏi tính toán thủ công, tốn thời gian.
\end{itemize}

\textbf{OSPlayground} được sinh ra với mục đích giải quyết những thách thức này. 
Nó cung cấp một môi trường mô phỏng hoàn chỉnh nơi các khái niệm OS không còn là những 
ký hiệu trừu tượng mà trở thành những quá trình tương tác có thể quan sát, 
thử nghiệm, và học hỏi trực tiếp.

\section{Mục tiêu dự án}

Mục tiêu chính của OSPlayground có thể được phân thành ba cấp độ:

\subsection*{1. Mục tiêu giáo dục (Educational Goals)}

\begin{enumerate}[leftmargin=1.5cm]
  \item Trực quan hoá các thuật toán lập lịch CPU qua biểu đồ Gantt động, 
  giúp sinh viên thấy rõ cách các tiến trình được xếp lịch trên CPU.
  
  \item Giáo dục về ảnh hưởng của các chỉ số hiệu năng (Average Waiting Time - AWT, 
  Average Turnaround Time - ATT) trên tính công bằng và hiệu quả của lập lịch.
  
  \item Minh hoạ các chính sách thay trang bộ nhớ (FIFO, LRU, OPT) 
  qua lưới khung trang tương tác, cho phép theo dõi page fault từng bước.
  
  \item Thực hành giao tiếp liên tiến trình thông qua Pipe và Signal 
  trong một môi trường mô phỏng an toàn, kiểm soát.
  
  \item Học cách viết shell script với pipeline (command1 | command2 | command3) 
  trên bộ lệnh mô phỏng.
\end{enumerate}

\subsection*{2. Mục tiêu kỹ thuật (Technical Goals)}

\begin{enumerate}[leftmargin=1.5cm]
  \item Xây dựng một kiến trúc phần mềm mô-đun, dễ mở rộng, 
  trong đó mỗi khối (CPU, Memory, IPC) hoạt động độc lập nhưng có thể tích hợp.
  
  \item Cài đặt chính xác các thuật toán OS theo tiêu chuẩn, 
  đảm bảo kết quả tính toán khớp với lý thuyết.
  
  \item Tạo giao diện người dùng đa cấp (CLI → TUI → GUI) 
  để phục vụ các nhu cầu sử dụng khác nhau.
  
  \item Hỗ trợ tải kịch bản (Scenario) từ tệp JSON 
  để đảm bảo tính tái lập và lưu trữ các test case.
  
  \item Triển khai hệ thống ghi log chi tiết 
  để hỗ trợ debug, verify, và tạo báo cáo tự động.
\end{enumerate}

\subsection*{3. Mục tiêu phát triển (Development Goals)}

\begin{enumerate}[leftmargin=1.5cm]
  \item Xây dựng dự án C++ tiêu chuẩn, 
  sử dụng C++17, CMake, không phụ thuộc vào các framework nặng.
  
  \item Đảm bảo tính cross-platform: 
  chạy được trên Windows (MSVC), Linux/WSL (GCC/Clang), macOS (Clang).
  
  \item Áp dụng các best practices: 
  SOLID principles, design patterns, unit testing, documentation.
  
  \item Tạo một nền tảng có thể mở rộng cho các tính năng tương lai 
  (networking simulation, virtual machine, advanced scheduling, ...).
\end{enumerate}

\section{Đóng góp chính của dự án}

\subsection*{1. Bộ thuật toán CPU Scheduling}

Dự án cài đặt 6 thuật toán lập lịch CPU chuẩn:

\begin{itemize}[leftmargin=1.5cm]
  \item \textbf{FCFS (First Come First Serve)}: Đơn giản, không tiền nhiệm
  \item \textbf{SJF (Shortest Job First)}: Tối ưu hoá AWT, không tiền nhiệm
  \item \textbf{SRTF (Shortest Remaining Time First)}: Phiên bản tiền nhiệm của SJF
  \item \textbf{RR (Round Robin)}: Công bằng thời gian, tiền nhiệm, có quantum
  \item \textbf{Priority Non-preemptive}: Ưu tiên, không thể ngắt
  \item \textbf{Priority Preemptive}: Ưu tiên, có thể ngắt khi công việc ưu tiên cao đến
\end{itemize}

Mỗi thuật toán cung cấp:
\begin{itemize}[leftmargin=1.5cm]
  \item Biểu đồ Gantt (ASCII trong CLI, vector graphic trong GUI)
  \item Tính toán AWT, ATT chính xác
  \item Log chi tiết từng bước thực thi
  \item So sánh performance trên cùng dataset
\end{itemize}

\subsection*{2. Bộ quản lý bộ nhớ}

Dự án triển khai:

\begin{itemize}[leftmargin=1.5cm]
  \item \textbf{Paging}: Chia không gian nhớ logic thành trang, vật lý thành khung
  \item \textbf{Segmentation}: Chia theo đoạn (segment)
  \item \textbf{Dịch địa chỉ}: Ánh xạ logic → physical với kiểm tra biên
  \item \textbf{3 chính sách thay trang}: FIFO, LRU, OPT
  \item \textbf{Thống kê}: Page fault count, fault rate, frame states theo thời gian
  \item \textbf{Trực quan}: Lưới khung trang (frame grid) tương tác theo thời gian
\end{itemize}

\subsection*{3. Mô-đun IPC và Signal}

\begin{itemize}[leftmargin=1.5cm]
  \item \textbf{Pipe}: Hàng đợi FIFO truyền chuỗi giữa tiến trình
  \item \textbf{Signal}: Hỗ trợ SIGINT, SIGUSR1, SIGKILL
  \item \textbf{Quản lý tín hiệu}: Xử lý, log sự kiện
  \item \textbf{Ảnh hưởng tới điểm}: Các signal ảnh hưởng tới điểm số hệ thống
\end{itemize}

\subsection*{4. Shell mini với Pipeline}

\begin{itemize}[leftmargin=1.5cm]
  \item \textbf{8+ lệnh}: ps, kill, sig, cpu, mem, cat, grep, wc, echo, ls
  \item \textbf{Pipeline}: Kết nối lệnh bằng \texttt{|}
  \item \textbf{Lệnh tương tác}: Có thể gửi signal, chạy thuật toán từ shell
  \item \textbf{Output processing}: Xử lý và biến đổi output qua pipeline
\end{itemize}

\subsection*{5. Giao diện đa cấp}

\begin{itemize}[leftmargin=1.5cm]
  \item \textbf{CLI}: Giao diện dòng lệnh cơ bản, nhanh gọn
  \item \textbf{TUI}: Dashboard tương tác với menu, menu con, trạng thái tức thời
  \item \textbf{GUI}: Giao diện đồ hoạ SDL2/ImGui với:
  \begin{itemize}
    \item Biểu đồ Gantt động (CPU panel)
    \item Lưới khung trang tương tác (Memory panel)
    \item Panel IPC, Scenarios, Score
    \item Dark theme, responsive layout, font system
  \end{itemize}
\end{itemize}

\subsection*{6. Hệ thống điểm tổng hợp}

\begin{itemize}[leftmargin=1.5cm]
  \item \textbf{CPU Score}: Dựa trên AWT, ATT (công thức chuẩn hoá)
  \item \textbf{Memory Score}: Dựa trên fault rate
  \item \textbf{IPC Score}: Dựa trên số lần kill/signal
  \item \textbf{System Stability Score}: Trung bình có trọng số của 3 điểm trên
\end{itemize}

\section{Phạm vi và giới hạn}

\subsection*{Phạm vi}

Dự án tập trung vào các khái niệm OS cốt lõi được dạy trong khóa học Hệ điều hành tiêu chuẩn:

\begin{itemize}[leftmargin=1.5cm]
  \item Lập lịch CPU (single-core model)
  \item Quản lý bộ nhớ ảo (virtual memory basics)
  \item IPC cơ bản (pipes, signals)
  \item Shell scripting cơ bản
\end{itemize}

\subsection*{Giới hạn}

Để giữ dự án có kích thước hợp lý và tập trung, các khía cạnh sau được ngoài phạm vi:

\begin{itemize}[leftmargin=1.5cm]
  \item \textbf{Không phải kernel thực}: Đây là mô phỏng user-space, không bao gồm bootloader, 
  interrupt handler, MMU driver thật
  \item \textbf{Không hỗ trợ multi-core}: Mô hình một lõi CPU duy nhất
  \item \textbf{Không có device I/O}: Không mô phỏng disk, network, thiết bị ngoại vi
  \item \textbf{Không hỗ trợ thread}: Chỉ mô phỏng tiến trình (process), không thread
  \item \textbf{Không có file system}: Không triển khai filesystem tương tác
  \item \textbf{Không hỗ trợ network}: Không mô phỏng network stack
\end{itemize}

\section{Cấu trúc báo cáo}

Báo cáo được tổ chức thành 13 chương chính:

\begin{description}[leftmargin=2cm,style=nextline]
  \item[\textbf{Chương 1 - Giới thiệu}] Động cơ, mục tiêu, đóng góp
  \item[\textbf{Chương 2 - Cơ sở lý thuyết}] Kiến thức nền tảng, công thức toán học
  \item[\textbf{Chương 3 - Yêu cầu và phạm vi}] Thông số kỹ thuật, ràng buộc
  \item[\textbf{Chương 4 - Kiến trúc}] Thiết kế tổng thể, flow dữ liệu
  \item[\textbf{Chương 5 - Chi tiết CPU}] Thuật toán, cài đặt, Gantt
  \item[\textbf{Chương 6 - Chi tiết Memory}] Dịch địa chỉ, thay trang
  \item[\textbf{Chương 7 - Chi tiết IPC}] Pipe, Signal, Shell
  \item[\textbf{Chương 8 - Hệ thống điểm}] Công thức, phương pháp đánh giá
  \item[\textbf{Chương 9 - Giao diện}] CLI, TUI, GUI
  \item[\textbf{Chương 10 - Hướng dẫn sử dụng}] Build, chạy, flow
  \item[\textbf{Chương 11 - Kiểm thử}] Unit test, functional test
  \item[\textbf{Chương 12 - Kết quả}] Dữ liệu, biểu đồ, so sánh
  \item[\textbf{Chương 13 - Kết luận}] Tổng kết, hướng phát triển
\end{description}

\clearpage


% ============================================================================
% CHƯƠNG 2: CƠ SỞ LÝ THUYẾT
% ============================================================================

\chapter{Cơ sở lý thuyết}
\label{ch:theory}

\section{Khái niệm cơ bản về hệ điều hành}

\subsection{Tiến trình (Process), luồng (Thread) và bối cảnh (Context)}

\textbf{Tiến trình (Process)} là một chương trình đang thực thi. Mỗi tiến trình có:
\begin{itemize}[leftmargin=1.5cm]
  \item \textbf{Bộ nhớ}: Bao gồm code, data, heap, stack
  \item \textbf{Trạng thái CPU}: Giá trị các thanh ghi
  \item \textbf{Thông tin bối cảnh}: PID (Process ID), cha mẹ, ưu tiên, ...
  \item \textbf{Tài nguyên}: Tập tin mở, socket, bộ nhớ chia sẻ, ...
\end{itemize}

\textbf{Luồng (Thread)} là đơn vị lập lịch nhỏ hơn, nằm bên trong tiến trình. 
Các luồng cùng tiến trình chia sẻ bộ nhớ nhưng có stack và thanh ghi riêng.

\textbf{Bối cảnh (Context)} là tập hợp thông tin CPU cần để tiếp tục thực thi một 
tiến trình: thanh ghi, con trỏ chương trình (PC), con trỏ ngăn xếp (SP), ...

\textbf{Chuyển ngữ cảnh (Context Switch)} xảy ra khi CPU chuyển từ tiến trình này sang 
tiến trình khác. Việc chuyển đổi này có chi phí (overhead) vì phải lưu/khôi phục trạng thái.

\subsection{Trạng thái tiến trình}

Trong mô phỏng, một tiến trình có thể ở các trạng thái:

\begin{itemize}[leftmargin=1.5cm]
  \item \textbf{New}: Tiến trình vừa được tạo, chưa sẵn sàng
  \item \textbf{Ready}: Tiến trình sẵn sàng chạy, chờ CPU
  \item \textbf{Running}: Tiến trình đang chạy trên CPU
  \item \textbf{Blocked}: Tiến trình chờ I/O hoặc tài nguyên
  \item \textbf{Terminated}: Tiến trình kết thúc
\end{itemize}

Trong dự án này, vì không mô phỏng I/O nên chủ yếu quan tâm đến Ready, Running, Terminated.

\section{Lập lịch CPU (CPU Scheduling)}

\subsection{Mục tiêu của lập lịch CPU}

Lập lịch CPU (CPU Scheduling) quyết định tiến trình nào được chạy trên CPU tại thời điểm nào. 
Mục tiêu chính:

\begin{itemize}[leftmargin=1.5cm]
  \item \textbf{Độ sử dụng CPU cao}: Giữ CPU luôn bận rộn, giảm thời gian rỗi
  \item \textbf{Throughput cao}: Hoàn thành nhiều tiến trình trong một đơn vị thời gian
  \item \textbf{Turnaround time thấp}: Giảm thời gian tổng để hoàn thành một tiến trình
  \item \textbf{Waiting time thấp}: Giảm thời gian chờ trong hàng đợi
  \item \textbf{Response time thấp}: Phản ứng nhanh đối với yêu cầu tương tác
  \item \textbf{Công bằng}: Mỗi tiến trình được cơ hội công bằng trên CPU
\end{itemize}

Các mục tiêu này thường xung đột với nhau, vì vậy mỗi thuật toán lập lịch là một 
thỏa hiệp giữa chúng.

\subsection{Các chỉ số đo lường}

Đối với tiến trình $i$, đặt:
\begin{itemize}[leftmargin=1.5cm]
  \item $A_i$ = arrival time (thời điểm đến)
  \item $B_i$ = burst time (thời gian CPU cần)
  \item $S_i$ = start time (lúc bắt đầu chạy)
  \item $F_i$ = finish time (lúc hoàn thành)
\end{itemize}

Khi đó:
\begin{align}
WT_i &= S_i - A_i & \text{(Waiting Time)} \\
TT_i &= F_i - A_i = WT_i + B_i & \text{(Turnaround Time)} \\
RT_i &= TT_i - B_i & \text{(Response Time)}
\end{align}

Các chỉ số trung bình:
\begin{align}
AWT &= \frac{1}{n}\sum_{i=1}^{n} WT_i & \text{(Average Waiting Time)} \\
ATT &= \frac{1}{n}\sum_{i=1}^{n} TT_i & \text{(Average Turnaround Time)}
\end{align}

\subsection{Tiền nhiệm vs. Không tiền nhiệm}

\textbf{Non-preemptive (Không tiền nhiệm)}: Khi một tiến trình bắt đầu chạy, 
nó sẽ chạy liên tục cho đến khi hoàn thành hoặc bị chặn (I/O). CPU không thể 
lấy lại từ tiến trình đó cho đến khi nó tự động từ bỏ.

\textbf{Preemptive (Tiền nhiệm)}: CPU có thể bị lấy lại từ tiến trình đang chạy 
để cấp cho tiến trình khác, thường dựa trên prioritization.

\section{Các thuật toán lập lịch}

\subsection{1. FCFS - First Come First Serve}

\textbf{Nguyên lý}: Tiến trình nào đến trước phục vụ trước. Không tiền nhiệm.

\textbf{Giả mã}:
\begin{lstlisting}[language=Python,caption={FCFS pseudocode}]
sort processes by arrival time
time = 0
for each process p in sorted list:
    if time < p.arrival:
        time = p.arrival  # CPU idle, jump to next arrival
    p.start_time = time
    p.finish_time = time + p.burst
    p.wait_time = p.start_time - p.arrival
    time = p.finish_time
\end{lstlisting}

\textbf{Độ phức tạp}: $O(n \log n)$ do sắp xếp

\textbf{Ưu/nhược điểm}:
\begin{itemize}[leftmargin=1.5cm]
  \item \textbf{Ưu}: Đơn giản, dễ cài đặt, công bằng
  \item \textbf{Nhược}: Gây \textit{Convoy Effect} - nếu tiến trình dài đến trước, 
  tất cả tiến trình ngắn phía sau phải chờ lâu, làm AWT tệ
\end{itemize}

\subsection{2. SJF - Shortest Job First}

\textbf{Nguyên lý}: Tại mỗi lần CPU rỗi, chọn tiến trình có burst time nhỏ nhất 
trong số những tiến trình đã đến. Không tiền nhiệm.

\textbf{Tính chất}:
\begin{itemize}[leftmargin=1.5cm]
  \item SJF tối ưu hoá AWT theo lý thuyết (giảm nhất trong các non-preemptive algorithms)
  \item Nhưng có khuyết điểm: có thể gây \textit{starvation} cho tiến trình dài nếu luôn có tiến trình ngắn đến
  \item Yêu cầu biết trước burst time, điều không thực tế trong hệ thống thực
\end{itemize}

\subsection{3. SRTF - Shortest Remaining Time First}

\textbf{Nguyên lý}: Phiên bản preemptive của SJF. Tại mỗi tick thời gian, 
CPU chọn tiến trình có \textit{remaining time} (thời gian còn lại) nhỏ nhất. 
Nếu tiến trình mới đến có remaining time nhỏ hơn tiến trình đang chạy, 
thì ngắt tiến trình hiện tại và chạy tiến trình mới.

\textbf{Tính chất}:
\begin{itemize}[leftmargin=1.5cm]
  \item Tối ưu hoá ATT trong tất cả các preemptive algorithms
  \item Chi phí context switch cao nếu quá chi tiết
  \item Cũng gây starvation cho tiến trình dài
\end{itemize}

\subsection{4. RR - Round Robin}

\textbf{Nguyên lý}: CPU được chia thành các time slice (quantum) bằng nhau. 
Mỗi tiến trình trong hàng đợi được cấp CPU tối đa một quantum. 
Nếu không hoàn thành, tiến trình được đặt ở cuối hàng đợi.

\textbf{Giả mã}:
\begin{lstlisting}[language=Python,caption={RR pseudocode}]
queue Q = empty
time = 0
enqueue all processes that arrive at time 0

while Q is not empty:
    if Q is empty:
        time = next arrival time
        enqueue all processes arriving at time
        continue
    
    p = Q.dequeue()
    slice = min(quantum, p.remaining_time)
    time += slice
    p.remaining_time -= slice
    
    enqueue all processes arriving at time
    
    if p.remaining_time > 0:
        Q.enqueue(p)
    else:
        p.finish_time = time
\end{lstlisting}

\textbf{Ảnh hưởng của quantum}:
\begin{itemize}[leftmargin=1.5cm]
  \item Quantum nhỏ → Response time tốt, nhưng overhead context switch cao
  \item Quantum lớn → Overhead thấp, nhưng giống FCFS hơn
  \item Quantum = $\infty$ → Trở thành FCFS
\end{itemize}

\subsection{5. Priority Scheduling}

\textbf{Nguyên lý}: Mỗi tiến trình có một mức ưu tiên (priority number). 
CPU luôn chọn tiến trình có ưu tiên cao nhất (priority number nhỏ nhất theo quy ước).

\textbf{Hai biến thể}:
\begin{itemize}[leftmargin=1.5cm]
  \item \textbf{Non-preemptive}: Tiến trình chạy cho đến khi xong
  \item \textbf{Preemptive}: Nếu tiến trình mới đến có ưu tiên cao hơn, ngắt tiến trình hiện tại
\end{itemize}

\textbf{Vấn đề Starvation}: Tiến trình ưu tiên thấp có thể chờ mãi mãi.

\textbf{Giải pháp - Aging}: Tăng dần ưu tiên của tiến trình theo thời gian chờ. 
Công thức ví dụ:
\begin{equation}
\text{priority}(t) = \text{original\_priority} - \lfloor t / \text{aging\_factor} \rfloor
\end{equation}

\section{Quản lý bộ nhớ ảo}

\subsection{Paging}

\textbf{Khái niệm}: Chia không gian nhớ logic (process memory space) thành các đơn vị 
cố định gọi là \textbf{trang (pages)}, và chia bộ nhớ vật lý thành các đơn vị 
tương ứng gọi là \textbf{khung (frames)}.

\textbf{Dịch địa chỉ}:
Một địa chỉ logic $L$ (logical address) bao gồm:
\begin{equation}
L = \text{page\_number} \times \text{pageSize} + \text{offset}
\end{equation}

Bảng trang (page table) ánh xạ page number sang frame number:
\begin{equation}
\text{physical\_address} = \text{pageTable}[\text{page\_number}] \times \text{pageSize} + \text{offset}
\end{equation}

\textbf{Page Fault}: Khi trang cần không có trong bộ nhớ chính (RAM), xảy ra page fault. 
Hệ điều hành phải:
\begin{enumerate}[leftmargin=1.5cm]
  \item Chọn một khung nạn nhân (theo chính sách thay trang)
  \item Ghi nội dung khung nạn nhân xuống disk
  \item Nạp trang cần từ disk vào khung
  \item Cập nhật bảng trang
  \item Tiếp tục tiến trình bị ngắt
\end{enumerate}

\subsection{Segmentation}

\textbf{Khái niệm}: Chia không gian nhớ logic thành các phần tử có ý nghĩa 
(segments): code, data, heap, stack, ...

\textbf{Dịch địa chỉ}:
Một địa chỉ logic bao gồm $(segment\_number, offset)$:
\begin{equation}
\text{physical\_address} = \text{base}[\text{segment}] + \text{offset}
\end{equation}

Điều kiện hợp lệ: $0 \le \text{offset} < \text{limit}[\text{segment}]$

\subsection{Các chính sách thay trang}

Khi xảy ra page fault, phải chọn khung nào để loại bỏ. Các chính sách:

\subsubsection*{FIFO - First In First Out}

Loại bỏ trang được nạp vào sớm nhất.

\textbf{Giả mã}:
\begin{lstlisting}[language=Python,caption={FIFO replacement}]
queue = []  # khung được nạp theo thứ tự

for each memory reference m:
    page = get_page(m)
    if page is in memory:
        continue
    else:  # page fault
        if queue.length < num_frames:
            add_to_memory(page)
            queue.append(page)
        else:
            victim = queue.pop_front()
            remove_from_memory(victim)
            add_to_memory(page)
            queue.append(page)
        faults += 1
\end{lstlisting}

\textbf{Vấn đề Belady's Anomaly}: Tăng số khung lại có thể tăng số page fault. 
Ví dụ: với sequence $1,2,3,4,1,2,5,1,2,3,4,5$ và 4 khung:
\begin{itemize}[leftmargin=1.5cm]
  \item 3 khung: 9 faults
  \item 4 khung: 10 faults
\end{itemize}

\subsubsection*{LRU - Least Recently Used}

Loại bỏ trang được dùng \textbf{lâu nhất} trong quá khứ.

\textbf{Giả mã}:
\begin{lstlisting}[language=Python,caption={LRU replacement}]
last_used = {}  # mapping page -> last use time

for each memory reference m at time t:
    page = get_page(m)
    last_used[page] = t
    
    if page is in memory:
        continue
    else:  # page fault
        if memory has free frame:
            add_to_memory(page)
        else:
            victim = page with minimum last_used time
            remove_from_memory(victim)
            add_to_memory(page)
        faults += 1
\end{lstlisting}

\textbf{Cơ sở lý thuyết}: LRU dựa trên \textit{Principle of Locality} - 
tiến trình có xu hướng sử dụng lại dữ liệu gần đây.

\subsubsection*{OPT - Optimal}

Loại bỏ trang sẽ được dùng \textbf{xa nhất} trong \textbf{tương lai}.

\textbf{Giả mã}:
\begin{lstlisting}[language=Python,caption={OPT replacement}]
next_use = {}  # mapping page -> next use time

for each memory reference m at position i:
    page = get_page(m)
    next_use[page] = next occurrence of page in reference string
    
    if page is in memory:
        continue
    else:  # page fault
        if memory has free frame:
            add_to_memory(page)
        else:
            victim = page with maximum next_use time
            remove_from_memory(victim)
            add_to_memory(page)
        faults += 1
\end{lstlisting}

\textbf{Tính chất}:
\begin{itemize}[leftmargin=1.5cm]
  \item Không thể triển khai thực tế vì không biết tương lai
  \item Sử dụng làm chuẩn so sánh (benchmark) cho các chính sách khác
  \item OPT cho số fault tối thiểu
\end{itemize}

\section{Giao tiếp liên tiến trình (IPC)}

\subsection{Pipe}

\textbf{Khái niệm}: Pipe là một kênh giao tiếp một chiều, tức là FIFO (First In First Out). 
Dữ liệu được ghi vào một đầu, đọc từ đầu kia.

\textbf{Quy tắc}:
\begin{itemize}[leftmargin=1.5cm]
  \item \textbf{Write}: Thêm dữ liệu vào cuối pipe
  \item \textbf{Read}: Lấy dữ liệu từ đầu pipe
  \item \textbf{Blocking}: Nếu pipe trống, read chờ; nếu pipe đầy, write chờ
\end{itemize}

\textbf{Ứng dụng}: Trong shell, \texttt{command1 | command2} ghi output của command1 vào pipe, 
command2 đọc từ pipe đó.

\subsection{Signal}

\textbf{Khái niệm}: Signal là một thông điệp bất đồng bộ (asynchronous). 
Khi một signal gửi tới tiến trình, hệ điều hành ngắt tiến trình đó, 
gọi signal handler, rồi tiếp tục tiến trình.

\textbf{Các signal phổ biến}:
\begin{itemize}[leftmargin=1.5cm]
  \item \textbf{SIGINT} (2): Interrupt - người dùng nhấn Ctrl+C
  \item \textbf{SIGKILL} (9): Kill - bắt buộc chết (không thể bắt)
  \item \textbf{SIGTERM} (15): Termination - yêu cầu kết thúc (có thể bắt)
  \item \textbf{SIGUSR1}, \textbf{SIGUSR2}: User-defined signals
  \item \textbf{SIGSTOP}, \textbf{SIGCONT}: Dừng/tiếp tục tiến trình
\end{itemize}

\section{Shell và Pipeline}

\textbf{Shell}: Trình thông dịch dòng lệnh (CLI interpreter). Đọc lệnh từ người dùng, 
phân tích cú pháp, gọi chương trình tương ứng.

\textbf{Pipeline}: Kết nối các lệnh bằng \texttt{|}, tức là output của lệnh trước 
trở thành input của lệnh sau.

\textbf{Ví dụ}:
\begin{verbatim}
cat file.txt | grep "search" | wc -l
\end{verbatim}

Luồng dữ liệu:
\begin{enumerate}[leftmargin=1.5cm]
  \item \texttt{cat file.txt} đọc file, ghi tất cả dòng vào pipe 1
  \item \texttt{grep "search"} đọc từ pipe 1, lọc dòng chứa "search", ghi vào pipe 2
  \item \texttt{wc -l} đếm số dòng từ pipe 2, in ra stdout
\end{enumerate}

\clearpage


% ============================================================================
% CHƯƠNG 3: YÊU CẦU VÀ PHẠM VI
% ============================================================================

\chapter{Yêu cầu và phạm vi}
\label{ch:requirements}

\section{Yêu cầu chức năng}

Phần này liệt kê các yêu cầu chức năng (Functional Requirements) của dự án, 
được phân chia theo các mô-đun chính.

\subsection{Yêu cầu CPU Scheduling (CPU-001 đến CPU-010)}

\begin{itemize}[leftmargin=1.5cm]
  \item \textbf{CPU-001}: Hệ thống phải cài đặt thuật toán FCFS với kết quả AWT/ATT chính xác
  \item \textbf{CPU-002}: Hệ thống phải cài đặt thuật toán SJF không tiền nhiệm
  \item \textbf{CPU-003}: Hệ thống phải cài đặt thuật toán SRTF (preemptive SJF)
  \item \textbf{CPU-004}: Hệ thống phải cài đặt thuật toán Round Robin với quantum tuỳ chỉnh
  \item \textbf{CPU-005}: Hệ thống phải cài đặt Priority Scheduling (non-preemptive)
  \item \textbf{CPU-006}: Hệ thống phải cài đặt Priority Scheduling (preemptive)
  \item \textbf{CPU-007}: Tất cả thuật toán phải xuất biểu đồ Gantt (ASCII trong CLI, vector trong GUI)
  \item \textbf{CPU-008}: Tính toán AWT, ATT phải chính xác để 2 chữ số thập phân
  \item \textbf{CPU-009}: So sánh hiệu năng các thuật toán trên cùng dataset
  \item \textbf{CPU-010}: Log chi tiết từng bước thực thi mỗi thuật toán
\end{itemize}

\subsection{Yêu cầu Memory Management (MEM-001 đến MEM-010)}

\begin{itemize}[leftmargin=1.5cm]
  \item \textbf{MEM-001}: Hỗ trợ dịch địa chỉ Paging (logic → physical)
  \item \textbf{MEM-002}: Hỗ trợ dịch địa chỉ Segmentation
  \item \textbf{MEM-003}: Cài đặt chính sách thay trang FIFO
  \item \textbf{MEM-004}: Cài đặt chính sách thay trang LRU
  \item \textbf{MEM-005}: Cài đặt chính sách thay trang OPT (Optimal)
  \item \textbf{MEM-006}: Tính toán số page fault chính xác
  \item \textbf{MEM-007}: Lưu trữ trạng thái khung trang (\textit{frame states}) theo từng bước
  \item \textbf{MEM-008}: Hiển thị lưới khung trang (frame grid) trong GUI
  \item \textbf{MEM-009}: Hỗ trợ cấu hình số khung (frames) từ kịch bản
  \item \textbf{MEM-010}: Xuất báo cáo fault rate, so sánh các chính sách
\end{itemize}

\subsection{Yêu cầu IPC và Signal (IPC-001 đến IPC-008)}

\begin{itemize}[leftmargin=1.5cm]
  \item \textbf{IPC-001}: Mô phỏng Pipe với FIFO semantics
  \item \textbf{IPC-002}: Hỗ trợ Signal SIGINT (tín hiệu ngắt)
  \item \textbf{IPC-003}: Hỗ trợ Signal SIGUSR1 (tín hiệu người dùng 1)
  \item \textbf{IPC-004}: Hỗ trợ Signal SIGKILL (bắt buộc kết thúc)
  \item \textbf{IPC-005}: Xử lý signal handler và cập nhật trạng thái tiến trình
  \item \textbf{IPC-006}: Log sự kiện gửi/nhận signal
  \item \textbf{IPC-007}: Các signal ảnh hưởng tới điểm số IPC
  \item \textbf{IPC-008}: Giao diện điều khiển signal qua CLI/TUI/GUI
\end{itemize}

\subsection{Yêu cầu Shell (SHELL-001 đến SHELL-008)}

\begin{itemize}[leftmargin=1.5cm]
  \item \textbf{SHELL-001}: Triển khai 8+ lệnh cơ bản (ps, kill, cpu, mem, cat, grep, wc, echo)
  \item \textbf{SHELL-002}: Hỗ trợ pipeline với toán tử \verb+|+
  \item \textbf{SHELL-003}: Lệnh \verb|ps| liệt kê tiến trình mô phỏng
  \item \textbf{SHELL-004}: Lệnh \verb|kill| gửi SIGKILL đến tiến trình
  \item \textbf{SHELL-005}: Lệnh \verb|cpu <algo> [q]| chạy lập lịch CPU
  \item \textbf{SHELL-006}: Lệnh \verb|mem <policy> <F>| chạy thay trang
  \item \textbf{SHELL-007}: Xử lý lỗi và thông báo lỗi rõ ràng
  \item \textbf{SHELL-008}: Hỗ trợ lịch sử lệnh (command history)
\end{itemize}

\subsection{Yêu cầu Giao diện (UI-001 đến UI-015)}

\begin{itemize}[leftmargin=1.5cm]
  \item \textbf{UI-001}: Giao diện CLI cơ bản với menu chính
  \item \textbf{UI-002}: Giao diện TUI Dashboard với các panel
  \item \textbf{UI-003}: Giao diện GUI SDL2/ImGui (tùy chọn)
  \item \textbf{UI-004}: Panel CPU với Gantt chart tương tác
  \item \textbf{UI-005}: Panel Memory với lưới khung trang
  \item \textbf{UI-006}: Panel IPC hiển thị pipe, signal
  \item \textbf{UI-007}: Panel Scenarios: nạp/lưu kịch bản JSON
  \item \textbf{UI-008}: Panel Score hiển thị điểm tổng hợp
  \item \textbf{UI-009}: Dark theme UI, responsive layout
  \item \textbf{UI-010}: Zoom, scroll, pan trong GUI
  \item \textbf{UI-011}: Xuất ảnh chụp (screenshot) từ GUI
  \item \textbf{UI-012}: Hỗ trợ click lựa chọn thuật toán, tham số
  \item \textbf{UI-013}: Hiển thị trạng thái tức thời (real-time)
  \item \textbf{UI-014}: Thông báo lỗi rõ ràng (error messages)
  \item \textbf{UI-015}: Hỗ trợ keyboard shortcuts
\end{itemize}

\subsection{Yêu cầu hệ thống điểm (SCORE-001 đến SCORE-005)}

\begin{itemize}[leftmargin=1.5cm]
  \item \textbf{SCORE-001}: Tính CPU Score dựa trên công thức chuẩn hoá
  \item \textbf{SCORE-002}: Tính Memory Score dựa trên fault rate
  \item \textbf{SCORE-003}: Tính IPC Score dựa trên số signal
  \item \textbf{SCORE-004}: Tính System Stability Score (trung bình có trọng số)
  \item \textbf{SCORE-005}: Lưu và so sánh điểm qua các lần chạy
\end{itemize}

\section{Yêu cầu phi chức năng}

\subsection{Hiệu năng (Performance)}

\begin{itemize}[leftmargin=1.5cm]
  \item \textbf{PERF-001}: Thời gian xử lý CPU Scheduling < 100ms cho 100 tiến trình
  \item \textbf{PERF-002}: Thời gian xử lý Memory Simulation < 50ms cho 1000 references
  \item \textbf{PERF-003}: Khởi động GUI < 2 giây
  \item \textbf{PERF-004}: Render Gantt chart ≥ 30 FPS
\end{itemize}

\subsection{Khả năng sử dụng (Usability)}

\begin{itemize}[leftmargin=1.5cm]
  \item \textbf{USE-001}: Giao diện trực quan, dễ học cho sinh viên mới
  \item \textbf{USE-002}: Tài liệu hướng dẫn chi tiết với ví dụ
  \item \textbf{USE-003}: Hỗ trợ multiple language (Vietnamese, English)
  \item \textbf{USE-004}: Tùy chỉnh giao diện (theme, font size)
\end{itemize}

\subsection{Độ tin cậy (Reliability)}

\begin{itemize}[leftmargin=1.5cm]
  \item \textbf{REL-001}: Kết quả tính toán phải chính xác tới 2 chữ số thập phân
  \item \textbf{REL-002}: Không crash trên input hợp lệ
  \item \textbf{REL-003}: Xử lý gracefully các input không hợp lệ
  \item \textbf{REL-004}: Recovery từ file JSON lỗi với thông báo rõ ràng
\end{itemize}

\subsection{Duy trì (Maintainability)}

\begin{itemize}[leftmargin=1.5cm]
  \item \textbf{MAINT-001}: Code theo chuẩn C++17, SOLID principles
  \item \textbf{MAINT-002}: Mô-đun hóa: mỗi khối có giao diện rõ ràng
  \item \textbf{MAINT-003}: Unit test coverage ≥ 70\%
  \item \textbf{MAINT-004}: Inline documentation + separate docs
  \item \textbf{MAINT-005}: Build system (CMake) dễ cấu hình trên nhiều nền tảng
\end{itemize}

\subsection{Bảo mật (Security)}

\begin{itemize}[leftmargin=1.5cm]
  \item \textbf{SEC-001}: Input validation trên tất cả user input
  \item \textbf{SEC-002}: Không bị buffer overflow
  \item \textbf{SEC-003}: Xử lý file JSON an toàn
\end{itemize}

\subsection{Tính khả chuyển (Portability)}

\begin{itemize}[leftmargin=1.5cm]
  \item \textbf{PORT-001}: Chạy trên Windows (MSVC), Linux (GCC/Clang), macOS (Clang)
  \item \textbf{PORT-002}: Hỗ trợ 64-bit, 32-bit (nếu có)
  \item \textbf{PORT-003}: Không phụ thuộc vào thư viện OS-specific
\end{itemize}

\section{Các ràng buộc và giới hạn}

\subsection{Giới hạn chức năng}

\begin{table}[H]
\centering
\caption{Giới hạn chức năng của hệ thống}
\begin{tabular}{ll}
\toprule
\textbf{Chỉ tiêu} & \textbf{Giới hạn} \\
\midrule
Số tiến trình tối đa & 1000 \\
Thời gian burst tối đa & 10000 units \\
Số khung bộ nhớ tối đa & 10000 \\
Độ dài reference string & 100000 \\
Quantum RR & 1 đến 1000 units \\
Mức ưu tiên & 1 đến 40 \\
\bottomrule
\end{tabular}
\end{table}

\subsection{Giới hạn kỹ thuật}

\begin{itemize}[leftmargin=1.5cm]
  \item \textbf{Ngôn ngữ}: C++17 trở lên (C++20 không yêu cầu)
  \item \textbf{Compiler}: GCC 7+, Clang 5+, MSVC 2017+
  \item \textbf{RAM}: Tối thiểu 512MB (khuyến cáo 2GB)
  \item \textbf{Disk}: 100MB để cài đặt (có GUI)
  \item \textbf{GPU}: Không yêu cầu (GUI dùng software rendering)
\end{itemize}

\subsection{Ràng buộc phát triển}

\begin{itemize}[leftmargin=1.5cm]
  \item Dự án là open-source (license MIT)
  \item Không sử dụng thư viện GPL để tránh ràng buộc GPL
  \item Tất cả dependencies phải được ghi rõ trong CMakeLists.txt
  \item Thường xuyên cập nhật documentation
\end{itemize}

\section{Tiêu chí chấp nhận}

Dự án được coi là hoàn thành khi:

\begin{enumerate}[leftmargin=1.5cm]
  \item ✓ Tất cả yêu cầu chức năng đạt (CPU-001 → SHELL-008)
  \item ✓ Kiểm thử chi tiết (unit test, integration test, UAT)
  \item ✓ Tất cả lỗi nghiêm trọng được sửa
  \item ✓ Documentation hoàn chỉnh (user guide, API doc, architecture)
  \item ✓ Hiệu năng đạt mục tiêu (< 100ms cho CPU, < 50ms cho Memory)
  \item ✓ GUI chạy mượt mà, không flickering
  \item ✓ Cross-platform testing thành công
  \item ✓ Code review pass
  \item ✓ Báo cáo đồ án hoàn chỉnh, trình bày rõ ràng
\end{enumerate}

\clearpage


% ============================================================================
% CHƯƠNG 4: KIẾN TRÚC HỆ THỐNG
% ============================================================================

\chapter{Kiến trúc hệ thống}
\label{ch:architecture}

\section{Tổng quan kiến trúc}

OSPlayground được thiết kế theo mô hình \textbf{mô-đun hóa (modular architecture)}, 
trong đó mỗi khối chức năng độc lập nhưng liên kết với nhau qua interface rõ ràng.

\subsection{Sơ đồ kiến trúc cấp cao}

\begin{figure}[H]
\centering
\begin{tikzpicture}[>=Stealth, node distance=1cm, every node/.style={draw, rounded corners, align=center, minimum width=3cm, minimum height=1cm}]
  % Layer 1 - UI
  \node[fill=blue!20] (cli) {CLI\\Engine};
  \node[fill=blue!20, right=1.5cm of cli] (tui) {TUI\\Dashboard};
  \node[fill=blue!20, right=1.5cm of tui] (gui) {GUI\\SDL2/ImGui};
  
  % Layer 2 - Core Modules
  \node[fill=green!20, below=2cm of cli] (cpu_mod) {CPU\\Scheduler};
  \node[fill=green!20, right=1.5cm of cpu_mod] (mem_mod) {Memory\\Manager};
  \node[fill=green!20, right=1.5cm of mem_mod] (ipc_mod) {IPC\\Pipe/Signal};
  \node[fill=green!20, right=1.5cm of ipc_mod] (shell_mod) {Shell\\Parser};
  
  % Layer 3 - Core System
  \node[fill=yellow!20, below=2cm of mem_mod, minimum width=12cm] (core_sys) {Core System (CPU/MEM/IPC Stats, System State)};
  
  % Layer 4 - Utils
  \node[fill=gray!20, below=1.5cm of core_sys, minimum width=12cm] (utils) {Utils (Logging, File I/O, JSON Parsing)};
  
  % Connections from UI to Modules
  \draw[->] (cli) -- (cpu_mod);
  \draw[->] (tui) -- (mem_mod);
  \draw[->] (gui) -- (ipc_mod);
  \draw[->] (gui) -- (shell_mod);
  
  % Connections from Modules to Core
  \draw[->] (cpu_mod) -- (core_sys);
  \draw[->] (mem_mod) -- (core_sys);
  \draw[->] (ipc_mod) -- (core_sys);
  \draw[->] (shell_mod) -- (core_sys);
  
  % Connections from Core to Utils
  \draw[->] (core_sys) -- (utils);
  
  % Back connections
  \draw[<-] (shell_mod) -- (core_sys);
  
\end{tikzpicture}
\caption{Sơ đồ kiến trúc cấp cao OSPlayground}
\label{fig:arch_highlevel}
\end{figure}

\section{Cấu trúc thư mục dự án}

\begin{figure}[H]
\centering
\begin{tikzpicture}[font=\ttfamily\small]
\node[anchor=west] (root) at (0,0) {OSPlayground/};
\node[anchor=west] at (1,-0.5) {├── CMakeLists.txt};
\node[anchor=west] at (1,-1) {├── main.cpp};
\node[anchor=west] at (1,-1.5) {├── assets/};
\node[anchor=west] at (2,-2) {│   ├── scenarios/};
\node[anchor=west] at (3,-2.4) {│   │   └── demo\_cpu.json};
\node[anchor=west] at (2,-2.8) {│   └── logs/};
\node[anchor=west] at (1,-3.2) {├── engine/};
\node[anchor=west] at (2,-3.6) {│   ├── loop.cpp, loop.h};
\node[anchor=west] at (1,-4) {├── sim/};
\node[anchor=west] at (2,-4.4) {│   ├── cpu/ (Schedulers)};
\node[anchor=west] at (2,-4.8) {│   ├── mem/ (Paging/Segmentation)};
\node[anchor=west] at (2,-5.2) {│   ├── ipc/ (Pipe/Signal)};
\node[anchor=west] at (2,-5.6) {│   ├── shell/ (Shell Parser)};
\node[anchor=west] at (2,-6) {│   └── core/ (System State)};
\node[anchor=west] at (1,-6.4) {├── ui/};
\node[anchor=west] at (2,-6.8) {│   ├── cli/ (CLI Interface)};
\node[anchor=west] at (2,-7.2) {│   ├── tui/ (TUI Dashboard)};
\node[anchor=west] at (2,-7.6) {│   └── gfx/ (GUI SDL2/ImGui)};
\node[anchor=west] at (1,-8) {├── utils/};
\node[anchor=west] at (2,-8.4) {│   ├── log.cpp, log.h};
\node[anchor=west] at (2,-8.8) {│   └── file.cpp, file.h};
\node[anchor=west] at (1,-9.2) {├── tests/};
\node[anchor=west] at (2,-9.6) {│   ├── cpu\_test.cpp};
\node[anchor=west] at (2,-10) {│   ├── mem\_test.cpp};
\node[anchor=west] at (2,-10.4) {│   └── ipc\_test.cpp};
\node[anchor=west] at (1,-10.8) {└── docs/};
\node[anchor=west] at (2,-11.2) {    └── report/ (Báo cáo này)};
\end{tikzpicture}
\caption{Cấu trúc thư mục dự án}
\label{fig:directory_structure}
\end{figure}

\section{Các mô-đun chính}

\subsection{1. Mô-đul CPU Scheduler (\texttt{sim/cpu/})}

\textbf{Trách nhiệm}: Cài đặt các thuật toán lập lịch CPU.

\textbf{Giao diện chính}:
\begin{lstlisting}[language=C++,caption={CPU Scheduler API}]
class Process {
  int pid, arrival, burst, priority;
  int wait_time, turnaround_time, start, finish;
};

struct GanttEntry { int pid; int start; int finish; };

class Scheduler {
  Result runFCFS(const vector<Process>& procs);
  Result runSJF(const vector<Process>& procs);
  Result runSRTF(const vector<Process>& procs);
  Result runRR(const vector<Process>& procs, int quantum);
  Result runPriorityNP(const vector<Process>& procs);
  Result runPriorityP(const vector<Process>& procs);
  
  vector<GanttEntry> getGantt() const;
  double getAWT() const;
  double getATT() const;
};

struct Result {
  double awt, att;
  vector<GanttEntry> gantt;
  vector<string> log;  // Chi tiết từng bước
};
\end{lstlisting}

\subsection{2. Mô-đul Memory Manager (\texttt{sim/mem/})}

\textbf{Trách nhiệm}: Mô phỏng quản lý bộ nhớ (paging/segmentation, thay trang).

\textbf{Giao diện chính}:
\begin{lstlisting}[language=C++,caption={Memory Manager API}]
class MMU {  // Memory Management Unit
  PhysicalAddress translate(LogicalAddress logical);
  bool isValidAddress(LogicalAddress addr) const;
};

class PageReplacer {
  void loadPage(int page, int physicalFrame);
  int selectVictim(int currentTime);  // Tuỳ theo chính sách
};

class MemorySimulator {
  Result runFIFO(int frames, vector<int> references);
  Result runLRU(int frames, vector<int> references);
  Result runOPT(int frames, vector<int> references);
  
  int getFaultCount() const;
  double getFaultRate() const;
  const vector<vector<int>>& getFrameStates() const;
};

struct Result {
  int faults, steps;
  vector<vector<int>> frame_states;  // [step][frame] = page_id
  vector<string> log;
};
\end{lstlisting}

\subsection{3. Mô-đul IPC (\texttt{sim/ipc/})}

\textbf{Trách nhiệm}: Mô phỏng giao tiếp liên tiến trình (Pipe, Signal).

\textbf{Giao diện chính}:
\begin{lstlisting}[language=C++,caption={IPC API}]
class Pipe {
  void write(const string& data);
  string read(int count);
  bool isEmpty() const;
  bool isFull() const;
};

enum Signal { SIGINT, SIGUSR1, SIGKILL };

class SignalHandler {
  void sendSignal(int pid, Signal sig);
  void handleSignal(int pid, Signal sig);
};

class IPCManager {
  void createPipe(int id);
  void sendSignal(int pid, Signal sig);
  const vector<SignalEvent>& getEvents() const;
};

struct SignalEvent {
  int pid, time;
  Signal sig;
  bool handled;
};
\end{lstlisting}

\subsection{4. Mô-đul Shell (\texttt{sim/shell/})}

\textbf{Trách nhiệm}: Phân tích cú pháp dòng lệnh và thực thi lệnh.

\textbf{Giao diện chính}:
\begin{lstlisting}[language=C++,caption={Shell API}]
class ShellParser {
  vector<Command> parseLine(const string& line);
  bool isValidPipeline(const vector<Command>& cmds) const;
};

class ShellExecutor {
  vector<string> executePipeline(const vector<Command>& cmds);
  vector<string> executeCommand(const Command& cmd);
};

struct Command {
  string name;  // "ps", "cpu", "kill", ...
  vector<string> args;
};
\end{lstlisting}

\subsection{5. Mô-đul Core System (\texttt{sim/core/})}

\textbf{Trách nhiệm}: Quản lý trạng thái hệ thống toàn cục.

\textbf{Giao diện chính}:
\begin{lstlisting}[language=C++,caption={Core System API}]
class SystemState {
  // Process management
  int createProcess(const Process& p);
  void killProcess(int pid);
  vector<Process> getAllProcesses() const;
  
  // Scoring
  double computeCPUScore() const;
  double computeMemoryScore() const;
  double computeIPCScore() const;
  double computeSystemScore() const;  // Weighted average
  
  // State
  const Scheduler& getCPUScheduler() const;
  const MemorySimulator& getMemory() const;
  const IPCManager& getIPC() const;
};
\end{lstlisting}

\section{Luồng dữ liệu}

\subsection{Luồng chạy CPU Scheduling}

\begin{enumerate}[leftmargin=1.5cm]
  \item \textbf{Input}: Danh sách tiến trình từ kịch bản JSON
  \item \textbf{Parsing}: CLI/GUI đọc JSON, tạo objects Process
  \item \textbf{Execution}: Scheduler.run*() xử lý, tạo Gantt, tính AWT/ATT
  \item \textbf{Output}: Kết quả được lưu vào SystemState
  \item \textbf{Visualization}: GUI vẽ Gantt chart, hiển thị chỉ số
  \item \textbf{Logging}: Tất cả chi tiết được ghi vào log
\end{enumerate}

\subsection{Luồng chạy Memory Simulation}

\begin{enumerate}[leftmargin=1.5cm]
  \item \textbf{Input}: Số khung (frames) và reference string từ JSON
  \item \textbf{Parsing}: Chuyển đổi input thành dạng xử lý được
  \item \textbf{Simulation}: MemorySimulator chạy chính sách thay trang, tính fault
  \item \textbf{State Tracking}: Lưu trạng thái khung ở mỗi bước
  \item \textbf{Output}: Fault count, frame\_states, log
  \item \textbf{Visualization}: GUI vẽ lưới khung trang
\end{enumerate}

\subsection{Luồng xử lý Shell Command}

\begin{enumerate}[leftmargin=1.5cm]
  \item \textbf{Input}: Người dùng gõ lệnh vào CLI
  \item \textbf{Parsing}: ShellParser tách lệnh và argument
  \item \textbf{Validation}: Kiểm tra cú pháp, số lượng argument
  \item \textbf{Execution}: ShellExecutor gọi hàm tương ứng
  \item \textbf{Pipeline}: Nếu có \verb+|+, truyền output qua các lệnh
  \item \textbf{Output}: In kết quả lên CLI
\end{enumerate}

\section{Các design patterns sử dụng}

\subsection{1. Strategy Pattern}

Các thuật toán lập lịch được triển khai dưới dạng các strategy khác nhau:

\begin{itemize}[leftmargin=1.5cm]
  \item Giao diện chung: \texttt{ISchedulingStrategy}
  \item Implementations: \texttt{FCFSStrategy}, \texttt{SJFStrategy}, ...
  \item Context: \texttt{Scheduler} chọn strategy theo parameter
\end{itemize}

\subsection{2. Factory Pattern}

Tạo đối tượng Scheduler, MemorySimulator, IPCManager:

\begin{itemize}[leftmargin=1.5cm]
  \item \texttt{SchedulerFactory::create(type, params)}
  \item \texttt{MemoryFactory::create(type, frames)}
\end{itemize}

\subsection{3. Observer Pattern}

GUI theo dõi sự thay đổi của SystemState:

\begin{itemize}[leftmargin=1.5cm]
  \item SystemState phát ra events
  \item GUI panels đăng ký nhận events
  \item Khi dữ liệu thay đổi, panels tự động cập nhật
\end{itemize}

\subsection{4. Command Pattern}

Shell commands được biểu diễn dưới dạng Command objects:

\begin{itemize}[leftmargin=1.5cm]
  \item \texttt{ICommand} interface
  \item Implementations: \texttt{PSCommand}, \texttt{CPUCommand}, ...
  \item \texttt{CommandInvoker} thực thi commands
\end{itemize}

\section{Giao tiếp giữa các mô-đul}

\subsection{Từ CPU sang Core System}

\begin{itemize}[leftmargin=1.5cm]
  \item Scheduler.run*() hoàn thành → kết quả
  \item SystemState::setCPUResult() lưu kết quả
  \item Tính CPU Score dựa trên AWT/ATT
\end{itemize}

\subsection{Từ Memory sang Core System}

\begin{itemize}[leftmargin=1.5cm]
  \item MemorySimulator.run*() hoàn thành → frame\_states, faults
  \item SystemState::setMemoryResult() lưu kết quả
  \item Tính Memory Score dựa trên fault rate
\end{itemize}

\subsection{Từ IPC sang Core System}

\begin{itemize}[leftmargin=1.5cm]
  \item IPCManager ghi nhận signal events
  \item SystemState theo dõi events
  \item Tính IPC Score dựa trên số kills
\end{itemize}

\subsection{Từ Shell sang Core System}

\begin{itemize}[leftmargin=1.5cm]
  \item Shell parser các lệnh, cập nhật SystemState trực tiếp
  \item Ví dụ: \verb|kill 5| → gọi SystemState::killProcess(5)
  \item Ví dụ: \verb|cpu srtf| → gọi Scheduler.runSRTF()
\end{itemize}

\section{Công thức hệ thống điểm}

Hệ thống tính điểm dựa trên ba thành phần:

\begin{equation}
\text{CPU Score} = 100 \cdot \frac{B}{B + (\text{AWT} + \text{ATT})}, \quad B = 20
\end{equation}

\begin{equation}
\text{Memory Score} = 100 \times \left(1 - \frac{\text{faults}}{\text{steps}}\right)
\end{equation}

\begin{equation}
\text{IPC Score} = \max\left(0, 100 - 15 \times \text{kills}\right)
\end{equation}

\begin{equation}
\text{System Stability Score} = 0.4 \times \text{CPU} + 0.4 \times \text{Memory} + 0.2 \times \text{IPC}
\end{equation}

\clearpage


% ============================================================================
% CHƯƠNG 5: CHI TIẾT CPU SCHEDULING
% ============================================================================

\chapter{Thiết kế chi tiết - CPU Scheduling}
\label{ch:cpu_detail}

\section{Mô hình dữ liệu}

\subsection{Struct Process}

\begin{lstlisting}[language=C++,caption={Cấu trúc Process}]
struct Process {
  int pid;              // Process ID (1-based)
  int arrival;          // Thời điểm đến CPU ready queue
  int burst;            // Thời gian CPU cần
  int priority;         // Mức ưu tiên (1-40, nhỏ hơn = cao hơn)
  
  // Kết quả tính toán
  int start_time;       // Lúc bắt đầu chạy
  int finish_time;      // Lúc hoàn thành
  int wait_time;        // start_time - arrival
  int turnaround_time;  // finish_time - arrival
  
  // Trạng thái
  ProcessState state;   // New, Ready, Running, Terminated
  int remaining_time;   // Thời gian còn lại (cho preemptive)
};
\end{lstlisting}

\subsection{Struct GanttEntry và Result}

\begin{lstlisting}[language=C++,caption={Gantt Chart và Result}]
struct GanttEntry {
  int pid;       // Process ID
  int start;     // Start time
  int finish;    // Finish time
  
  GanttEntry(int p, int s, int f) 
    : pid(p), start(s), finish(f) {}
};

struct Result {
  double awt;                    // Average Waiting Time
  double att;                    // Average Turnaround Time
  vector<GanttEntry> gantt;      // Biểu đồ Gantt
  vector<Process> processes;     // Danh sách tiến trình có kết quả
  vector<string> execution_log;  // Log chi tiết từng bước
  int total_time;                // Tổng thời gian simulation
  int cpu_idle_time;             // Thời gian CPU rỗi
  double cpu_utilization;        // % sử dụng CPU
};
\end{lstlisting}

\section{FCFS - First Come First Serve}

\subsection{Nguyên lý}

FCFS sắp xếp các tiến trình theo thứ tự arrival time. Tiến trình nào đến trước phục vụ trước.

\textbf{Tính chất}:
\begin{itemize}[leftmargin=1.5cm]
  \item Non-preemptive: Không ngắt tiến trình đang chạy
  \item FIFO: Kiểu hàng đợi tự nhiên
  \item Dễ cài đặt nhưng dễ gây Convoy Effect
\end{itemize}

\subsection{Giả mã và cài đặt}

\begin{lstlisting}[language=C++,caption={Cài đặt FCFS}]
Result Scheduler::runFCFS(const vector<Process>& procs) {
  vector<Process> processes = procs;
  
  // Sắp xếp theo arrival time, rồi PID
  sort(processes.begin(), processes.end(), 
    [](const Process& a, const Process& b) {
      if (a.arrival != b.arrival) return a.arrival < b.arrival;
      return a.pid < b.pid;
    });
  
  Result result;
  result.gantt.clear();
  
  int current_time = 0;
  double sum_awt = 0, sum_att = 0;
  
  for (auto& p : processes) {
    // Nếu CPU rỗi, chờ tới arrival time tiếp theo
    if (current_time < p.arrival) {
      result.execution_log.push_back(
        "CPU idle from " + to_string(current_time) 
        + " to " + to_string(p.arrival));
      current_time = p.arrival;
    }
    
    // Tiến trình p chạy
    p.start_time = current_time;
    p.finish_time = current_time + p.burst;
    p.wait_time = p.start_time - p.arrival;
    p.turnaround_time = p.finish_time - p.arrival;
    
    result.gantt.push_back(
      GanttEntry(p.pid, p.start_time, p.finish_time));
    
    result.execution_log.push_back(
      "P" + to_string(p.pid) + ": start=" + to_string(p.start_time)
      + ", finish=" + to_string(p.finish_time)
      + ", wait=" + to_string(p.wait_time));
    
    current_time = p.finish_time;
    sum_awt += p.wait_time;
    sum_att += p.turnaround_time;
  }
  
  result.awt = sum_awt / processes.size();
  result.att = sum_att / processes.size();
  result.processes = processes;
  result.total_time = current_time;
  
  return result;
}
\end{lstlisting}

\section{SJF - Shortest Job First}

\subsection{Nguyên lý}

Tại mỗi lần CPU rỗi, chọn tiến trình có burst time nhỏ nhất trong số những tiến trình 
đã đến. SJF tối ưu hoá AWT trong tất cả non-preemptive algorithms.

\subsection{Giả mã}

\begin{lstlisting}[language=C++,caption={Cài đặt SJF}]
Result Scheduler::runSJF(const vector<Process>& procs) {
  vector<Process> processes = procs;
  Result result;
  
  vector<bool> done(processes.size(), false);
  int current_time = 0;
  int completed = 0;
  
  while (completed < processes.size()) {
    // Tìm tiến trình chưa xong có burst nhỏ nhất, arrival <= current_time
    int next_idx = -1;
    int min_burst = INT_MAX;
    
    for (int i = 0; i < processes.size(); ++i) {
      if (!done[i] && processes[i].arrival <= current_time) {
        if (processes[i].burst < min_burst) {
          min_burst = processes[i].burst;
          next_idx = i;
        }
      }
    }
    
    if (next_idx == -1) {
      // Không có tiến trình sẵn sàng, CPU rỗi
      // Tìm arrival time sớm nhất tiếp theo
      int next_arrival = INT_MAX;
      for (int i = 0; i < processes.size(); ++i) {
        if (!done[i] && processes[i].arrival > current_time) {
          next_arrival = min(next_arrival, processes[i].arrival);
        }
      }
      if (next_arrival != INT_MAX) {
        result.execution_log.push_back(
          "CPU idle from " + to_string(current_time) 
          + " to " + to_string(next_arrival));
        current_time = next_arrival;
      }
    } else {
      // Chạy tiến trình next_idx
      Process& p = processes[next_idx];
      p.start_time = current_time;
      p.finish_time = current_time + p.burst;
      p.wait_time = p.start_time - p.arrival;
      p.turnaround_time = p.finish_time - p.arrival;
      
      result.gantt.push_back(
        GanttEntry(p.pid, p.start_time, p.finish_time));
      
      result.execution_log.push_back(
        "P" + to_string(p.pid) + ": burst=" + to_string(p.burst));
      
      current_time = p.finish_time;
      done[next_idx] = true;
      completed++;
    }
  }
  
  // Tính trung bình
  double sum_awt = 0, sum_att = 0;
  for (const auto& p : processes) {
    sum_awt += p.wait_time;
    sum_att += p.turnaround_time;
  }
  result.awt = sum_awt / processes.size();
  result.att = sum_att / processes.size();
  result.processes = processes;
  result.total_time = current_time;
  
  return result;
}
\end{lstlisting}

\section{SRTF - Shortest Remaining Time First}

\subsection{Nguyên lý}

Phiên bản preemptive của SJF. Tại mỗi tick thời gian (hoặc khi tiến trình mới đến), 
CPU chọn tiến trình có remaining time nhỏ nhất. Nếu tiến trình mới có remaining time 
nhỏ hơn tiến trình đang chạy, ngắt tiến trình hiện tại.

\subsection{Đặc điểm}

\begin{itemize}[leftmargin=1.5cm]
  \item Tối ưu hoá ATT trong tất cả preemptive algorithms
  \item Chi phí context switch cao nếu mô phỏng quá chi tiết (mỗi tick)
  \item Gây starvation cho tiến trình dài
  \item Yêu cầu biết trước burst time (không thực tế)
\end{itemize}

\section{Round Robin (RR)}

\subsection{Nguyên lý}

Mỗi tiến trình được cấp CPU một time slice (quantum) cố định. Nếu tiến trình không 
hoàn thành trong quantum, nó được đặt ở cuối hàng đợi sẵn sàng.

\begin{lstlisting}[language=C++,caption={Cài đặt Round Robin}]
Result Scheduler::runRR(const vector<Process>& procs, int quantum) {
  vector<Process> processes = procs;
  Result result;
  
  queue<int> ready_queue;
  vector<int> remaining_time(processes.size());
  
  for (int i = 0; i < processes.size(); ++i) {
    remaining_time[i] = processes[i].burst;
  }
  
  int current_time = 0;
  int completed = 0;
  
  // Đưa tất cả tiến trình arrive tại thời điểm 0 vào hàng đợi
  for (int i = 0; i < processes.size(); ++i) {
    if (processes[i].arrival == current_time) {
      ready_queue.push(i);
    }
  }
  
  while (completed < processes.size()) {
    if (ready_queue.empty()) {
      // Tìm tiến trình sớm nhất arrive
      int next_arrival = INT_MAX;
      int next_idx = -1;
      for (int i = 0; i < processes.size(); ++i) {
        if (processes[i].arrival > current_time && 
            remaining_time[i] > 0) {
          if (processes[i].arrival < next_arrival) {
            next_arrival = processes[i].arrival;
            next_idx = i;
          }
        }
      }
      if (next_idx != -1) {
        current_time = next_arrival;
        ready_queue.push(next_idx);
      }
    } else {
      int idx = ready_queue.front();
      ready_queue.pop();
      
      int time_slice = min(quantum, remaining_time[idx]);
      processes[idx].start_time = (processes[idx].start_time == 0) 
                                   ? current_time 
                                   : processes[idx].start_time;
      
      current_time += time_slice;
      remaining_time[idx] -= time_slice;
      
      // Đưa các tiến trình arrive vào hàng đợi
      for (int i = 0; i < processes.size(); ++i) {
        if (processes[i].arrival == current_time && 
            remaining_time[i] > 0 && i != idx) {
          ready_queue.push(i);
        }
      }
      
      if (remaining_time[idx] == 0) {
        processes[idx].finish_time = current_time;
        processes[idx].wait_time = processes[idx].finish_time 
                                  - processes[idx].arrival 
                                  - processes[idx].burst;
        processes[idx].turnaround_time = processes[idx].finish_time 
                                        - processes[idx].arrival;
        result.gantt.push_back(
          GanttEntry(processes[idx].pid, 
                     processes[idx].start_time, 
                     current_time));
        completed++;
      } else {
        result.gantt.push_back(
          GanttEntry(processes[idx].pid, 
                     current_time - time_slice, 
                     current_time));
        ready_queue.push(idx);
      }
    }
  }
  
  // Tính trung bình
  double sum_awt = 0, sum_att = 0;
  for (const auto& p : processes) {
    sum_awt += p.wait_time;
    sum_att += p.turnaround_time;
  }
  result.awt = sum_awt / processes.size();
  result.att = sum_att / processes.size();
  result.processes = processes;
  result.total_time = current_time;
  
  return result;
}
\end{lstlisting}

\subsection{Ảnh hưởng của quantum}

\begin{table}[H]
\centering
\caption{Ảnh hưởng của Quantum trên RR}
\begin{tabular}{lll}
\toprule
\textbf{Quantum} & \textbf{Ưu điểm} & \textbf{Nhược điểm} \\
\midrule
Quá nhỏ (1-2 units) & Response time tốt & Context switch overhead cao \\
Vừa phải (4-8 units) & Cân bằng & Phù hợp hầu hết trường hợp \\
Quá lớn (>100 units) & Overhead thấp & Giống FCFS hơn, response tệ \\
\bottomrule
\end{tabular}
\label{tab:quantum_effect}
\end{table}

\section{Priority Scheduling}

\subsection{Non-preemptive Priority}

\begin{itemize}[leftmargin=1.5cm]
  \item Mỗi tiến trình có priority number (1-40, nhỏ = cao)
  \item Chọn tiến trình ưu tiên cao nhất trong hàng đợi sẵn sàng
  \item Chạy cho đến kết thúc
\end{itemize}

\subsection{Preemptive Priority}

\begin{itemize}[leftmargin=1.5cm]
  \item Nếu tiến trình mới đến có priority cao hơn tiến trình đang chạy, ngắt
  \item Tiến trình bị ngắt quay lại hàng đợi sẵn sàng
\end{itemize}

\subsection{Vấn đề Starvation và Aging}

\textbf{Starvation}: Tiến trình ưu tiên thấp có thể chờ vô hạn.

\textbf{Giải pháp Aging}: Tăng dần priority của tiến trình theo thời gian chờ.

\begin{equation}
\text{current\_priority}(t) = \text{original\_priority} - \left\lfloor \frac{t}{k} \right\rfloor
\end{equation}

Với $k$ là hằng số aging factor (ví dụ: k=10).

\section{Vẽ Gantt Chart}

\subsection{ASCII Gantt (CLI)}

\begin{lstlisting}[language=C++,caption={Vẽ Gantt ASCII}]
void printGanttASCII(const vector<GanttEntry>& gantt) {
  // Vẽ timeline
  cout << "\nGantt Chart:\n";
  cout << " ";
  for (int t = 0; t <= getMaxTime(gantt); ++t) {
    cout << setw(3) << t;
  }
  cout << "\n";
  
  // Vẽ bar
  cout << " |";
  for (const auto& entry : gantt) {
    int width = entry.finish - entry.start;
    cout << setw(width) << ("P" + to_string(entry.pid)) << "|";
  }
  cout << "\n";
  
  // Vẽ time scale
  cout << " ";
  for (const auto& entry : gantt) {
    cout << entry.start << setw(entry.finish - entry.start) << entry.finish;
  }
  cout << "\n";
}
\end{lstlisting}

\subsection{Vector Graphic Gantt (GUI)}

GUI sử dụng ImGui để vẽ Gantt chart:
\begin{itemize}[leftmargin=1.5cm]
  \item X-axis: time
  \item Y-axis: process ID
  \item Mỗi entry là một hình chữ nhật có màu khác nhau
  \item Hover hiển thị chi tiết entry
  \item Zoom, pan, export hình ảnh
\end{itemize}

\clearpage


% ============================================================================
% CHƯƠNG 6: CHI TIẾT QUẢN LÝ BỘ NHỚ
% ============================================================================

\chapter{Thiết kế chi tiết - Quản lý bộ nhớ}
\label{ch:memory_detail}

\section{Mô hình dữ liệu}

\subsection{Địa chỉ Logic và Vật Lý}

\begin{lstlisting}[language=C++,caption={Địa chỉ Logic và Vật lý}]
struct LogicalAddress {
  int page_number;   // Paging
  int segment_number; // Segmentation
  int offset;        // Offset trong page/segment
};

struct PhysicalAddress {
  int frame_number;
  int offset;
  
  int toLinear(int page_size) const {
    return frame_number * page_size + offset;
  }
};
\end{lstlisting}

\subsection{Bảng Trang}

\begin{lstlisting}[language=C++,caption={Page Table}]
struct PageTableEntry {
  int frame_number;     // Khung chứa trang
  bool present;         // Trang có trong bộ nhớ?
  bool modified;        // Có được sửa?
  int last_access_time; // Lần truy cập cuối (cho LRU)
  int next_access_time; // Lần truy cập tiếp (cho OPT)
};

class PageTable {
private:
  vector<PageTableEntry> table;
  
public:
  bool translate(int page_num, int& frame_num) const {
    if (page_num < 0 || page_num >= table.size()) return false;
    if (!table[page_num].present) return false;
    frame_num = table[page_num].frame_number;
    return true;
  }
  
  void updateEntry(int page_num, const PageTableEntry& entry) {
    table[page_num] = entry;
  }
};
\end{lstlisting}

\section{Dịch địa chỉ}

\subsection{Paging}

\begin{lstlisting}[language=C++,caption={Dịch địa chỉ Paging}]
class MMU {
private:
  PageTable page_table;
  int page_size;
  int num_frames;
  
public:
  MMU(int ps, int nf) : page_size(ps), num_frames(nf) {}
  
  // Dịch logical → physical
  bool translate(int logical_addr, int& physical_addr) const {
    int page_num = logical_addr / page_size;
    int offset = logical_addr % page_size;
    
    // Kiểm tra trang hợp lệ
    int frame_num;
    if (!page_table.translate(page_num, frame_num)) {
      return false;  // Page fault
    }
    
    physical_addr = frame_num * page_size + offset;
    return true;
  }
  
  bool isPageFault(int logical_addr) const {
    int frame_num;
    int page_num = logical_addr / page_size;
    return !page_table.translate(page_num, frame_num);
  }
};
\end{lstlisting}

\section{Chính sách thay trang}

\subsection{FIFO - First In First Out}

\textbf{Nguyên lý}: Loại bỏ trang được nạp vào sớm nhất.

\textbf{Cài đặt}:
\begin{lstlisting}[language=C++,caption={FIFO Page Replacement}]
class FIFOReplacer {
private:
  queue<int> page_queue;        // Queue của pages
  unordered_set<int> in_memory; // Pages trong bộ nhớ
  int num_frames;
  
public:
  FIFOReplacer(int nf) : num_frames(nf) {}
  
  int selectVictim() {
    if (page_queue.empty()) return -1;
    int victim = page_queue.front();
    page_queue.pop();
    in_memory.erase(victim);
    return victim;
  }
  
  void addPage(int page) {
    if (in_memory.find(page) != in_memory.end()) {
      return;  // Page đã có trong bộ nhớ
    }
    
    if (in_memory.size() >= num_frames) {
      selectVictim();  // Đủ bộ nhớ
    }
    
    page_queue.push(page);
    in_memory.insert(page);
  }
};

Result MemorySimulator::runFIFO(int frames, 
                               const vector<int>& references) {
  FIFOReplacer fifo(frames);
  Result result;
  result.frame_states.clear();
  
  vector<int> current_frames;
  int fault_count = 0;
  
  for (int ref : references) {
    if (find(current_frames.begin(), current_frames.end(), ref) 
        == current_frames.end()) {
      // Page fault
      fault_count++;
      
      if (current_frames.size() < frames) {
        current_frames.push_back(ref);
      } else {
        // Loại bỏ trang cũ nhất
        int victim = fifo.selectVictim();
        auto it = find(current_frames.begin(), current_frames.end(), victim);
        *it = ref;
      }
    }
    
    result.frame_states.push_back(current_frames);
    result.execution_log.push_back(
      "Reference " + to_string(ref) + ": " +
      (fault_count > 0 ? "FAULT" : "HIT"));
  }
  
  result.faults = fault_count;
  result.steps = references.size();
  return result;
}
\end{lstlisting}

\subsection{LRU - Least Recently Used}

\textbf{Nguyên lý}: Loại bỏ trang được dùng lâu nhất trong quá khứ.

\textbf{Cách cài đặt}: Dùng \texttt{unordered\_map} để lưu last access time.

\begin{lstlisting}[language=C++,caption={LRU Page Replacement}]
class LRUReplacer {
private:
  unordered_map<int, int> last_used;  // page -> last_time
  unordered_set<int> in_memory;
  int num_frames;
  
public:
  LRUReplacer(int nf) : num_frames(nf) {}
  
  int selectVictim(int current_time) {
    int victim = -1;
    int min_time = INT_MAX;
    
    for (int page : in_memory) {
      if (last_used[page] < min_time) {
        min_time = last_used[page];
        victim = page;
      }
    }
    
    in_memory.erase(victim);
    return victim;
  }
  
  void addPage(int page, int current_time) {
    if (in_memory.find(page) != in_memory.end()) {
      last_used[page] = current_time;
      return;
    }
    
    if (in_memory.size() >= num_frames) {
      selectVictim(current_time);
    }
    
    in_memory.insert(page);
    last_used[page] = current_time;
  }
};

Result MemorySimulator::runLRU(int frames, 
                              const vector<int>& references) {
  LRUReplacer lru(frames);
  Result result;
  
  vector<int> current_frames;
  int fault_count = 0;
  
  for (int step = 0; step < references.size(); ++step) {
    int ref = references[step];
    
    if (find(current_frames.begin(), current_frames.end(), ref) 
        == current_frames.end()) {
      // Page fault
      fault_count++;
      
      if (current_frames.size() < frames) {
        current_frames.push_back(ref);
      } else {
        // Loại bỏ LRU
        int victim = lru.selectVictim(step);
        auto it = find(current_frames.begin(), current_frames.end(), victim);
        *it = ref;
      }
    } else {
      // Hit: cập nhật last_used time
      lru.addPage(ref, step);
    }
    
    result.frame_states.push_back(current_frames);
  }
  
  result.faults = fault_count;
  result.steps = references.size();
  return result;
}
\end{lstlisting}

\subsection{OPT - Optimal}

\textbf{Nguyên lý}: Loại bỏ trang sẽ được dùng xa nhất trong tương lai.

\textbf{Lưu ý}: Chỉ có thể triển khai vì biết trước toàn bộ reference string.

\begin{lstlisting}[language=C++,caption={OPT Page Replacement}]
class OPTReplacer {
private:
  unordered_map<int, int> next_use;  // page -> next_use_time
  unordered_set<int> in_memory;
  int num_frames;
  
  // Tính next use time cho mỗi page
  void computeNextUse(int current_pos, 
                      const vector<int>& references) {
    next_use.clear();
    
    for (int page : in_memory) {
      next_use[page] = INT_MAX;  // Không sử dụng nữa
      
      for (int i = current_pos + 1; i < references.size(); ++i) {
        if (references[i] == page) {
          next_use[page] = i;
          break;
        }
      }
    }
  }
  
public:
  OPTReplacer(int nf) : num_frames(nf) {}
  
  int selectVictim(int current_pos, const vector<int>& references) {
    computeNextUse(current_pos, references);
    
    int victim = -1;
    int max_time = -1;
    
    for (int page : in_memory) {
      if (next_use[page] > max_time) {
        max_time = next_use[page];
        victim = page;
      }
    }
    
    in_memory.erase(victim);
    return victim;
  }
  
  void addPage(int page) {
    in_memory.insert(page);
  }
};

Result MemorySimulator::runOPT(int frames, 
                              const vector<int>& references) {
  OPTReplacer opt(frames);
  Result result;
  
  vector<int> current_frames;
  int fault_count = 0;
  
  for (int step = 0; step < references.size(); ++step) {
    int ref = references[step];
    
    if (find(current_frames.begin(), current_frames.end(), ref) 
        == current_frames.end()) {
      // Page fault
      fault_count++;
      
      if (current_frames.size() < frames) {
        current_frames.push_back(ref);
        opt.addPage(ref);
      } else {
        // Loại bỏ trang sẽ dùng xa nhất
        int victim = opt.selectVictim(step, references);
        auto it = find(current_frames.begin(), current_frames.end(), victim);
        *it = ref;
        opt.addPage(ref);
      }
    }
    
    result.frame_states.push_back(current_frames);
  }
  
  result.faults = fault_count;
  result.steps = references.size();
  return result;
}
\end{lstlisting}

\section{Thống kê và báo cáo}

\subsection{Các chỉ số}

\begin{table}[H]
\centering
\caption{Chỉ số đánh giá chính sách thay trang}
\begin{tabular}{ll}
\toprule
\textbf{Chỉ số} & \textbf{Định nghĩa} \\
\midrule
Faults & Số lần page fault xảy ra \\
Steps & Tổng số reference \\
Fault rate & faults / steps \\
Hit count & steps - faults \\
Hit rate & 1 - fault rate \\
\bottomrule
\end{tabular}
\end{table}

\section{Trực quan hóa}

\subsection{Lưới khung trang (Frame Grid)}

Trong GUI, lưới khung trang hiển thị:
\begin{itemize}[leftmargin=1.5cm]
  \item X-axis: thời gian (step)
  \item Y-axis: khung (frame)
  \item Ô vuông: page ID có trong khung, hoặc trống
  \item Màu sắc: khác nhau cho mỗi page, giúp theo dõi
\end{itemize}

\begin{figure}[H]
\centering
\begin{tikzpicture}[scale=0.8]
  % Draw grid
  \foreach \i in {0,...,3} {
    \node at (-0.8, -\i*0.8) {F\i};
    \foreach \j in {0,...,11} {
      \draw (\j*0.8, -\i*0.8) rectangle ++(0.7, -0.7);
      % Fill với các page ID (placeholder)
      \node[font=\tiny] at (\j*0.8+0.35, -\i*0.8-0.35) {P1};
    }
  }
  
  % Label
  \node at (4.8, 0.5) {Reference string: 7,0,1,2,0,3,0,4,2,3,0,3};
  \node at (-1.5, -2) {Frames};
  \node at (4.8, -2.5) {Time steps};
\end{tikzpicture}
\caption{Lưới khung trang (Frame Grid) minh hoạ}
\label{fig:frame_grid}
\end{figure}

\clearpage


% ============================================================================
% CHƯƠNG 7: CHI TIẾT IPC VÀ SHELL
% ============================================================================

\chapter{Thiết kế chi tiết - IPC và Shell}
\label{ch:ipc_shell}

\section{Inter-Process Communication (IPC)}

\subsection{Pipe}

\subsubsection{Mô hình}

Pipe là một hàng đợi FIFO một chiều:
\begin{itemize}[leftmargin=1.5cm]
  \item \textbf{Write end}: Ghi dữ liệu vào cuối
  \item \textbf{Read end}: Đọc dữ liệu từ đầu
  \item \textbf{Buffer}: Lưu trữ tối đa N bytes
\end{itemize}

\subsubsection{Cài đặt}

\begin{lstlisting}[language=C++,caption={Pipe implementation}]
class Pipe {
private:
  queue<char> buffer;
  const int MAX_SIZE = 1024;
  
public:
  bool write(const string& data) {
    for (char c : data) {
      if (buffer.size() >= MAX_SIZE) {
        return false;  // Buffer đầy, write block
      }
      buffer.push(c);
    }
    return true;
  }
  
  string read(int count) {
    string result;
    while (count > 0 && !buffer.empty()) {
      result += buffer.front();
      buffer.pop();
      count--;
    }
    return result;
  }
  
  bool isEmpty() const { return buffer.empty(); }
  int size() const { return buffer.size(); }
};
\end{lstlisting}

\subsection{Signal}

\subsubsection{Các signal hỗ trợ}

\begin{table}[H]
\centering
\caption{Các signal được hỗ trợ}
\begin{tabular}{lll}
\toprule
\textbf{Signal} & \textbf{Số} & \textbf{Ý nghĩa} \\
\midrule
SIGINT & 2 & Interrupt (Ctrl+C) \\
SIGUSR1 & 10 & User-defined 1 \\
SIGKILL & 9 & Kill (không bắt được) \\
\bottomrule
\end{tabular}
\end{table}

\subsubsection{Cài đặt}

\begin{lstlisting}[language=C++,caption={Signal handling}]
enum class Signal { SIGINT, SIGUSR1, SIGKILL };

struct SignalEvent {
  int pid;
  Signal sig;
  int timestamp;
  bool delivered;
};

class SignalHandler {
private:
  vector<SignalEvent> events;
  
public:
  void sendSignal(int pid, Signal sig, int time) {
    events.push_back({pid, sig, time, false});
    
    // Log event
    string sig_name = (sig == Signal::SIGINT) ? "SIGINT" :
                      (sig == Signal::SIGUSR1) ? "SIGUSR1" :
                      "SIGKILL";
    log("Send " + sig_name + " to PID " + to_string(pid));
  }
  
  void handleSignal(int pid, Signal sig) {
    for (auto& event : events) {
      if (event.pid == pid && event.sig == sig && !event.delivered) {
        event.delivered = true;
        
        if (sig == Signal::SIGKILL) {
          // Tiến trình bị kill
          // Cập nhật trạng thái tiến trình
        }
        break;
      }
    }
  }
  
  const vector<SignalEvent>& getEvents() const {
    return events;
  }
};
\end{lstlisting}

\subsection{IPCManager}

\begin{lstlisting}[language=C++,caption={IPC Manager}]
class IPCManager {
private:
  map<int, Pipe> pipes;
  SignalHandler signal_handler;
  
public:
  void createPipe(int pipe_id) {
    pipes[pipe_id] = Pipe();
  }
  
  void writeToPipe(int pipe_id, const string& data) {
    if (pipes.find(pipe_id) != pipes.end()) {
      pipes[pipe_id].write(data);
    }
  }
  
  string readFromPipe(int pipe_id, int count) {
    if (pipes.find(pipe_id) != pipes.end()) {
      return pipes[pipe_id].read(count);
    }
    return "";
  }
  
  void sendSignal(int pid, Signal sig, int time) {
    signal_handler.sendSignal(pid, sig, time);
  }
  
  void handleSignal(int pid, Signal sig) {
    signal_handler.handleSignal(pid, sig);
  }
  
  const vector<SignalEvent>& getSignalEvents() const {
    return signal_handler.getEvents();
  }
};
\end{lstlisting}

\section{Shell mini}

\subsection{Cú pháp và ngữ pháp}

\subsubsection{EBNF - Extended Backus-Naur Form}

\begin{lstlisting}[caption={EBNF của Shell mini}]
program       ::= command_line EOF
command_line  ::= command ( '|' command )*
command       ::= CMD_NAME [ ARG { ARG } ]
CMD_NAME      ::= 'ps' | 'kill' | 'cpu' | 'mem' | 'cat' 
                | 'grep' | 'wc' | 'echo' | 'sig' | 'ls'
ARG           ::= STRING
STRING        ::= sequence_of_non_whitespace_characters
\end{lstlisting}

\subsection{Các lệnh được hỗ trợ}

\subsubsection{ps - List processes}

\begin{lstlisting}
ps [options]
  -a: Show all processes
  -r: Show running processes only
  (default: show all)
\end{lstlisting}

Ví dụ output:
\begin{verbatim}
PID  ARRIVAL BURST PRIORITY STATE
1    0       10    2        Ready
2    5       15    1        Running
3    10      8     3        Terminated
\end{verbatim}

\subsubsection{kill - Kill a process}

\begin{lstlisting}
kill <PID>
  Gửi SIGKILL đến PID
\end{lstlisting}

\subsubsection{sig - Send signal}

\begin{lstlisting}
sig <PID> <SIGNAME>
  SIGNAME: INT, USR1, KILL
\end{lstlisting}

\subsubsection{cpu - Run CPU scheduler}

\begin{lstlisting}
cpu <ALGORITHM> [QUANTUM]
  ALGORITHM: fcfs, sjf, srtf, rr, prio_np, prio_p
  QUANTUM: cho RR (default: 4)
\end{lstlisting}

\subsubsection{mem - Run memory simulator}

\begin{lstlisting}
mem <POLICY> <FRAMES> [REFERENCES...]
  POLICY: fifo, lru, opt, all
  FRAMES: số khung bộ nhớ
  REFERENCES: chuỗi tham chiếu (cách nhau bằng dấu cách)
\end{lstlisting}

\subsubsection{cat, grep, wc - Text processing}

\begin{lstlisting}
cat [FILE]
  Đọc file hoặc stdin

grep PATTERN
  Lọc dòng chứa pattern

wc [OPTIONS]
  -l: đếm dòng (default)
  -w: đếm từ
  -c: đếm ký tự
\end{lstlisting}

\subsubsection{echo - Print}

\begin{lstlisting}
echo [TEXT...]
  In text ra stdout
\end{lstlisting}

\subsubsection{ls - List files}

\begin{lstlisting}
ls [DIRECTORY]
  Liệt kê file/thư mục
\end{lstlisting}

\subsection{Pipeline}

Pipeline kết nối các lệnh bằng toán tử \verb+|+:

\begin{lstlisting}
command1 | command2 | command3
\end{lstlisting}

Luồng dữ liệu:
\begin{enumerate}[leftmargin=1.5cm]
  \item \texttt{command1} tạo output → vector<string> (mỗi dòng là một string)
  \item Output truyền tới \texttt{command2} làm input
  \item \texttt{command2} xử lý → output mới
  \item \texttt{command3} nhận input từ command2
  \item Kết quả cuối cùng được in ra stdout
\end{enumerate}

\subsection{Cài đặt Shell Parser}

\begin{lstlisting}[language=C++,caption={Shell Parser}]
struct Command {
  string name;
  vector<string> args;
};

class ShellParser {
public:
  vector<Command> parseLine(const string& line) {
    vector<Command> commands;
    
    // Tách bằng '|'
    vector<string> pipeline_parts = split(line, '|');
    
    for (const string& part : pipeline_parts) {
      // Tách bằng whitespace
      vector<string> tokens = split(part, ' ');
      
      if (tokens.empty()) continue;
      
      Command cmd;
      cmd.name = tokens[0];
      for (int i = 1; i < tokens.size(); ++i) {
        cmd.args.push_back(tokens[i]);
      }
      
      commands.push_back(cmd);
    }
    
    return commands;
  }
  
private:
  vector<string> split(const string& s, char delim) {
    vector<string> result;
    string current;
    
    for (char c : s) {
      if (c == delim) {
        if (!current.empty()) {
          result.push_back(current);
          current.clear();
        }
      } else {
        current += c;
      }
    }
    
    if (!current.empty()) {
      result.push_back(current);
    }
    
    return result;
  }
};
\end{lstlisting}

\subsection{Cài đặt Shell Executor}

\begin{lstlisting}[language=C++,caption={Shell Executor}]
class ShellExecutor {
private:
  SystemState& system_state;
  
public:
  ShellExecutor(SystemState& sys) : system_state(sys) {}
  
  vector<string> executePipeline(const vector<Command>& commands) {
    vector<string> output;
    
    for (int i = 0; i < commands.size(); ++i) {
      vector<string> input = output;
      output = executeCommand(commands[i], input);
    }
    
    return output;
  }
  
private:
  vector<string> executeCommand(const Command& cmd, 
                                const vector<string>& input) {
    vector<string> output;
    
    if (cmd.name == "ps") {
      output = executPS(cmd.args);
    } else if (cmd.name == "kill") {
      output = executeKill(cmd.args);
    } else if (cmd.name == "cpu") {
      output = executeCPU(cmd.args);
    } else if (cmd.name == "mem") {
      output = executeMEM(cmd.args);
    } else if (cmd.name == "grep") {
      output = executeGrep(cmd.args, input);
    } else if (cmd.name == "wc") {
      output = executeWC(cmd.args, input);
    } else if (cmd.name == "cat") {
      output = executeCat(cmd.args);
    } else if (cmd.name == "echo") {
      output = executeEcho(cmd.args);
    }
    
    return output;
  }
  
  vector<string> executeGrep(const vector<string>& args, 
                            const vector<string>& input) {
    vector<string> output;
    if (args.empty()) return output;
    
    string pattern = args[0];
    for (const string& line : input) {
      if (line.find(pattern) != string::npos) {
        output.push_back(line);
      }
    }
    return output;
  }
  
  vector<string> executeWC(const vector<string>& args, 
                          const vector<string>& input) {
    vector<string> output;
    
    bool count_lines = true;
    if (!args.empty() && args[0] == "-w") {
      count_lines = false;
    }
    
    if (count_lines) {
      output.push_back(to_string(input.size()));
    } else {
      int total_words = 0;
      for (const string& line : input) {
        total_words += count_words(line);
      }
      output.push_back(to_string(total_words));
    }
    
    return output;
  }
  
  int count_words(const string& s) {
    int count = 0;
    bool in_word = false;
    
    for (char c : s) {
      if (isspace(c)) {
        in_word = false;
      } else if (!in_word) {
        in_word = true;
        count++;
      }
    }
    
    return count;
  }
};
\end{lstlisting}

\section{Ví dụ sử dụng Shell}

\subsection{Ví dụ 1: Liệt kê tiến trình}

\begin{verbatim}
$ ps
PID  ARRIVAL BURST PRIORITY STATE
1    0       10    2        Ready
2    5       15    1        Ready
3    10      8     3        Ready
\end{verbatim}

\subsection{Ví dụ 2: Chạy CPU Scheduler}

\begin{verbatim}
$ cpu rr 4
Running Round Robin with quantum=4 on 3 processes
Average Waiting Time: 8.33
Average Turnaround Time: 18.33
Gantt: P1 | P2 | P3 | P1 | P2
\end{verbatim}

\subsection{Ví dụ 3: Pipeline}

\begin{verbatim}
$ ps | grep "PID" | wc -l
3
\end{verbatim}

Giải thích:
\begin{enumerate}[leftmargin=1.5cm]
  \item \texttt{ps} xuất ra danh sách tiến trình (4 dòng: header + 3 process)
  \item \texttt{grep "PID"} lọc dòng chứa "PID" → 1 dòng (header)
  \item \texttt{wc -l} đếm số dòng → 1
\end{enumerate}

\clearpage


% ============================================================================
% CHƯƠNG 8: HỆ THỐNG ĐIỂM VÀ ĐÁNH GIÁ
% ============================================================================

\chapter{Hệ thống điểm và đánh giá}
\label{ch:scoring}

\section{Khái niệm tổng quát}

OSPlayground sử dụng hệ thống điểm tổng hợp (System Stability Score) để đánh giá 
toàn bộ hiệu năng của mô phỏng hệ thống. Điểm được tính dựa trên ba thành phần:

\begin{enumerate}[leftmargin=1.5cm]
  \item \textbf{CPU Score}: Đánh giá hiệu năng lập lịch CPU
  \item \textbf{Memory Score}: Đánh giá hiệu năng quản lý bộ nhớ
  \item \textbf{IPC Score}: Đánh giá ảnh hưởng của giao tiếp liên tiến trình
\end{enumerate}

Các điểm thành phần được kết hợp với trọng số để tạo ra System Stability Score:

\begin{equation}
\text{TOTAL} = 0.4 \times \text{CPU} + 0.4 \times \text{Memory} + 0.2 \times \text{IPC}
\end{equation}

\section{CPU Score}

\subsection{Công thức}

\begin{equation}
\text{CPU Score} = 100 \cdot \frac{B}{B + (\text{AWT} + \text{ATT})}, \quad B = 20
\end{equation}

Trong đó:
\begin{itemize}[leftmargin=1.5cm]
  \item $B = 20$ là hằng số chuẩn hoá
  \item AWT = Average Waiting Time
  \item ATT = Average Turnaround Time
\end{itemize}

\subsection{Giải thích}

Công thức này được thiết kế để:
\begin{enumerate}[leftmargin=1.5cm]
  \item Khuyến khích AWT và ATT thấp
  \item Giữ điểm trong khoảng [0, 100]
  \item Đạt 50 điểm khi $(AWT + ATT) = B = 20$
  \item Đạt 100 điểm khi AWT + ATT = 0 (lý tưởng)
\end{enumerate}

\subsection{Ví dụ}

\begin{table}[H]
\centering
\caption{Ví dụ tính CPU Score}
\begin{tabular}{cccc}
\toprule
\textbf{AWT} & \textbf{ATT} & \textbf{AWT+ATT} & \textbf{CPU Score} \\
\midrule
0 & 0 & 0 & 100.0 \\
5 & 5 & 10 & 66.7 \\
10 & 10 & 20 & 50.0 \\
15 & 15 & 30 & 40.0 \\
20 & 20 & 40 & 33.3 \\
\bottomrule
\end{tabular}
\end{table}

\subsection{Cài đặt}

\begin{lstlisting}[language=C++,caption={CPU Score Computation}]
double computeCPUScore(double awt, double att) {
  const double B = 20.0;
  double denominator = awt + att;
  
  if (denominator == 0) {
    return 100.0;  // Perfect case
  }
  
  double score = 100.0 * B / (B + denominator);
  return min(score, 100.0);  // Cap at 100
}
\end{lstlisting}

\section{Memory Score}

\subsection{Công thức}

\begin{equation}
\text{Memory Score} = 100 \times \left(1 - \frac{\text{faults}}{\text{steps}}\right)
\end{equation}

Trong đó:
\begin{itemize}[leftmargin=1.5cm]
  \item \texttt{faults} = số lần page fault xảy ra
  \item \texttt{steps} = tổng số memory references
\end{itemize}

\subsection{Giải thích}

\begin{itemize}[leftmargin=1.5cm]
  \item Fault rate = faults / steps (tỷ lệ lỗi trang)
  \item Hit rate = 1 - fault rate (tỷ lệ thành công)
  \item Memory Score = Hit rate × 100
  \item Điểm 100 khi không có fault (tất cả hit)
  \item Điểm 0 khi fault rate = 100\%
\end{itemize}

\subsection{Ví dụ}

\begin{table}[H]
\centering
\caption{Ví dụ tính Memory Score}
\begin{tabular}{ccccc}
\toprule
\textbf{Faults} & \textbf{Steps} & \textbf{Hit Rate} & \textbf{Score} \\
\midrule
0 & 100 & 100\% & 100.0 \\
10 & 100 & 90\% & 90.0 \\
20 & 100 & 80\% & 80.0 \\
50 & 100 & 50\% & 50.0 \\
100 & 100 & 0\% & 0.0 \\
\bottomrule
\end{tabular}
\end{table}

\subsection{Cài đặt}

\begin{lstlisting}[language=C++,caption={Memory Score Computation}]
double computeMemoryScore(int faults, int steps) {
  if (steps == 0) {
    return 100.0;  // No simulation
  }
  
  double fault_rate = static_cast<double>(faults) / steps;
  double hit_rate = 1.0 - fault_rate;
  double score = hit_rate * 100.0;
  
  return max(0.0, min(score, 100.0));
}
\end{lstlisting}

\section{IPC Score}

\subsection{Công thức}

\begin{equation}
\text{IPC Score} = \max\left(0, 100 - 15 \times \text{kills}\right)
\end{equation}

Trong đó:
\begin{itemize}[leftmargin=1.5cm]
  \item \texttt{kills} = số lần SIGKILL được gửi
  \item Hệ số 15 = penalty per kill (có thể tuỳ chỉnh)
\end{itemize}

\subsection{Giải thích}

\begin{itemize}[leftmargin=1.5cm]
  \item Mục tiêu: khuyến khích sử dụng signal có trách nhiệm
  \item Mỗi SIGKILL làm giảm 15 điểm
  \item Khi kills ≥ 7, điểm trở thành 0
  \item SIGINT, SIGUSR1 không phạt (chỉ là thông báo)
\end{itemize}

\subsection{Ví dụ}

\begin{table}[H]
\centering
\caption{Ví dụ tính IPC Score}
\begin{tabular}{cc}
\toprule
\textbf{Kills} & \textbf{IPC Score} \\
\midrule
0 & 100.0 \\
1 & 85.0 \\
2 & 70.0 \\
3 & 55.0 \\
4 & 40.0 \\
5 & 25.0 \\
6 & 10.0 \\
7+ & 0.0 \\
\bottomrule
\end{tabular}
\end{table}

\subsection{Cài đặt}

\begin{lstlisting}[language=C++,caption={IPC Score Computation}]
double computeIPCScore(int kills) {
  const int PENALTY_PER_KILL = 15;
  double score = 100.0 - (kills * PENALTY_PER_KILL);
  return max(0.0, min(score, 100.0));
}
\end{lstlisting}

\section{System Stability Score}

\subsection{Công thức tổng hợp}

\begin{equation}
\text{System Stability Score} = 0.4 \times \text{CPU} + 0.4 \times \text{Memory} + 0.2 \times \text{IPC}
\end{equation}

\subsection{Giải thích trọng số}

\begin{itemize}[leftmargin=1.5cm]
  \item \textbf{CPU (40\%)}: Lập lịch CPU là khía cạnh chính của OS
  \item \textbf{Memory (40\%)}: Quản lý bộ nhớ cũng rất quan trọng
  \item \textbf{IPC (20\%)}: Giao tiếp liên tiến trình ít hơn nhưng cần xem xét
\end{itemize}

\subsection{Ví dụ}

\begin{table}[H]
\centering
\caption{Ví dụ tính System Stability Score}
\begin{tabular}{ccccc}
\toprule
\textbf{CPU} & \textbf{Memory} & \textbf{IPC} & \textbf{Calculation} & \textbf{Total} \\
\midrule
100 & 100 & 100 & $0.4 \times 100 + 0.4 \times 100 + 0.2 \times 100$ & 100.0 \\
80 & 90 & 70 & $0.4 \times 80 + 0.4 \times 90 + 0.2 \times 70$ & 82.0 \\
50 & 50 & 50 & $0.4 \times 50 + 0.4 \times 50 + 0.2 \times 50$ & 50.0 \\
100 & 0 & 100 & $0.4 \times 100 + 0.4 \times 0 + 0.2 \times 100$ & 60.0 \\
\bottomrule
\end{tabular}


% ============================================================================
% CHƯƠNG 9: GIAO DIỆN NGƯỜI DÙNG
% ============================================================================

\chapter{Giao diện người dùng}
\label{ch:ui}

\section{Tổng quan ba lớp giao diện}

OSPlayground cung cấp ba lớp giao diện để phục vụ các nhu cầu khác nhau:

\begin{itemize}[leftmargin=1.5cm]
  \item \textbf{CLI (Command Line Interface)}: Đơn giản, nhanh gọn, dành cho developer
  \item \textbf{TUI (Text User Interface)}: Dashboard tương tác, menu, dành cho học tập
  \item \textbf{GUI (Graphical User Interface)}: Đồ hoạ, trực quan, dành cho trình bày
\end{itemize}

\section{CLI - Command Line Interface}

\subsection{Cấu trúc}

\begin{verbatim}
$ ./osplayground
OSPlayground v1.0.0
=================================
1. CPU Scheduling
2. Memory Management
3. IPC & Signals
4. Shell Terminal
5. View Scores
6. Load Scenario
0. Exit

Choose option: 
\end{verbatim}

\subsection{Menu chính}

\begin{enumerate}[leftmargin=1.5cm]
  \item \textbf{CPU Scheduling}
  \begin{enumerate}
    \item Select algorithm (FCFS, SJF, SRTF, RR, Priority NP, Priority P)
    \item Set quantum (cho RR)
    \item Input processes
    \item View Gantt chart (ASCII)
    \item View AWT, ATT
  \end{enumerate}
  
  \item \textbf{Memory Management}
  \begin{enumerate}
    \item Select policy (FIFO, LRU, OPT, All)
    \item Input frames, reference string
    \item View faults, hit rate
    \item View frame states
  \end{enumerate}
  
  \item \textbf{IPC \& Signals}
  \begin{enumerate}
    \item Send signal (SIGINT, SIGUSR1, SIGKILL)
    \item View events
  \end{enumerate}
  
  \item \textbf{Shell Terminal}
  \begin{enumerate}
    \item Enter commands
    \item Support pipeline
  \end{enumerate}
  
  \item \textbf{View Scores}
  \begin{enumerate}
    \item Show CPU, Memory, IPC scores
    \item Show System Stability Score
  \end{enumerate}
  
  \item \textbf{Load Scenario}
  \begin{enumerate}
    \item Load JSON file
    \item Auto-populate processes/memory config
  \end{enumerate}
\end{enumerate}

\section{TUI - Text User Interface}

\subsection{Thành phần}

TUI cung cấp dashboard tương tác với các panel:

\begin{figure}[H]
\centering
\begin{tikzpicture}[node distance=0.5cm]
  \draw (0,0) rectangle (12,8);
  \draw (0,7) -- (12,7);
  
  \node[anchor=west] at (0.2, 7.5) {\textbf{OSPlayground Dashboard}};
  
  % Top panels
  \draw (0,6) rectangle (4,7);
  \draw (4,6) rectangle (8,7);
  \draw (8,6) rectangle (12,7);
  
  \node[anchor=center] at (2, 6.5) {CPU Score};
  \node[anchor=center] at (6, 6.5) {Memory Score};
  \node[anchor=center] at (10, 6.5) {IPC Score};
  
  % Main panel
  \draw (0,0.5) rectangle (9,6);
  \node[anchor=nw] at (0.1, 5.9) {Main Panel:};
  \node[anchor=nw] at (0.2, 5.5) {1. CPU Scheduler};
  \node[anchor=nw] at (0.2, 5.1) {2. Memory Sim};
  \node[anchor=nw] at (0.2, 4.7) {3. IPC Manager};
  \node[anchor=nw] at (0.2, 4.3) {4. Shell Terminal};
  \node[anchor=nw] at (0.2, 3.9) {5. Scenario Loader};
  
  % Status panel
  \draw (9,0.5) rectangle (12,6);
  \node[anchor=nw] at (9.1, 5.9) {Status:};
  \node[anchor=nw] at (9.2, 5.5) {Processes: 5};
  \node[anchor=nw] at (9.2, 5.1) {Frames: 4};
  \node[anchor=nw] at (9.2, 4.7) {Faults: 0};
  \node[anchor=nw] at (9.2, 4.3) {Signals: 0};
  \node[anchor=nw] at (9.2, 3.9) {System Score: N/A};
  
  % Footer
  \draw (0,0) rectangle (12,0.5);
  \node[anchor=west] at (0.2, 0.25) {Press 'h' for help | 'q' to quit | Use arrow keys to navigate};
\end{tikzpicture}
\caption{Layout TUI Dashboard}
\label{fig:tui_layout}
\end{figure}

\subsection{Tính năng}

\begin{itemize}[leftmargin=1.5cm]
  \item \textbf{Real-time updates}: Cập nhật trạng thái tức thời
  \item \textbf{Keyboard navigation}: Di chuyển bằng arrow keys
  \item \textbf{Panel focus}: Tab để chuyển panel
  \item \textbf{Quick access}: Phím tắt (h=help, q=quit, etc.)
  \item \textbf{Color coding}: Hiển thị trạng thái bằng màu sắc
\end{itemize}

\section{GUI - Graphical User Interface}

\subsection{Công nghệ}

\begin{itemize}[leftmargin=1.5cm]
  \item \textbf{Backend}: SDL2 (cross-platform graphics)
  \item \textbf{UI Framework}: Dear ImGui (immediate mode GUI)
  \item \textbf{Rendering}: Software rendering (không cần GPU)
  \item \textbf{Font}: System font hoặc embedded font
  \item \textbf{Theme}: Dark theme mặc định, support custom theme
\end{itemize}

\subsection{Các panel chính}

\subsubsection{1. CPU Panel}

\begin{figure}[H]
\centering
\begin{tikzpicture}
  \draw (0,0) rectangle (10,6);
  
  \node[anchor=nw] at (0.2, 5.8) {\textbf{CPU Scheduler}};
  
  % Algorithm selection
  \draw (0.2, 4.5) rectangle (3.5, 5.2);
  \node[anchor=nw] at (0.3, 5.0) {Algorithm};
  \node[anchor=nw] at (0.3, 4.7) {(Dropdown)};
  
  % Quantum input
  \draw (3.8, 4.5) rectangle (5.5, 5.2);
  \node[anchor=nw] at (3.9, 5.0) {Quantum};
  \node[anchor=nw] at (3.9, 4.7) {[Input box]};
  
  % Run button
  \draw (5.8, 4.5) rectangle (7, 5.2);
  \node[anchor=center] at (6.4, 4.85) {Run};
  
  % Gantt chart
  \draw (0.2, 1) rectangle (9.8, 4.3);
  \node[anchor=nw] at (0.3, 4.1) {Gantt Chart:};
  \node[anchor=center] at (5, 2.5) {[Gantt visualization here]};
  
  % Results
  \node[anchor=nw] at (0.3, 0.7) {AWT: 8.33  |  ATT: 18.33  |  CPU Score: 72.5};
  
\end{tikzpicture}
\caption{GUI - CPU Panel}
\label{fig:gui_cpu}
\end{figure}

\textbf{Tính năng}:
\begin{itemize}[leftmargin=1.5cm]
  \item Dropdown chọn thuật toán
  \item Input quantum (cho RR)
  \item Gantt chart vẽ động
  \item Hover trên bar hiển thị chi tiết
  \item Zoom in/out
  \item Export chart to PNG
\end{itemize}

\subsubsection{2. Memory Panel}

\begin{figure}[H]
\centering
\begin{tikzpicture}
  \draw (0,0) rectangle (10,6);
  
  \node[anchor=nw] at (0.2, 5.8) {\textbf{Memory Manager}};
  
  % Configuration
  \draw (0.2, 4.8) rectangle (10.2, 5.5);
  \node[anchor=nw] at (0.3, 5.2) {Policy: (Dropdown)};
  \node[anchor=nw] at (2.5, 5.2) {Frames: (Spinner)};
  \node[anchor=nw] at (5, 5.2) {[Run]};
  
  % Frame grid
  \draw (0.2, 1.2) rectangle (10.2, 4.8);
  \node[anchor=nw] at (0.3, 4.6) {Frame States (by time):};
  \node[anchor=center] at (5, 3) {[Grid visualization]};
  
  % Stats
  \node[anchor=nw] at (0.3, 0.8) {Faults: 5 | Hit Rate: 95\% | Fault Rate: 5\%};
  \node[anchor=nw] at (0.3, 0.3) {Memory Score: 95.0};
\end{tikzpicture}
\caption{GUI - Memory Panel}
\label{fig:gui_memory}
\end{figure}

\textbf{Tính năng}:
\begin{itemize}[leftmargin=1.5cm]
  \item Chọn policy (FIFO, LRU, OPT)
  \item Tuỳ chỉnh số frames
  \item Lưới khung trang theo thời gian
  \item Color-coded pages (mỗi page ID có màu khác)
  \item Scroll/pan lưới
  \item Thống kê chi tiết
\end{itemize}

\subsubsection{3. IPC Panel}

\begin{itemize}[leftmargin=1.5cm]
  \item Danh sách các signal events
  \item Pipe status viewer
  \item Send signal form
  \item Event log
\end{itemize}

\subsubsection{4. Scenarios Panel}

\begin{itemize}[leftmargin=1.5cm]
  \item Browse file
  \item Load JSON scenario
  \item Preview processes/memory config
  \item Save results
\end{itemize}

\subsubsection{5. Score Panel}

\begin{itemize}[leftmargin=1.5cm]
  \item Hiển thị CPU, Memory, IPC scores
  \item System Stability Score
  \item Score history (biểu đồ)
  \item Export report
\end{itemize}

\subsection{Thiết kế visual}

\begin{itemize}[leftmargin=1.5cm]
  \item \textbf{Color scheme}: Dark theme (background #1e1e1e, accent #007acc)
  \item \textbf{Typography}: monospace font cho code, sans-serif cho UI
  \item \textbf{Icons}: Simple ASCII-based icons (⚙ ⚡ ▶ ⏹)
  \item \textbf{Layout}: Responsive, adaptive to window size
  \item \textbf{Animation}: Smooth transitions, no flicker
\end{itemize}

\section{Công cụ và thư viện}

\subsection{Phụ thuộc GUI}

\begin{table}[H]
\centering
\caption{Thư viện sử dụng cho GUI}
\begin{tabular}{lll}
\toprule
\textbf{Thư viện} & \textbf{Phiên bản} & \textbf{Mục đích} \\
\midrule
SDL2 & 2.26.0+ & Graphics context, event handling \\
Dear ImGui & 1.89+ & UI components \\
nlohmann/json & 3.11.0+ & JSON parsing \\
fmt & 10.0.0+ & String formatting \\
\bottomrule
\end{tabular}
\end{table}

\section{Keyboard shortcuts}

\begin{table}[H]
\centering
\caption{Phím tắt chính}
\begin{tabular}{ll}
\toprule
\textbf{Phím} & \textbf{Hành động} \\
\midrule
\texttt{Ctrl+Q} / \texttt{ESC} & Quit \\
\texttt{Tab} & Switch panel \\
\texttt{Enter} & Run / Confirm \\
\texttt{Space} & Play/Pause animation \\
\texttt{Ctrl+S} & Save screenshot \\
\texttt{Ctrl+O} & Open scenario \\
\texttt{F1} & Help \\
\texttt{+} / \texttt{-} & Zoom in/out \\
\bottomrule
\end{tabular}
\end{table}

\clearpage


% ============================================================================
% APPENDICES (Placeholder)
% ============================================================================

\appendix

\chapter{Hướng Dẫn Cài Đặt và Chạy}
\label{app:installation}

\section{Yêu cầu hệ thống}

\begin{itemize}[leftmargin=1.5cm]
  \item \textbf{C++ Compiler}: GCC 7+, Clang 5+, hoặc MSVC 2017+
  \item \textbf{CMake}: 3.20 hoặc cao hơn
  \item \textbf{Python}: 3.6+ (cho script build)
  \item \textbf{Tùy chọn}: SDL2 dev library (nếu build GUI)
\end{itemize}

\section{Build trên Windows (MSVC)}

\begin{lstlisting}[language=bash]
# PowerShell
./scripts/build.ps1 -Configure
./scripts/build.ps1 -Build
./scripts/build.ps1 -Run
\end{lstlisting}

\section{Build trên Linux/WSL}

\begin{lstlisting}[language=bash]
# Bash
./scripts/build.sh configure
./scripts/build.sh build
./scripts/build.sh run
\end{lstlisting}

\section{Build với GUI (tùy chọn)}

\begin{lstlisting}[language=bash]
# Cài SDL2 dev
apt-get install libsdl2-dev  # Linux
brew install sdl2            # macOS

# Build với GUI
mkdir build && cd build
cmake -DUSE_GFX_SDL_IMGUI=ON ..
make
./osplayground
\end{lstlisting}

\chapter{Tài liệu tham khảo}
\addcontentsline{toc}{chapter}{Tài liệu tham khảo}

\begin{thebibliography}{9}

\bibitem{silberschatz}
Silberschatz, A., Galvin, P. B., Gagne, G. (2018). 
\textit{Operating System Concepts (10th ed.)}. 
Wiley.

\bibitem{ostep}
Arpaci-Dusseau, R. H., Arpaci-Dusseau, A. C. (2018). 
\textit{Operating Systems: Three Easy Pieces}. 
Arpaci-Dusseau Books.

\bibitem{tanenbaum}
Tanenbaum, A. S., Bos, H. (2014). 
\textit{Modern Operating Systems (4th ed.)}. 
Pearson.

\bibitem{imgui}
Cornut, O. (2023). 
\textit{Dear ImGui}. 
GitHub: \url{https://github.com/ocornut/imgui}

\bibitem{sdl2}
SDL2 Project (2023).
\textit{Simple DirectMedia Layer 2.0}.
\url{https://www.libsdl.org/}

\bibitem{cmake}
CMake Project. (2023).
\textit{CMake 3.27+}.
\url{https://cmake.org/}

\bibitem{cpp17}
ISO/IEC. (2017).
\textit{ISO/IEC 14882:2017 – Programming Language C++}.

\bibitem{json}
nlohmann. (2023).
\textit{JSON for Modern C++}.
GitHub: \url{https://github.com/nlohmann/json}

\end{thebibliography}

% ============================================================================
% END OF DOCUMENT
% ============================================================================

\end{document}
