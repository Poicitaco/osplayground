% ============================================================================
% CHƯƠNG 0: MỤC LỤC VÀ TÓM TẮT
% ============================================================================

% ============================================================================
% Mục lục
% ============================================================================
{\hypersetup{linkcolor=black}
\pagenumbering{roman}
\setcounter{page}{1}

\tableofcontents
\clearpage

\listoffigures
\clearpage

\listoftables
\clearpage
}

% ============================================================================
% TÓM TẮT (Abstract)
% ============================================================================
\chapter*{Tóm tắt}
\addcontentsline{toc}{chapter}{Tóm tắt}

\textbf{OSPlayground} là dự án mô phỏng toàn diện các khái niệm cốt lõi của hệ điều hành (Operating System), 
được xây dựng dưới dạng các "mini-worlds" tương tác. Dự án cung cấp công cụ học tập trực quan hoá 
các thuật toán phức tạp của OS thông qua các giao diện người dùng đa dạng.

\subsection*{Phạm vi chính}

\begin{itemize}[leftmargin=1.5cm]
  \item \textbf{Lập lịch CPU (CPU Scheduling):} 
  Mô phỏng 6 thuật toán chính: FCFS (First Come First Serve), SJF (Shortest Job First), 
  SRTF (Shortest Remaining Time First), RR (Round Robin), Priority Non-preemptive, Priority Preemptive. 
  Các thuật toán được trực quan hoá bằng biểu đồ Gantt và tính toán các chỉ số hiệu năng (AWT, ATT).

  \item \textbf{Quản lý bộ nhớ (Memory Management):} 
  Mô phỏng các kỹ thuật quản lý bộ nhớ ảo bao gồm Paging và Segmentation. 
  Triển khai ba chính sách thay trang (Page Replacement): FIFO (First In First Out), 
  LRU (Least Recently Used), và OPT (Optimal). Hiển thị trạng thái khung trang theo thời gian thực.

  \item \textbf{Giao tiếp liên tiến trình (IPC - Inter-Process Communication):} 
  Mô phỏng Pipe (ống dẫn), Signal (tín hiệu) bao gồm SIGINT, SIGUSR1, SIGKILL. 
  Quan sát tác động của các tín hiệu lên trạng thái và điểm số hệ thống.

  \item \textbf{Shell mini:} 
  Triển khai shell dòng lệnh với hỗ trợ pipeline, cho phép kết hợp các lệnh 
  (\texttt{ps}, \texttt{kill}, \texttt{cpu}, \texttt{mem}, \texttt{grep}, \texttt{wc}) 
  để tương tác với mô phỏng hệ thống.

  \item \textbf{Giao diện người dùng:} 
  Ba lớp giao diện:
  \begin{enumerate}[nosep]
    \item CLI (Command Line Interface) cơ bản
    \item TUI (Text User Interface) - Dashboard tương tác
    \item GUI (Graphical User Interface) sử dụng SDL2 và Dear ImGui với biểu đồ Gantt, lưới khung trang
  \end{enumerate}

  \item \textbf{Hệ thống điểm tổng hợp (System Stability Score):} 
  Đánh giá toàn bộ hiệu năng hệ thống thông qua công thức tích hợp, kết hợp 
  các chỉ số từ CPU, Memory, và IPC với trọng số thích hợp.
\end{itemize}

\subsection*{Công nghệ sử dụng}

\begin{itemize}[leftmargin=1.5cm]
  \item \textbf{Ngôn ngữ:} C++17 (tiêu chuẩn C++ năm 2017)
  \item \textbf{Build system:} CMake 3.20+
  \item \textbf{Giao diện:} SDL2, Dear ImGui (Dear ImGui v1.89+)
  \item \textbf{Logging:} Hệ thống log tùy chỉnh với timestamp
  \item \textbf{Cấu hình:} JSON (libcurl/nlohmann json)
  \item \textbf{Platform:} Windows (MSVC), Linux/WSL (GCC/Clang), macOS (Clang)
\end{itemize}

\subsection*{Mục tiêu học tập}

Dự án hướng đến việc giúp sinh viên, nhà giáo dục, và những người yêu thích hệ điều hành:
\begin{enumerate}[leftmargin=1.5cm]
  \item Hiểu sâu các thuật toán lập lịch CPU qua trực quan hoá Gantt chart
  \item Nắm bắt được ảnh hưởng của các chỉ số (AWT, ATT) đối với hiệu năng
  \item Quan sát các chính sách thay trang và tương tác giữa bộ nhớ ảo và vật lý
  \item Thực hành giao tiếp liên tiến trình thông qua pipeline và signal
  \item Tổng hợp kiến thức thông qua hệ điểm toàn diện

\end{enumerate}

\subsection*{Kết quả chính}

\begin{itemize}[leftmargin=1.5cm]
  \item Mô-đun CPU với 6 thuật toán + Gantt chart (ASCII và GUI)
  \item Mô-đun Memory với 3 chính sách thay trang + lưới khung trang tương tác
  \item Mô-đun IPC với Pipe và 3 loại Signal
  \item Shell mini hỗ trợ 8+ lệnh và pipeline
  \item GUI đầy đủ với 5 panel chính
  \item Hệ thống điểm tổng hợp và ghi log chi tiết
  \item Hỗ trợ kịch bản JSON để tái lập các thử nghiệm
\end{itemize}

\subsection*{Cấu trúc báo cáo}

\begin{description}[leftmargin=2cm,style=nextline]
  \item[\textbf{Chương 1: Giới thiệu}] Động cơ, mục tiêu, đóng góp chính của dự án
  \item[\textbf{Chương 2: Cơ sở lý thuyết}] Kiến thức OS nền tảng, các khái niệm cốt lõi
  \item[\textbf{Chương 3: Yêu cầu và phạm vi}] Thông số kỹ thuật, giới hạn, ràng buộc
  \item[\textbf{Chương 4: Kiến trúc hệ thống}] Thiết kế tổng thể, mô-đun, flow dữ liệu
  \item[\textbf{Chương 5: Chi tiết CPU Scheduling}] Thiết kế, thuật toán, cài đặt
  \item[\textbf{Chương 6: Chi tiết Memory Management}] Dịch địa chỉ, thay trang, trực quan
  \item[\textbf{Chương 7: Chi tiết IPC và Shell}] Pipe, Signal, lệnh shell, pipeline
  \item[\textbf{Chương 8: Hệ thống điểm và đánh giá}] Công thức, chỉ số, phương pháp
  \item[\textbf{Chương 9: Giao diện người dùng}] CLI, TUI, GUI - thiết kế và tính năng
  \item[\textbf{Chương 10: Hướng dẫn sử dụng}] Build, chạy, flow demo, thao tác
  \item[\textbf{Chương 11: Kế hoạch kiểm thử}] Unit test, functional test, chấp nhận
  \item[\textbf{Chương 12: Kết quả thực nghiệm}] Dữ liệu, biểu đồ, so sánh hiệu năng
  \item[\textbf{Chương 13: Kết luận}] Tổng kết, hướng phát triển, bài học
\end{description}

\clearpage

% ============================================================================
% LỜI CẢM ƠN
% ============================================================================
\chapter*{Lời cảm ơn}
\addcontentsline{toc}{chapter}{Lời cảm ơn}

Xin chân thành cảm ơn:

\begin{itemize}[leftmargin=1.5cm]
  \item Thầy/cô giảng viên hướng dẫn đã tạo điều kiện, hướng dẫn tận tình trong suốt quá trình thực hiện dự án
  \item Các bạn cùng lớp, cùng nhóm đã hỗ trợ, góp ý và cung cấp feedback quý báu
  \item Đội ngũ Open Source tại Dear ImGui, SDL2 và cộng đồng C++ vì những thư viện tuyệt vời
  \item Gia đình, bạn bè đã khuyến khích, động viên trong quá trình học tập và phát triển
\end{itemize}

Hy vọng báo cáo này sẽ có ích cho những ai quan tâm đến Hệ điều hành.

\clearpage
