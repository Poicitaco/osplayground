% ============================================================================
% CHƯƠNG 1: GIỚI THIỆU
% ============================================================================

\chapter{Giới thiệu}
\label{ch:introduction}

\section{Động cơ và lý do thực hiện}

Hệ điều hành (Operating System) là một trong những môn học cơ bản và quan trọng nhất 
trong chương trình đào tạo Khoa học máy tính và Công nghệ thông tin. Các khái niệm như 
lập lịch CPU, quản lý bộ nhớ, giao tiếp liên tiến trình là những nền tảng không thể thiếu 
để hiểu rõ cách máy tính hoạt động.

Tuy nhiên, các khái niệm này thường được trình bày một cách trừu tượng trong giáo trình lý thuyết, 
gây khó khăn cho sinh viên trong việc hình dung và thực hành. Một số thách thức chính:

\begin{itemize}[leftmargin=1.5cm]
  \item \textbf{Tính trừu tượng cao:} Các thuật toán lập lịch CPU, thay trang bộ nhớ 
  thường chỉ được giải thích qua ví dụ tính tay trên giấy, không có công cụ tương tác để thử nghiệm.
  
  \item \textbf{Khó khăn trong hình dung:} Sinh viên khó tưởng tượng được cách các tiến trình 
  chuyển đổi, bộ nhớ được chia sẻ, hoặc các tín hiệu được gửi trong hệ thống thực.
  
  \item \textbf{Thiếu công cụ học tập tương tác:} Không có nền tảng cho phép học sinh 
  "chơi" với các khái niệm này một cách trực quan.
  
  \item \textbf{Khó so sánh:} Để so sánh hiệu năng của các thuật toán khác nhau 
  trên cùng bộ dữ liệu đòi hỏi tính toán thủ công, tốn thời gian.
\end{itemize}

\textbf{OSPlayground} được sinh ra với mục đích giải quyết những thách thức này. 
Nó cung cấp một môi trường mô phỏng hoàn chỉnh nơi các khái niệm OS không còn là những 
ký hiệu trừu tượng mà trở thành những quá trình tương tác có thể quan sát, 
thử nghiệm, và học hỏi trực tiếp.

\section{Mục tiêu dự án}

Mục tiêu chính của OSPlayground có thể được phân thành ba cấp độ:

\subsection*{1. Mục tiêu giáo dục (Educational Goals)}

\begin{enumerate}[leftmargin=1.5cm]
  \item Trực quan hoá các thuật toán lập lịch CPU qua biểu đồ Gantt động, 
  giúp sinh viên thấy rõ cách các tiến trình được xếp lịch trên CPU.
  
  \item Giáo dục về ảnh hưởng của các chỉ số hiệu năng (Average Waiting Time - AWT, 
  Average Turnaround Time - ATT) trên tính công bằng và hiệu quả của lập lịch.
  
  \item Minh hoạ các chính sách thay trang bộ nhớ (FIFO, LRU, OPT) 
  qua lưới khung trang tương tác, cho phép theo dõi page fault từng bước.
  
  \item Thực hành giao tiếp liên tiến trình thông qua Pipe và Signal 
  trong một môi trường mô phỏng an toàn, kiểm soát.
  
  \item Học cách viết shell script với pipeline (command1 | command2 | command3) 
  trên bộ lệnh mô phỏng.
\end{enumerate}

\subsection*{2. Mục tiêu kỹ thuật (Technical Goals)}

\begin{enumerate}[leftmargin=1.5cm]
  \item Xây dựng một kiến trúc phần mềm mô-đun, dễ mở rộng, 
  trong đó mỗi khối (CPU, Memory, IPC) hoạt động độc lập nhưng có thể tích hợp.
  
  \item Cài đặt chính xác các thuật toán OS theo tiêu chuẩn, 
  đảm bảo kết quả tính toán khớp với lý thuyết.
  
  \item Tạo giao diện người dùng đa cấp (CLI → TUI → GUI) 
  để phục vụ các nhu cầu sử dụng khác nhau.
  
  \item Hỗ trợ tải kịch bản (Scenario) từ tệp JSON 
  để đảm bảo tính tái lập và lưu trữ các test case.
  
  \item Triển khai hệ thống ghi log chi tiết 
  để hỗ trợ debug, verify, và tạo báo cáo tự động.
\end{enumerate}

\subsection*{3. Mục tiêu phát triển (Development Goals)}

\begin{enumerate}[leftmargin=1.5cm]
  \item Xây dựng dự án C++ tiêu chuẩn, 
  sử dụng C++17, CMake, không phụ thuộc vào các framework nặng.
  
  \item Đảm bảo tính cross-platform: 
  chạy được trên Windows (MSVC), Linux/WSL (GCC/Clang), macOS (Clang).
  
  \item Áp dụng các best practices: 
  SOLID principles, design patterns, unit testing, documentation.
  
  \item Tạo một nền tảng có thể mở rộng cho các tính năng tương lai 
  (networking simulation, virtual machine, advanced scheduling, ...).
\end{enumerate}

\section{Đóng góp chính của dự án}

\subsection*{1. Bộ thuật toán CPU Scheduling}

Dự án cài đặt 6 thuật toán lập lịch CPU chuẩn:

\begin{itemize}[leftmargin=1.5cm]
  \item \textbf{FCFS (First Come First Serve)}: Đơn giản, không tiền nhiệm
  \item \textbf{SJF (Shortest Job First)}: Tối ưu hoá AWT, không tiền nhiệm
  \item \textbf{SRTF (Shortest Remaining Time First)}: Phiên bản tiền nhiệm của SJF
  \item \textbf{RR (Round Robin)}: Công bằng thời gian, tiền nhiệm, có quantum
  \item \textbf{Priority Non-preemptive}: Ưu tiên, không thể ngắt
  \item \textbf{Priority Preemptive}: Ưu tiên, có thể ngắt khi công việc ưu tiên cao đến
\end{itemize}

Mỗi thuật toán cung cấp:
\begin{itemize}[leftmargin=1.5cm]
  \item Biểu đồ Gantt (ASCII trong CLI, vector graphic trong GUI)
  \item Tính toán AWT, ATT chính xác
  \item Log chi tiết từng bước thực thi
  \item So sánh performance trên cùng dataset
\end{itemize}

\subsection*{2. Bộ quản lý bộ nhớ}

Dự án triển khai:

\begin{itemize}[leftmargin=1.5cm]
  \item \textbf{Paging}: Chia không gian nhớ logic thành trang, vật lý thành khung
  \item \textbf{Segmentation}: Chia theo đoạn (segment)
  \item \textbf{Dịch địa chỉ}: Ánh xạ logic → physical với kiểm tra biên
  \item \textbf{3 chính sách thay trang}: FIFO, LRU, OPT
  \item \textbf{Thống kê}: Page fault count, fault rate, frame states theo thời gian
  \item \textbf{Trực quan}: Lưới khung trang (frame grid) tương tác theo thời gian
\end{itemize}

\subsection*{3. Mô-đun IPC và Signal}

\begin{itemize}[leftmargin=1.5cm]
  \item \textbf{Pipe}: Hàng đợi FIFO truyền chuỗi giữa tiến trình
  \item \textbf{Signal}: Hỗ trợ SIGINT, SIGUSR1, SIGKILL
  \item \textbf{Quản lý tín hiệu}: Xử lý, log sự kiện
  \item \textbf{Ảnh hưởng tới điểm}: Các signal ảnh hưởng tới điểm số hệ thống
\end{itemize}

\subsection*{4. Shell mini với Pipeline}

\begin{itemize}[leftmargin=1.5cm]
  \item \textbf{8+ lệnh}: ps, kill, sig, cpu, mem, cat, grep, wc, echo, ls
  \item \textbf{Pipeline}: Kết nối lệnh bằng \texttt{|}
  \item \textbf{Lệnh tương tác}: Có thể gửi signal, chạy thuật toán từ shell
  \item \textbf{Output processing}: Xử lý và biến đổi output qua pipeline
\end{itemize}

\subsection*{5. Giao diện đa cấp}

\begin{itemize}[leftmargin=1.5cm]
  \item \textbf{CLI}: Giao diện dòng lệnh cơ bản, nhanh gọn
  \item \textbf{TUI}: Dashboard tương tác với menu, menu con, trạng thái tức thời
  \item \textbf{GUI}: Giao diện đồ hoạ SDL2/ImGui với:
  \begin{itemize}
    \item Biểu đồ Gantt động (CPU panel)
    \item Lưới khung trang tương tác (Memory panel)
    \item Panel IPC, Scenarios, Score
    \item Dark theme, responsive layout, font system
  \end{itemize}
\end{itemize}

\subsection*{6. Hệ thống điểm tổng hợp}

\begin{itemize}[leftmargin=1.5cm]
  \item \textbf{CPU Score}: Dựa trên AWT, ATT (công thức chuẩn hoá)
  \item \textbf{Memory Score}: Dựa trên fault rate
  \item \textbf{IPC Score}: Dựa trên số lần kill/signal
  \item \textbf{System Stability Score}: Trung bình có trọng số của 3 điểm trên
\end{itemize}

\section{Phạm vi và giới hạn}

\subsection*{Phạm vi}

Dự án tập trung vào các khái niệm OS cốt lõi được dạy trong khóa học Hệ điều hành tiêu chuẩn:

\begin{itemize}[leftmargin=1.5cm]
  \item Lập lịch CPU (single-core model)
  \item Quản lý bộ nhớ ảo (virtual memory basics)
  \item IPC cơ bản (pipes, signals)
  \item Shell scripting cơ bản
\end{itemize}

\subsection*{Giới hạn}

Để giữ dự án có kích thước hợp lý và tập trung, các khía cạnh sau được ngoài phạm vi:

\begin{itemize}[leftmargin=1.5cm]
  \item \textbf{Không phải kernel thực}: Đây là mô phỏng user-space, không bao gồm bootloader, 
  interrupt handler, MMU driver thật
  \item \textbf{Không hỗ trợ multi-core}: Mô hình một lõi CPU duy nhất
  \item \textbf{Không có device I/O}: Không mô phỏng disk, network, thiết bị ngoại vi
  \item \textbf{Không hỗ trợ thread}: Chỉ mô phỏng tiến trình (process), không thread
  \item \textbf{Không có file system}: Không triển khai filesystem tương tác
  \item \textbf{Không hỗ trợ network}: Không mô phỏng network stack
\end{itemize}

\section{Cấu trúc báo cáo}

Báo cáo được tổ chức thành 13 chương chính:

\begin{description}[leftmargin=2cm,style=nextline]
  \item[\textbf{Chương 1 - Giới thiệu}] Động cơ, mục tiêu, đóng góp
  \item[\textbf{Chương 2 - Cơ sở lý thuyết}] Kiến thức nền tảng, công thức toán học
  \item[\textbf{Chương 3 - Yêu cầu và phạm vi}] Thông số kỹ thuật, ràng buộc
  \item[\textbf{Chương 4 - Kiến trúc}] Thiết kế tổng thể, flow dữ liệu
  \item[\textbf{Chương 5 - Chi tiết CPU}] Thuật toán, cài đặt, Gantt
  \item[\textbf{Chương 6 - Chi tiết Memory}] Dịch địa chỉ, thay trang
  \item[\textbf{Chương 7 - Chi tiết IPC}] Pipe, Signal, Shell
  \item[\textbf{Chương 8 - Hệ thống điểm}] Công thức, phương pháp đánh giá
  \item[\textbf{Chương 9 - Giao diện}] CLI, TUI, GUI
  \item[\textbf{Chương 10 - Hướng dẫn sử dụng}] Build, chạy, flow
  \item[\textbf{Chương 11 - Kiểm thử}] Unit test, functional test
  \item[\textbf{Chương 12 - Kết quả}] Dữ liệu, biểu đồ, so sánh
  \item[\textbf{Chương 13 - Kết luận}] Tổng kết, hướng phát triển
\end{description}

\clearpage
