% ============================================================================
% CHƯƠNG 4: KIẾN TRÚC HỆ THỐNG
% ============================================================================

\chapter{Kiến trúc hệ thống}
\label{ch:architecture}

\section{Tổng quan kiến trúc}

OSPlayground được thiết kế theo mô hình \textbf{mô-đun hóa (modular architecture)}, 
trong đó mỗi khối chức năng độc lập nhưng liên kết với nhau qua interface rõ ràng.

\subsection{Sơ đồ kiến trúc cấp cao}

\begin{figure}[H]
\centering
\begin{tikzpicture}[>=Stealth, node distance=1cm, every node/.style={draw, rounded corners, align=center, minimum width=3cm, minimum height=1cm}]
  % Layer 1 - UI
  \node[fill=blue!20] (cli) {CLI\\Engine};
  \node[fill=blue!20, right=1.5cm of cli] (tui) {TUI\\Dashboard};
  \node[fill=blue!20, right=1.5cm of tui] (gui) {GUI\\SDL2/ImGui};
  
  % Layer 2 - Core Modules
  \node[fill=green!20, below=2cm of cli] (cpu_mod) {CPU\\Scheduler};
  \node[fill=green!20, right=1.5cm of cpu_mod] (mem_mod) {Memory\\Manager};
  \node[fill=green!20, right=1.5cm of mem_mod] (ipc_mod) {IPC\\Pipe/Signal};
  \node[fill=green!20, right=1.5cm of ipc_mod] (shell_mod) {Shell\\Parser};
  
  % Layer 3 - Core System
  \node[fill=yellow!20, below=2cm of mem_mod, minimum width=12cm] (core_sys) {Core System (CPU/MEM/IPC Stats, System State)};
  
  % Layer 4 - Utils
  \node[fill=gray!20, below=1.5cm of core_sys, minimum width=12cm] (utils) {Utils (Logging, File I/O, JSON Parsing)};
  
  % Connections from UI to Modules
  \draw[->] (cli) -- (cpu_mod);
  \draw[->] (tui) -- (mem_mod);
  \draw[->] (gui) -- (ipc_mod);
  \draw[->] (gui) -- (shell_mod);
  
  % Connections from Modules to Core
  \draw[->] (cpu_mod) -- (core_sys);
  \draw[->] (mem_mod) -- (core_sys);
  \draw[->] (ipc_mod) -- (core_sys);
  \draw[->] (shell_mod) -- (core_sys);
  
  % Connections from Core to Utils
  \draw[->] (core_sys) -- (utils);
  
  % Back connections
  \draw[<-] (shell_mod) -- (core_sys);
  
\end{tikzpicture}
\caption{Sơ đồ kiến trúc cấp cao OSPlayground}
\label{fig:arch_highlevel}
\end{figure}

\section{Cấu trúc thư mục dự án}

\begin{figure}[H]
\centering
\begin{tikzpicture}[font=\ttfamily\small]
\node[anchor=west] (root) at (0,0) {OSPlayground/};
\node[anchor=west] at (1,-0.5) {├── CMakeLists.txt};
\node[anchor=west] at (1,-1) {├── main.cpp};
\node[anchor=west] at (1,-1.5) {├── assets/};
\node[anchor=west] at (2,-2) {│   ├── scenarios/};
\node[anchor=west] at (3,-2.4) {│   │   └── demo\_cpu.json};
\node[anchor=west] at (2,-2.8) {│   └── logs/};
\node[anchor=west] at (1,-3.2) {├── engine/};
\node[anchor=west] at (2,-3.6) {│   ├── loop.cpp, loop.h};
\node[anchor=west] at (1,-4) {├── sim/};
\node[anchor=west] at (2,-4.4) {│   ├── cpu/ (Schedulers)};
\node[anchor=west] at (2,-4.8) {│   ├── mem/ (Paging/Segmentation)};
\node[anchor=west] at (2,-5.2) {│   ├── ipc/ (Pipe/Signal)};
\node[anchor=west] at (2,-5.6) {│   ├── shell/ (Shell Parser)};
\node[anchor=west] at (2,-6) {│   └── core/ (System State)};
\node[anchor=west] at (1,-6.4) {├── ui/};
\node[anchor=west] at (2,-6.8) {│   ├── cli/ (CLI Interface)};
\node[anchor=west] at (2,-7.2) {│   ├── tui/ (TUI Dashboard)};
\node[anchor=west] at (2,-7.6) {│   └── gfx/ (GUI SDL2/ImGui)};
\node[anchor=west] at (1,-8) {├── utils/};
\node[anchor=west] at (2,-8.4) {│   ├── log.cpp, log.h};
\node[anchor=west] at (2,-8.8) {│   └── file.cpp, file.h};
\node[anchor=west] at (1,-9.2) {├── tests/};
\node[anchor=west] at (2,-9.6) {│   ├── cpu\_test.cpp};
\node[anchor=west] at (2,-10) {│   ├── mem\_test.cpp};
\node[anchor=west] at (2,-10.4) {│   └── ipc\_test.cpp};
\node[anchor=west] at (1,-10.8) {└── docs/};
\node[anchor=west] at (2,-11.2) {    └── report/ (Báo cáo này)};
\end{tikzpicture}
\caption{Cấu trúc thư mục dự án}
\label{fig:directory_structure}
\end{figure}

\section{Các mô-đun chính}

\subsection{1. Mô-đul CPU Scheduler (\texttt{sim/cpu/})}

\textbf{Trách nhiệm}: Cài đặt các thuật toán lập lịch CPU.

\textbf{Giao diện chính}:
\begin{lstlisting}[language=C++,caption={CPU Scheduler API}]
class Process {
  int pid, arrival, burst, priority;
  int wait_time, turnaround_time, start, finish;
};

struct GanttEntry { int pid; int start; int finish; };

class Scheduler {
  Result runFCFS(const vector<Process>& procs);
  Result runSJF(const vector<Process>& procs);
  Result runSRTF(const vector<Process>& procs);
  Result runRR(const vector<Process>& procs, int quantum);
  Result runPriorityNP(const vector<Process>& procs);
  Result runPriorityP(const vector<Process>& procs);
  
  vector<GanttEntry> getGantt() const;
  double getAWT() const;
  double getATT() const;
};

struct Result {
  double awt, att;
  vector<GanttEntry> gantt;
  vector<string> log;  // Chi tiết từng bước
};
\end{lstlisting}

\subsection{2. Mô-đul Memory Manager (\texttt{sim/mem/})}

\textbf{Trách nhiệm}: Mô phỏng quản lý bộ nhớ (paging/segmentation, thay trang).

\textbf{Giao diện chính}:
\begin{lstlisting}[language=C++,caption={Memory Manager API}]
class MMU {  // Memory Management Unit
  PhysicalAddress translate(LogicalAddress logical);
  bool isValidAddress(LogicalAddress addr) const;
};

class PageReplacer {
  void loadPage(int page, int physicalFrame);
  int selectVictim(int currentTime);  // Tuỳ theo chính sách
};

class MemorySimulator {
  Result runFIFO(int frames, vector<int> references);
  Result runLRU(int frames, vector<int> references);
  Result runOPT(int frames, vector<int> references);
  
  int getFaultCount() const;
  double getFaultRate() const;
  const vector<vector<int>>& getFrameStates() const;
};

struct Result {
  int faults, steps;
  vector<vector<int>> frame_states;  // [step][frame] = page_id
  vector<string> log;
};
\end{lstlisting}

\subsection{3. Mô-đul IPC (\texttt{sim/ipc/})}

\textbf{Trách nhiệm}: Mô phỏng giao tiếp liên tiến trình (Pipe, Signal).

\textbf{Giao diện chính}:
\begin{lstlisting}[language=C++,caption={IPC API}]
class Pipe {
  void write(const string& data);
  string read(int count);
  bool isEmpty() const;
  bool isFull() const;
};

enum Signal { SIGINT, SIGUSR1, SIGKILL };

class SignalHandler {
  void sendSignal(int pid, Signal sig);
  void handleSignal(int pid, Signal sig);
};

class IPCManager {
  void createPipe(int id);
  void sendSignal(int pid, Signal sig);
  const vector<SignalEvent>& getEvents() const;
};

struct SignalEvent {
  int pid, time;
  Signal sig;
  bool handled;
};
\end{lstlisting}

\subsection{4. Mô-đul Shell (\texttt{sim/shell/})}

\textbf{Trách nhiệm}: Phân tích cú pháp dòng lệnh và thực thi lệnh.

\textbf{Giao diện chính}:
\begin{lstlisting}[language=C++,caption={Shell API}]
class ShellParser {
  vector<Command> parseLine(const string& line);
  bool isValidPipeline(const vector<Command>& cmds) const;
};

class ShellExecutor {
  vector<string> executePipeline(const vector<Command>& cmds);
  vector<string> executeCommand(const Command& cmd);
};

struct Command {
  string name;  // "ps", "cpu", "kill", ...
  vector<string> args;
};
\end{lstlisting}

\subsection{5. Mô-đul Core System (\texttt{sim/core/})}

\textbf{Trách nhiệm}: Quản lý trạng thái hệ thống toàn cục.

\textbf{Giao diện chính}:
\begin{lstlisting}[language=C++,caption={Core System API}]
class SystemState {
  // Process management
  int createProcess(const Process& p);
  void killProcess(int pid);
  vector<Process> getAllProcesses() const;
  
  // Scoring
  double computeCPUScore() const;
  double computeMemoryScore() const;
  double computeIPCScore() const;
  double computeSystemScore() const;  // Weighted average
  
  // State
  const Scheduler& getCPUScheduler() const;
  const MemorySimulator& getMemory() const;
  const IPCManager& getIPC() const;
};
\end{lstlisting}

\section{Luồng dữ liệu}

\subsection{Luồng chạy CPU Scheduling}

\begin{enumerate}[leftmargin=1.5cm]
  \item \textbf{Input}: Danh sách tiến trình từ kịch bản JSON
  \item \textbf{Parsing}: CLI/GUI đọc JSON, tạo objects Process
  \item \textbf{Execution}: Scheduler.run*() xử lý, tạo Gantt, tính AWT/ATT
  \item \textbf{Output}: Kết quả được lưu vào SystemState
  \item \textbf{Visualization}: GUI vẽ Gantt chart, hiển thị chỉ số
  \item \textbf{Logging}: Tất cả chi tiết được ghi vào log
\end{enumerate}

\subsection{Luồng chạy Memory Simulation}

\begin{enumerate}[leftmargin=1.5cm]
  \item \textbf{Input}: Số khung (frames) và reference string từ JSON
  \item \textbf{Parsing}: Chuyển đổi input thành dạng xử lý được
  \item \textbf{Simulation}: MemorySimulator chạy chính sách thay trang, tính fault
  \item \textbf{State Tracking}: Lưu trạng thái khung ở mỗi bước
  \item \textbf{Output}: Fault count, frame\_states, log
  \item \textbf{Visualization}: GUI vẽ lưới khung trang
\end{enumerate}

\subsection{Luồng xử lý Shell Command}

\begin{enumerate}[leftmargin=1.5cm]
  \item \textbf{Input}: Người dùng gõ lệnh vào CLI
  \item \textbf{Parsing}: ShellParser tách lệnh và argument
  \item \textbf{Validation}: Kiểm tra cú pháp, số lượng argument
  \item \textbf{Execution}: ShellExecutor gọi hàm tương ứng
  \item \textbf{Pipeline}: Nếu có \verb+|+, truyền output qua các lệnh
  \item \textbf{Output}: In kết quả lên CLI
\end{enumerate}

\section{Các design patterns sử dụng}

\subsection{1. Strategy Pattern}

Các thuật toán lập lịch được triển khai dưới dạng các strategy khác nhau:

\begin{itemize}[leftmargin=1.5cm]
  \item Giao diện chung: \texttt{ISchedulingStrategy}
  \item Implementations: \texttt{FCFSStrategy}, \texttt{SJFStrategy}, ...
  \item Context: \texttt{Scheduler} chọn strategy theo parameter
\end{itemize}

\subsection{2. Factory Pattern}

Tạo đối tượng Scheduler, MemorySimulator, IPCManager:

\begin{itemize}[leftmargin=1.5cm]
  \item \texttt{SchedulerFactory::create(type, params)}
  \item \texttt{MemoryFactory::create(type, frames)}
\end{itemize}

\subsection{3. Observer Pattern}

GUI theo dõi sự thay đổi của SystemState:

\begin{itemize}[leftmargin=1.5cm]
  \item SystemState phát ra events
  \item GUI panels đăng ký nhận events
  \item Khi dữ liệu thay đổi, panels tự động cập nhật
\end{itemize}

\subsection{4. Command Pattern}

Shell commands được biểu diễn dưới dạng Command objects:

\begin{itemize}[leftmargin=1.5cm]
  \item \texttt{ICommand} interface
  \item Implementations: \texttt{PSCommand}, \texttt{CPUCommand}, ...
  \item \texttt{CommandInvoker} thực thi commands
\end{itemize}

\section{Giao tiếp giữa các mô-đul}

\subsection{Từ CPU sang Core System}

\begin{itemize}[leftmargin=1.5cm]
  \item Scheduler.run*() hoàn thành → kết quả
  \item SystemState::setCPUResult() lưu kết quả
  \item Tính CPU Score dựa trên AWT/ATT
\end{itemize}

\subsection{Từ Memory sang Core System}

\begin{itemize}[leftmargin=1.5cm]
  \item MemorySimulator.run*() hoàn thành → frame\_states, faults
  \item SystemState::setMemoryResult() lưu kết quả
  \item Tính Memory Score dựa trên fault rate
\end{itemize}

\subsection{Từ IPC sang Core System}

\begin{itemize}[leftmargin=1.5cm]
  \item IPCManager ghi nhận signal events
  \item SystemState theo dõi events
  \item Tính IPC Score dựa trên số kills
\end{itemize}

\subsection{Từ Shell sang Core System}

\begin{itemize}[leftmargin=1.5cm]
  \item Shell parser các lệnh, cập nhật SystemState trực tiếp
  \item Ví dụ: \verb|kill 5| → gọi SystemState::killProcess(5)
  \item Ví dụ: \verb|cpu srtf| → gọi Scheduler.runSRTF()
\end{itemize}

\section{Công thức hệ thống điểm}

Hệ thống tính điểm dựa trên ba thành phần:

\begin{equation}
\text{CPU Score} = 100 \cdot \frac{B}{B + (\text{AWT} + \text{ATT})}, \quad B = 20
\end{equation}

\begin{equation}
\text{Memory Score} = 100 \times \left(1 - \frac{\text{faults}}{\text{steps}}\right)
\end{equation}

\begin{equation}
\text{IPC Score} = \max\left(0, 100 - 15 \times \text{kills}\right)
\end{equation}

\begin{equation}
\text{System Stability Score} = 0.4 \times \text{CPU} + 0.4 \times \text{Memory} + 0.2 \times \text{IPC}
\end{equation}

\clearpage
