% ============================================================================
% CHƯƠNG 5: CHI TIẾT CPU SCHEDULING
% ============================================================================

\chapter{Thiết kế chi tiết - CPU Scheduling}
\label{ch:cpu_detail}

\section{Mô hình dữ liệu}

\subsection{Struct Process}

\begin{lstlisting}[language=C++,caption={Cấu trúc Process}]
struct Process {
  int pid;              // Process ID (1-based)
  int arrival;          // Thời điểm đến CPU ready queue
  int burst;            // Thời gian CPU cần
  int priority;         // Mức ưu tiên (1-40, nhỏ hơn = cao hơn)
  
  // Kết quả tính toán
  int start_time;       // Lúc bắt đầu chạy
  int finish_time;      // Lúc hoàn thành
  int wait_time;        // start_time - arrival
  int turnaround_time;  // finish_time - arrival
  
  // Trạng thái
  ProcessState state;   // New, Ready, Running, Terminated
  int remaining_time;   // Thời gian còn lại (cho preemptive)
};
\end{lstlisting}

\subsection{Struct GanttEntry và Result}

\begin{lstlisting}[language=C++,caption={Gantt Chart và Result}]
struct GanttEntry {
  int pid;       // Process ID
  int start;     // Start time
  int finish;    // Finish time
  
  GanttEntry(int p, int s, int f) 
    : pid(p), start(s), finish(f) {}
};

struct Result {
  double awt;                    // Average Waiting Time
  double att;                    // Average Turnaround Time
  vector<GanttEntry> gantt;      // Biểu đồ Gantt
  vector<Process> processes;     // Danh sách tiến trình có kết quả
  vector<string> execution_log;  // Log chi tiết từng bước
  int total_time;                // Tổng thời gian simulation
  int cpu_idle_time;             // Thời gian CPU rỗi
  double cpu_utilization;        // % sử dụng CPU
};
\end{lstlisting}

\section{FCFS - First Come First Serve}

\subsection{Nguyên lý}

FCFS sắp xếp các tiến trình theo thứ tự arrival time. Tiến trình nào đến trước phục vụ trước.

\textbf{Tính chất}:
\begin{itemize}[leftmargin=1.5cm]
  \item Non-preemptive: Không ngắt tiến trình đang chạy
  \item FIFO: Kiểu hàng đợi tự nhiên
  \item Dễ cài đặt nhưng dễ gây Convoy Effect
\end{itemize}

\subsection{Giả mã và cài đặt}

\begin{lstlisting}[language=C++,caption={Cài đặt FCFS}]
Result Scheduler::runFCFS(const vector<Process>& procs) {
  vector<Process> processes = procs;
  
  // Sắp xếp theo arrival time, rồi PID
  sort(processes.begin(), processes.end(), 
    [](const Process& a, const Process& b) {
      if (a.arrival != b.arrival) return a.arrival < b.arrival;
      return a.pid < b.pid;
    });
  
  Result result;
  result.gantt.clear();
  
  int current_time = 0;
  double sum_awt = 0, sum_att = 0;
  
  for (auto& p : processes) {
    // Nếu CPU rỗi, chờ tới arrival time tiếp theo
    if (current_time < p.arrival) {
      result.execution_log.push_back(
        "CPU idle from " + to_string(current_time) 
        + " to " + to_string(p.arrival));
      current_time = p.arrival;
    }
    
    // Tiến trình p chạy
    p.start_time = current_time;
    p.finish_time = current_time + p.burst;
    p.wait_time = p.start_time - p.arrival;
    p.turnaround_time = p.finish_time - p.arrival;
    
    result.gantt.push_back(
      GanttEntry(p.pid, p.start_time, p.finish_time));
    
    result.execution_log.push_back(
      "P" + to_string(p.pid) + ": start=" + to_string(p.start_time)
      + ", finish=" + to_string(p.finish_time)
      + ", wait=" + to_string(p.wait_time));
    
    current_time = p.finish_time;
    sum_awt += p.wait_time;
    sum_att += p.turnaround_time;
  }
  
  result.awt = sum_awt / processes.size();
  result.att = sum_att / processes.size();
  result.processes = processes;
  result.total_time = current_time;
  
  return result;
}
\end{lstlisting}

\section{SJF - Shortest Job First}

\subsection{Nguyên lý}

Tại mỗi lần CPU rỗi, chọn tiến trình có burst time nhỏ nhất trong số những tiến trình 
đã đến. SJF tối ưu hoá AWT trong tất cả non-preemptive algorithms.

\subsection{Giả mã}

\begin{lstlisting}[language=C++,caption={Cài đặt SJF}]
Result Scheduler::runSJF(const vector<Process>& procs) {
  vector<Process> processes = procs;
  Result result;
  
  vector<bool> done(processes.size(), false);
  int current_time = 0;
  int completed = 0;
  
  while (completed < processes.size()) {
    // Tìm tiến trình chưa xong có burst nhỏ nhất, arrival <= current_time
    int next_idx = -1;
    int min_burst = INT_MAX;
    
    for (int i = 0; i < processes.size(); ++i) {
      if (!done[i] && processes[i].arrival <= current_time) {
        if (processes[i].burst < min_burst) {
          min_burst = processes[i].burst;
          next_idx = i;
        }
      }
    }
    
    if (next_idx == -1) {
      // Không có tiến trình sẵn sàng, CPU rỗi
      // Tìm arrival time sớm nhất tiếp theo
      int next_arrival = INT_MAX;
      for (int i = 0; i < processes.size(); ++i) {
        if (!done[i] && processes[i].arrival > current_time) {
          next_arrival = min(next_arrival, processes[i].arrival);
        }
      }
      if (next_arrival != INT_MAX) {
        result.execution_log.push_back(
          "CPU idle from " + to_string(current_time) 
          + " to " + to_string(next_arrival));
        current_time = next_arrival;
      }
    } else {
      // Chạy tiến trình next_idx
      Process& p = processes[next_idx];
      p.start_time = current_time;
      p.finish_time = current_time + p.burst;
      p.wait_time = p.start_time - p.arrival;
      p.turnaround_time = p.finish_time - p.arrival;
      
      result.gantt.push_back(
        GanttEntry(p.pid, p.start_time, p.finish_time));
      
      result.execution_log.push_back(
        "P" + to_string(p.pid) + ": burst=" + to_string(p.burst));
      
      current_time = p.finish_time;
      done[next_idx] = true;
      completed++;
    }
  }
  
  // Tính trung bình
  double sum_awt = 0, sum_att = 0;
  for (const auto& p : processes) {
    sum_awt += p.wait_time;
    sum_att += p.turnaround_time;
  }
  result.awt = sum_awt / processes.size();
  result.att = sum_att / processes.size();
  result.processes = processes;
  result.total_time = current_time;
  
  return result;
}
\end{lstlisting}

\section{SRTF - Shortest Remaining Time First}

\subsection{Nguyên lý}

Phiên bản preemptive của SJF. Tại mỗi tick thời gian (hoặc khi tiến trình mới đến), 
CPU chọn tiến trình có remaining time nhỏ nhất. Nếu tiến trình mới có remaining time 
nhỏ hơn tiến trình đang chạy, ngắt tiến trình hiện tại.

\subsection{Đặc điểm}

\begin{itemize}[leftmargin=1.5cm]
  \item Tối ưu hoá ATT trong tất cả preemptive algorithms
  \item Chi phí context switch cao nếu mô phỏng quá chi tiết (mỗi tick)
  \item Gây starvation cho tiến trình dài
  \item Yêu cầu biết trước burst time (không thực tế)
\end{itemize}

\section{Round Robin (RR)}

\subsection{Nguyên lý}

Mỗi tiến trình được cấp CPU một time slice (quantum) cố định. Nếu tiến trình không 
hoàn thành trong quantum, nó được đặt ở cuối hàng đợi sẵn sàng.

\begin{lstlisting}[language=C++,caption={Cài đặt Round Robin}]
Result Scheduler::runRR(const vector<Process>& procs, int quantum) {
  vector<Process> processes = procs;
  Result result;
  
  queue<int> ready_queue;
  vector<int> remaining_time(processes.size());
  
  for (int i = 0; i < processes.size(); ++i) {
    remaining_time[i] = processes[i].burst;
  }
  
  int current_time = 0;
  int completed = 0;
  
  // Đưa tất cả tiến trình arrive tại thời điểm 0 vào hàng đợi
  for (int i = 0; i < processes.size(); ++i) {
    if (processes[i].arrival == current_time) {
      ready_queue.push(i);
    }
  }
  
  while (completed < processes.size()) {
    if (ready_queue.empty()) {
      // Tìm tiến trình sớm nhất arrive
      int next_arrival = INT_MAX;
      int next_idx = -1;
      for (int i = 0; i < processes.size(); ++i) {
        if (processes[i].arrival > current_time && 
            remaining_time[i] > 0) {
          if (processes[i].arrival < next_arrival) {
            next_arrival = processes[i].arrival;
            next_idx = i;
          }
        }
      }
      if (next_idx != -1) {
        current_time = next_arrival;
        ready_queue.push(next_idx);
      }
    } else {
      int idx = ready_queue.front();
      ready_queue.pop();
      
      int time_slice = min(quantum, remaining_time[idx]);
      processes[idx].start_time = (processes[idx].start_time == 0) 
                                   ? current_time 
                                   : processes[idx].start_time;
      
      current_time += time_slice;
      remaining_time[idx] -= time_slice;
      
      // Đưa các tiến trình arrive vào hàng đợi
      for (int i = 0; i < processes.size(); ++i) {
        if (processes[i].arrival == current_time && 
            remaining_time[i] > 0 && i != idx) {
          ready_queue.push(i);
        }
      }
      
      if (remaining_time[idx] == 0) {
        processes[idx].finish_time = current_time;
        processes[idx].wait_time = processes[idx].finish_time 
                                  - processes[idx].arrival 
                                  - processes[idx].burst;
        processes[idx].turnaround_time = processes[idx].finish_time 
                                        - processes[idx].arrival;
        result.gantt.push_back(
          GanttEntry(processes[idx].pid, 
                     processes[idx].start_time, 
                     current_time));
        completed++;
      } else {
        result.gantt.push_back(
          GanttEntry(processes[idx].pid, 
                     current_time - time_slice, 
                     current_time));
        ready_queue.push(idx);
      }
    }
  }
  
  // Tính trung bình
  double sum_awt = 0, sum_att = 0;
  for (const auto& p : processes) {
    sum_awt += p.wait_time;
    sum_att += p.turnaround_time;
  }
  result.awt = sum_awt / processes.size();
  result.att = sum_att / processes.size();
  result.processes = processes;
  result.total_time = current_time;
  
  return result;
}
\end{lstlisting}

\subsection{Ảnh hưởng của quantum}

\begin{table}[H]
\centering
\caption{Ảnh hưởng của Quantum trên RR}
\begin{tabular}{lll}
\toprule
\textbf{Quantum} & \textbf{Ưu điểm} & \textbf{Nhược điểm} \\
\midrule
Quá nhỏ (1-2 units) & Response time tốt & Context switch overhead cao \\
Vừa phải (4-8 units) & Cân bằng & Phù hợp hầu hết trường hợp \\
Quá lớn (>100 units) & Overhead thấp & Giống FCFS hơn, response tệ \\
\bottomrule
\end{tabular}
\label{tab:quantum_effect}
\end{table}

\section{Priority Scheduling}

\subsection{Non-preemptive Priority}

\begin{itemize}[leftmargin=1.5cm]
  \item Mỗi tiến trình có priority number (1-40, nhỏ = cao)
  \item Chọn tiến trình ưu tiên cao nhất trong hàng đợi sẵn sàng
  \item Chạy cho đến kết thúc
\end{itemize}

\subsection{Preemptive Priority}

\begin{itemize}[leftmargin=1.5cm]
  \item Nếu tiến trình mới đến có priority cao hơn tiến trình đang chạy, ngắt
  \item Tiến trình bị ngắt quay lại hàng đợi sẵn sàng
\end{itemize}

\subsection{Vấn đề Starvation và Aging}

\textbf{Starvation}: Tiến trình ưu tiên thấp có thể chờ vô hạn.

\textbf{Giải pháp Aging}: Tăng dần priority của tiến trình theo thời gian chờ.

\begin{equation}
\text{current\_priority}(t) = \text{original\_priority} - \left\lfloor \frac{t}{k} \right\rfloor
\end{equation}

Với $k$ là hằng số aging factor (ví dụ: k=10).

\section{Vẽ Gantt Chart}

\subsection{ASCII Gantt (CLI)}

\begin{lstlisting}[language=C++,caption={Vẽ Gantt ASCII}]
void printGanttASCII(const vector<GanttEntry>& gantt) {
  // Vẽ timeline
  cout << "\nGantt Chart:\n";
  cout << " ";
  for (int t = 0; t <= getMaxTime(gantt); ++t) {
    cout << setw(3) << t;
  }
  cout << "\n";
  
  // Vẽ bar
  cout << " |";
  for (const auto& entry : gantt) {
    int width = entry.finish - entry.start;
    cout << setw(width) << ("P" + to_string(entry.pid)) << "|";
  }
  cout << "\n";
  
  // Vẽ time scale
  cout << " ";
  for (const auto& entry : gantt) {
    cout << entry.start << setw(entry.finish - entry.start) << entry.finish;
  }
  cout << "\n";
}
\end{lstlisting}

\subsection{Vector Graphic Gantt (GUI)}

GUI sử dụng ImGui để vẽ Gantt chart:
\begin{itemize}[leftmargin=1.5cm]
  \item X-axis: time
  \item Y-axis: process ID
  \item Mỗi entry là một hình chữ nhật có màu khác nhau
  \item Hover hiển thị chi tiết entry
  \item Zoom, pan, export hình ảnh
\end{itemize}

\clearpage
